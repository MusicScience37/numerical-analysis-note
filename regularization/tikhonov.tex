% !TEX root = ../main.tex
%

\chapter{Tikhonov 正則化}

ここでは基本的な Tikhonov 正則化についてまとめる.
Tikhonov 正則化では,評価関数
\begin{equation}
    E_{\lambda}(\bm{x}) \equiv \|A \bm{x} - \bm{y}\|_2^2 + \lambda \|\bm{x}\|_2^2
    \label{eq:regularization_tikhonov_objective}
\end{equation}
を最小化する.
ここで,
$\bm{x} \in \setC^n$,
$\bm{y} \in \setC^m$,
$A \in \setC^{m \times n}$
である.

\section{係数行列による解}

評価関数 \eqref{eq:regularization_tikhonov_objective} を最小化する $\bm{x}$ は
係数行列 $A$ により次のように表すことができる.

\begin{theorem}\label{theorem:regularization_tikhonov_solution}
    $\lambda > 0$ の場合,
    評価関数 \eqref{eq:regularization_tikhonov_objective} が最小となるのは,
    $\bm{x} = (A^* A + \lambda I)^{-1} A^* \bm{y}$ の場合である.
\end{theorem}
\begin{proof}
    この場合,ノルムを展開することで,次のようになる.
    \begin{align}
        E_{\lambda}(\bm{x})
         & = \|A \bm{x} - \bm{y}\|_2^2 + \lambda \|\bm{x}\|_2^2
        \notag                                                                                              \\
         & = (A \bm{x} - \bm{y})^* (A \bm{x} - \bm{y}) + \lambda \bm{x}^* \bm{x}
        \notag                                                                                              \\
         & = \bm{x}^* (A^* A + \lambda I) \bm{x} - \bm{x}^* A^* \bm{y} - \bm{y}^* A \bm{x} + \|\bm{y}\|_2^2
    \end{align}
    ここで,
    $\lambda > 0$ の場合は
    $\bm{x} \neq \bm{0}$ において
    $\bm{x}^* (A^* A + \lambda I) \bm{x} = \|A \bm{x}\|_2^2 + \lambda \|\bm{x}\|_2^2 > 0$
    となることから,
    エルミート行列 $(A^* A + \lambda I)$ は正定値であり,
    正則となる.
    このことを用いると,さらに次のように展開できる.
    \begin{align}
        E_{\lambda}(\bm{x})
         & = \bm{x}^* (A^* A + \lambda I) \bm{x} - \bm{x}^* A^* \bm{y} - \bm{y}^* A \bm{x} + \|\bm{y}\|_2^2
        \notag                                                                                              \\
         & = \bm{x}^* (A^* A + \lambda I) \bm{x} - \bm{x}^* A^* \bm{y} - \bm{y}^* A \bm{x}
        + \bm{y}^* A (A^* A + \lambda I)^{-1} A^* \bm{y}
        - \bm{y}^* A (A^* A + \lambda I)^{-1} A^* \bm{y}
        + \|\bm{y}\|_2^2
        \notag                                                                                              \\
         & = (\bm{x} - (A^* A + \lambda I)^{-1} A^* \bm{y})^*
        (A^* A + \lambda I)
        (\bm{x} - (A^* A + \lambda I)^{-1} A^* \bm{y})
        - \bm{y}^* A (A^* A + \lambda I)^{-1} A^* \bm{y}
        + \|\bm{y}\|_2^2
    \end{align}
    エルミート行列 $(A^* A + \lambda I)$ が正定値であることから,
    この式が最小となるのは
    $(\bm{x} - (A^* A + \lambda I)^{-1} A^* \bm{y})$
    が零ベクトルとなる場合,
    つまり,
    $\bm{x} = (A^* A + \lambda I)^{-1} A^* \bm{y}$
    の場合である.
\end{proof}

ここで,$\lambda = 0$ でも,行列 $A$ のランクが $n$ の場合は $A^*A$ が正定値になるため同様に解が求まる.

\section{特異値分解による解法}

行列 $A$ を次のように特異値分解する.

\begin{equation}
    A = U
    \begin{pmatrix}
        \Sigma & O \\
        O      & O
    \end{pmatrix}
    V^*
\end{equation}

ここで,$U \in \setC^{m \times m}$ と $V \in \setC^{n \times n}$ はユニタリ行列で,
$\Sigma \in \setR^{r \times r}$ は正の実数による対角行列である.
ただし,ランク $r$ は $m$ または $n$ に等しくても良い.

この分解を用いると,
定理 \ref{theorem:regularization_tikhonov_solution} の解は次のように変形できる.

\begin{align}
    x_{\lambda}
     & \equiv (A^* A + \lambda I)^{-1} A^* \bm{y}
    \notag                                        \\
     & = \left(V
    \begin{pmatrix}
        \Sigma & O \\
        O      & O
    \end{pmatrix}
    U^* U
    \begin{pmatrix}
        \Sigma & O \\
        O      & O
    \end{pmatrix}
    V^* + \lambda I \right)^{-1}
    V
    \begin{pmatrix}
        \Sigma & O \\
        O      & O
    \end{pmatrix}
    U^* \bm{y}
    \notag                                        \\
     & = \left(V
    \begin{pmatrix}
        \Sigma^2 & O \\
        O        & O
    \end{pmatrix}
    V^* + \lambda V V^* \right)^{-1}
    V
    \begin{pmatrix}
        \Sigma & O \\
        O      & O
    \end{pmatrix}
    U^* \bm{y}
    \notag                                        \\
     & = \left(V
    \begin{pmatrix}
        \Sigma^2 + \lambda I & O         \\
        O                    & \lambda I
    \end{pmatrix}
    V^* \right)^{-1}
    V
    \begin{pmatrix}
        \Sigma & O \\
        O      & O
    \end{pmatrix}
    U^* \bm{y}
    \notag                                        \\
     & = V
    \begin{pmatrix}
        (\Sigma^2 + \lambda I)^{-1} & O              \\
        O                           & \lambda^{-1} I
    \end{pmatrix}
    V^*
    V
    \begin{pmatrix}
        \Sigma & O \\
        O      & O
    \end{pmatrix}
    U^* \bm{y}
    \notag                                        \\
     & = V
    \begin{pmatrix}
        (\Sigma^2 + \lambda I)^{-1} \Sigma & O \\
        O                                  & O
    \end{pmatrix}
    U^* \bm{y}
\end{align}

さらに,
$U = (\bm{u}_1, \bm{u}_2, \ldots, \bm{u}_m)$,
$V = (\bm{v}_1, \bm{v}_2, \ldots, \bm{v}_n)$,
$\Sigma = \diag(\sigma_1, \sigma_2, \ldots, \sigma_r)$
とすると,次のように表記できる.

\begin{align}
    x_{\lambda}
     & = \sum_{i = 1}^{r} \frac{\sigma_i}{\sigma_i^2 + \lambda} (\bm{u}_i^* \bm{y}) \bm{v}_i
    \label{eq:regularization_tikhonov_solution-by-svd}
\end{align}

特異値分解を一回行い
$\bm{u}_i^* \bm{y}$
を計算しておけば,
ある正則化パラメータ $\lambda$ に対する解 $\bm{x}_{\lambda}$ は単純な線形和で求めることができる.

また,
$U_1 \equiv (\bm{u}_1, \bm{u}_2, \ldots, \bm{u}_r)$,
$U_2 \equiv (\bm{u}_{r+1}, \bm{u}_{r+2}, \ldots, \bm{u}_m)$
と定義した場合,$U$ がユニタリ行列であることから
$U_1^* U_1 = I$,
$U_1^* U_2 = O$,
$U_2^* U_1 = O$,
$U_2^* U_2 = I$,
$U_1 U_1^* + U_2 U_2^* = I$
が成り立つことを利用すると,
正則化パラメータを評価する際にしばしば利用される評価関数内のノルムは次のように計算できる
\footnote{行列 $U$ のうち $U_1$ の部分だけを求めた方が計算コストを抑えられるため,%
    $U_2$ はなるべく使用しない計算式を求めている.}.

\begin{align}
    \|A \bm{x}_{\lambda} - \bm{y}\|_2^2
     & = \left\| U
    \begin{pmatrix}
        \Sigma & O \\
        O      & O
    \end{pmatrix}
    V^* V
    \begin{pmatrix}
        (\Sigma^2 + \lambda I)^{-1} \Sigma & O \\
        O                                  & O
    \end{pmatrix}
    U^* \bm{y} - \bm{y} \right\|_2^2
    \notag                                                                                        \\
     & = \left\|
    \begin{pmatrix}
        U_1 & U_2
    \end{pmatrix}
    \begin{pmatrix}
        \Sigma (\Sigma^2 + \lambda I)^{-1} \Sigma & O \\
        O                                         & O
    \end{pmatrix}
    \begin{pmatrix}
        U_1^* \\ U_2^*
    \end{pmatrix}
    \bm{y} - \bm{y} \right\|_2^2
    \notag                                                                                        \\
     & = \left\| U_1 \Sigma (\Sigma^2 + \lambda I)^{-1} \Sigma U_1^* \bm{y} - \bm{y} \right\|_2^2
    \notag                                                                                        \\
     & = \left\| U_1 \Sigma (\Sigma^2 + \lambda I)^{-1} \Sigma U_1^* \bm{y}
    - U_1 U_1^* \bm{y} - U_2 U_2^* \bm{y} \right\|_2^2
    \notag                                                                                        \\
     & = \left\| U_1 (\Sigma (\Sigma^2 + \lambda I)^{-1} \Sigma - I) U_1^* \bm{y}
    - U_2 U_2^* \bm{y} \right\|_2^2
    \notag                                                                                        \\
     & = \left\| U_1 (\Sigma (\Sigma^2 + \lambda I)^{-1} \Sigma - I) U_1^* \bm{y} \right\|_2^2
    + \left\| U_2 U_2^* \bm{y} \right\|_2^2
    \notag                                                                                        \\
     & = \left\| (\Sigma (\Sigma^2 + \lambda I)^{-1} \Sigma - I) U_1^* \bm{y} \right\|_2^2
    + \left\| U_2 U_2^* \bm{y} \right\|_2^2
    \notag                                                                                        \\
     & = \sum_{i = 1}^r \left(\frac{\lambda}{\sigma_i^2 + \lambda}\right)^2 (\bm{u}_i^* \bm{y})^2
    + \left\| (I - U_1 U_1^*) \bm{y} \right\|_2^2
\end{align}

\begin{align}
    \|\bm{x}_{\lambda}\|_2^2
     & = \left\| V
    \begin{pmatrix}
        (\Sigma^2 + \lambda I)^{-1} \Sigma & O \\
        O                                  & O
    \end{pmatrix}
    U^* \bm{y} \right\|_2^2
    \notag                                                                                          \\
     & = \left\|
    \begin{pmatrix}
        (\Sigma^2 + \lambda I)^{-1} \Sigma & O \\
        O                                  & O
    \end{pmatrix}
    U^* \bm{y} \right\|_2^2
    \notag                                                                                          \\
     & = \sum_{i = 1}^r \left(\frac{\sigma_i}{\sigma_i^2 + \lambda}\right)^2  (\bm{u}_i^* \bm{y})^2
\end{align}

どちらも正則化パラメータごとに異なる部分は $O(r)$ オーダーで計算できる.

特異値分解による Tikhonov 正則化では,
正則化パラメータを変更した際の再計算が容易なため,
多数の正則化パラメータを試したい場合に便利である.
