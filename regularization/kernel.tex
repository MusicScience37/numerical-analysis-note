% !TEX root = ../main.tex
%

\chapter{カーネルによる補間への応用}

\ref{sec:regularization_tikhonov_underdetermined} 節で示した変形は,
カーネルによる補間に応用できる.

関数 $f : X \to \setC$ を
サンプル点 $(\bm{r}_i, y_i) \in X \times \setC$ ($i = 1, 2, \ldots, m$) から補間することを考える.
カーネルを用いた補間では,カーネル $K : X \times X \to \setC$ を用いて
\begin{equation}
    f(\bm{r}) = \sum_{i=1}^m c_i K(\bm{r}, \bm{r}_i)
\end{equation}
のようにおき,$y_i = f(\bm{r}_i)$ となるように係数 $c_i$ を決める.

\section{導出}

まず,カーネルによる補間の導出を行う.

関数 $f$ はある関数空間 $\mathcal{H}$ に属するものとし,
その関数空間 $\mathcal{H}$ には基底となる
関数 $\alpha_1(\bm{r}), \alpha_2(\bm{r}), \ldots, \alpha_N(\bm{r})$ が存在するものとする
\footnote{この時点ではまだ $N$ が有限であるとする.}.
それらの基底を用いて関数 $f$ を次のようにおく.
\begin{equation}
    f(\bm{r}) = \sum_{i = 1}^{N} x_i \alpha_i(\bm{r})
\end{equation}
ここで,
$A_{ij} = \alpha_j(\bm{r}_i)$ となる行列 $A \in \setC^{m \times N}$ を考えると,
最小二乗法の評価関数は
\begin{equation}
    \|A \bm{x} - \bm{y}\|_2
\end{equation}
となる.
これに Tikhonov 正則化を適用すると,
式 \eqref{eq:regularization_tikhonov_exchange-mat} より最適解は
\begin{align}
    \bm{x} & = A^* (AA^* + \lambda I)^{-1} \bm{y}
\end{align}
となる.
