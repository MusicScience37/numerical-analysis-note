% !TEX root = ../main.tex
%

\chapter{一般化 Tikhonov 正則化}

前章の Tikhonov 正則化では,
正則化項が $\|\bm{x}\|_2^2$ となっていたが,
正則化項として係数行列を追加した $\|L\bm{x}\|_2^2$ を用いることも考えられる.
例えば,変数 $\bm{x}$ の隣り合う要素の差をとるように係数行列 $L$ を決めることにより,
解が滑らかになるような正則化を行うことができる.
このような一般化した Tikhonov 正則化では,評価関数
\begin{equation}
    E_{\lambda}(\bm{x}) \equiv \|A \bm{x} - \bm{y}\|_2^2 + \lambda \|L \bm{x}\|_2^2
    \label{eq:regularization_gen-tikhonov_objective}
\end{equation}
を最小化する.
ここで,
$\bm{x} \in \setC^n$,
$\bm{y} \in \setC^m$,
$A \in \setC^{m \times n}$,
$L \in \setC^{p \times n}$
である.

ここで,正則化項の変更により注意することがある.
Tikhonov 正則化では,$\lambda > 0$ であれば解が唯一となっていた
(定理 \ref{theorem:regularization_tikhonov_solution}).
しかし,一般化 Tikhonov 正則化においては,
$\lambda > 0$ であるからといって解が唯一になるとは限らない.

\begin{theorem}
    $A$ と $L$ の核空間の共通部分が $\bm{0}$ 以外の要素を持つ場合,
    $\lambda > 0$ であったとしても
    評価関数 \eqref{eq:regularization_gen-tikhonov_objective} が最小となる $\bm{x}$ が
    唯一に定まらない.
\end{theorem}
\begin{proof}
    評価関数 \eqref{eq:regularization_gen-tikhonov_objective} が最小となる
    $\bm{x}$ の 1 つを $\bar{\bm{x}}$ とする.
    また,$A$ と $L$ の核空間の共通部分が $\bm{0}$ 以外に持つ要素の 1 つを $\bm{x}_0$ とする.
    このとき,
    \begin{align}
        E_{\lambda}(\bar{\bm{x}} + \bm{x}_0)
         & = \|A (\bar{\bm{x}} + \bm{x}_0) - \bm{y}\|_2^2 + \lambda \|L (\bar{\bm{x}} + \bm{x}_0)\|_2^2
        \notag                                                                                          \\
         & = \|A \bar{\bm{x}} - \bm{y}\|_2^2 + \lambda \|L \bar{\bm{x}}\|_2^2
        \notag                                                                                          \\
         & = E(\bar{\bm{x}})
    \end{align}
    となる.
    よって,評価関数 $E(\bm{x})$ が最小となる $\bm{x}$ は唯一に定まらない.
\end{proof}

一方,$A$ と $L$ の核空間の共通部分が $\bm{0}$ のみであれば
$\lambda > 0$ のときに解が唯一となる.

\begin{theorem}\label{theorem:regularization_gen-tikhonov_solution}
    $A$ と $L$ の核空間の共通部分が $\bm{0}$ のみでかつ,
    $\lambda > 0$ の場合,
    評価関数 \eqref{eq:regularization_gen-tikhonov_objective} が最小となるのは,
    $\bm{x} = (A^* A + \lambda L^* L)^{-1} A^* \bm{y}$ の場合である.
\end{theorem}
\begin{proof}
    この場合,ノルムを展開することで,次のようになる.
    \begin{align}
        E_{\lambda}(\bm{x})
         & = \|A \bm{x} - \bm{y}\|_2^2 + \lambda \|L \bm{x}\|_2^2
        \notag                                                                                                  \\
         & = (A \bm{x} - \bm{y})^* (A \bm{x} - \bm{y}) + \lambda \bm{x}^* L^* L \bm{x}
        \notag                                                                                                  \\
         & = \bm{x}^* (A^* A + \lambda L^* L) \bm{x} - \bm{x}^* A^* \bm{y} - \bm{y}^* A \bm{x} + \|\bm{y}\|_2^2
    \end{align}
    ここで,
    $A$ と $L$ の核空間の共通部分が $\bm{0}$ のみでかつ
    $\lambda > 0$ の場合は
    $\bm{x} \neq \bm{0}$ において
    $\bm{x}^* (A^* A + \lambda L^* L) \bm{x} = \|A \bm{x}\|_2^2 + \lambda \|L \bm{x}\|_2^2 > 0$
    となることから,
    エルミート行列 $(A^* A + \lambda I)$ は正定値であり,
    正則となる.
    このことを用いると,さらに次のように展開できる.
    \begin{align}
        E_{\lambda}(\bm{x})
         & = \bm{x}^* (A^* A + \lambda L^* L) \bm{x} - \bm{x}^* A^* \bm{y} - \bm{y}^* A \bm{x} + \|\bm{y}\|_2^2
        \notag                                                                                                  \\
         & = \bm{x}^* (A^* A + \lambda L^* L) \bm{x} - \bm{x}^* A^* \bm{y} - \bm{y}^* A \bm{x}
        + \bm{y}^* A (A^* A + \lambda L^* L)^{-1} A^* \bm{y}
        - \bm{y}^* A (A^* A + \lambda L^* L)^{-1} A^* \bm{y}
        + \|\bm{y}\|_2^2
        \notag                                                                                                  \\
         & = (\bm{x} - (A^* A + \lambda L^* L)^{-1} A^* \bm{y})^*
        (A^* A + \lambda L^* L)
        (\bm{x} - (A^* A + \lambda L^* L)^{-1} A^* \bm{y})
        - \bm{y}^* A (A^* A + \lambda L^* L)^{-1} A^* \bm{y}
        + \|\bm{y}\|_2^2
    \end{align}
    エルミート行列 $(A^* A + \lambda L^* L)$ が正定値であることから,
    この式が最小となるのは
    $(\bm{x} - (A^* A + \lambda L^* L)^{-1} A^* \bm{y})$
    が零ベクトルとなる場合,
    つまり,
    $\bm{x} = (A^* A + \lambda L^* L)^{-1} A^* \bm{y}$
    の場合である.
\end{proof}

\section{一般化 Tikhonov 正則化}

$m \ge n \ge p$ の場合は
次のように表される一般化特異値分解を行うことができる \cite{Hansen1998}.

\begin{align}
    A       & =U
    \begin{pmatrix}
        \Sigma & O \\
        O      & I
    \end{pmatrix}
    W^{-1}, &
    L       & =V
    \begin{pmatrix}
        M & O
    \end{pmatrix}
    W^{-1}
\end{align}

ここで,
$U \in \setC^{m \times n}$,
$V \in \setC^{p \times p}$
はユニタリ行列で,
$W \in \setC^{n \times n}$
は正則行列とし,
$\Sigma = \diag(\sigma_1, \sigma_2, \ldots, \sigma_p)$,
$M = \diag(\mu_1, \mu_2, \ldots, \mu_p)$
は対角行列である.
$\sigma_i$ と $\mu_i$ については
$0 \le \sigma_1 \le \sigma_2 \le \ldots \le \sigma_p \le 1$,
$1 \ge \mu_1 \ge \mu_2 \ge \ldots \ge \mu_p > 0$,
$\sigma_i^2 + \mu_i^2 = 1$
を満たすものとし,
$\gamma_i = \sigma_i / \mu_i$
を一般化特異値と呼ぶ.
これを用いると,
次のように評価関数 \eqref{eq:regularization_gen-tikhonov_objective} を最小化する
$\bm{x}$ を表せる\cite{Hansen1998}.

\begin{equation}
    \bm{x}_\lambda =
    \sum_{i=1}^{p} \frac{\gamma_i / \mu_i}{\gamma_i^2+\lambda}
    (\bm{u}^*\bm{y}) \bm{w}_i
    +\sum_{i=p+1}^n (\bm{u}_i^*\bm{y}) \bm{w}_i
\end{equation}

$L=I$ のときこの式が Tikhonov 正則化の場合の
式 \eqref{eq:regularization_tikhonov_solution-by-svd} に一致することは
簡単に確かめられる.
