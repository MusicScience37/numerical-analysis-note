% !TEX root = ../main.tex
%

\chapter{一般化 Tikhonov 正則化}

\index{いっぱんか Tikhonov せいそくか@一般化 Tikhonov 正則化}
前章の Tikhonov 正則化では,
正則化項が $\|\bm{x}\|_2^2$ となっていたが,
正則化項として係数行列を追加した $\|L\bm{x}\|_2^2$ を用いることも考えられる.
例えば,変数 $\bm{x}$ の隣り合う要素の差をとるように係数行列 $L$ を決めることにより,
解が滑らかになるような正則化を行うことができる.
このような一般化した Tikhonov 正則化では,評価関数
\begin{equation}
    E_{\lambda}(\bm{x}) \equiv \|A \bm{x} - \bm{y}\|_2^2 + \lambda \|L \bm{x}\|_2^2
    \label{eq:regularization_gen-tikhonov_objective}
\end{equation}
を最小化する.
ここで,
$\bm{x} \in \setC^n$,
$\bm{y} \in \setC^m$,
$A \in \setC^{m \times n}$,
$L \in \setC^{p \times n}$
である.

ここで,正則化項の変更により注意することがある.
Tikhonov 正則化では,$\lambda > 0$ であれば解が唯一となっていた
(定理 \ref{theorem:regularization_tikhonov_solution}).
しかし,一般化 Tikhonov 正則化においては,
$\lambda > 0$ であるからといって解が唯一になるとは限らない.

\begin{theorem}
    $A$ と $L$ の核空間の共通部分が $\bm{0}$ 以外の要素を持つ場合,
    $\lambda > 0$ であったとしても
    評価関数 \eqref{eq:regularization_gen-tikhonov_objective} が最小となる $\bm{x}$ が
    唯一に定まらない.
\end{theorem}
\begin{proof}
    評価関数 \eqref{eq:regularization_gen-tikhonov_objective} が最小となる
    $\bm{x}$ の 1 つを $\bar{\bm{x}}$ とする.
    また,$A$ と $L$ の核空間の共通部分が $\bm{0}$ 以外に持つ要素の 1 つを $\bm{x}_0$ とする.
    このとき,
    \begin{align}
        E_{\lambda}(\bar{\bm{x}} + \bm{x}_0)
         & = \|A (\bar{\bm{x}} + \bm{x}_0) - \bm{y}\|_2^2 + \lambda \|L (\bar{\bm{x}} + \bm{x}_0)\|_2^2
        \notag                                                                                          \\
         & = \|A \bar{\bm{x}} - \bm{y}\|_2^2 + \lambda \|L \bar{\bm{x}}\|_2^2
        \notag                                                                                          \\
         & = E(\bar{\bm{x}})
    \end{align}
    となる.
    よって,評価関数 $E(\bm{x})$ が最小となる $\bm{x}$ は唯一に定まらない.
\end{proof}

一方,$A$ と $L$ の核空間の共通部分が $\bm{0}$ のみであれば
$\lambda > 0$ のときに解が唯一となる.

\begin{theorem}\label{theorem:regularization_gen-tikhonov_solution}
    $A$ と $L$ の核空間の共通部分が $\bm{0}$ のみでかつ,
    $\lambda > 0$ の場合,
    評価関数 \eqref{eq:regularization_gen-tikhonov_objective} が最小となるのは,
    $\bm{x} = (A^* A + \lambda L^* L)^{-1} A^* \bm{y}$ の場合である.
\end{theorem}
\begin{proof}
    この場合,ノルムを展開することで,次のようになる.
    \begin{align}
        E_{\lambda}(\bm{x})
         & = \|A \bm{x} - \bm{y}\|_2^2 + \lambda \|L \bm{x}\|_2^2
        \notag                                                                                                  \\
         & = (A \bm{x} - \bm{y})^* (A \bm{x} - \bm{y}) + \lambda \bm{x}^* L^* L \bm{x}
        \notag                                                                                                  \\
         & = \bm{x}^* (A^* A + \lambda L^* L) \bm{x} - \bm{x}^* A^* \bm{y} - \bm{y}^* A \bm{x} + \|\bm{y}\|_2^2
    \end{align}
    ここで,
    $A$ と $L$ の核空間の共通部分が $\bm{0}$ のみでかつ
    $\lambda > 0$ の場合は
    $\bm{x} \neq \bm{0}$ において
    $\bm{x}^* (A^* A + \lambda L^* L) \bm{x} = \|A \bm{x}\|_2^2 + \lambda \|L \bm{x}\|_2^2 > 0$
    となることから,
    エルミート行列 $(A^* A + \lambda I)$ は正定値であり,
    正則となる.
    このことを用いると,さらに次のように展開できる.
    \begin{align}
        E_{\lambda}(\bm{x})
         & = \bm{x}^* (A^* A + \lambda L^* L) \bm{x} - \bm{x}^* A^* \bm{y} - \bm{y}^* A \bm{x} + \|\bm{y}\|_2^2
        \notag                                                                                                  \\
         & = \bm{x}^* (A^* A + \lambda L^* L) \bm{x} - \bm{x}^* A^* \bm{y} - \bm{y}^* A \bm{x}
        + \bm{y}^* A (A^* A + \lambda L^* L)^{-1} A^* \bm{y}
        - \bm{y}^* A (A^* A + \lambda L^* L)^{-1} A^* \bm{y}
        + \|\bm{y}\|_2^2
        \notag                                                                                                  \\
         & = (\bm{x} - (A^* A + \lambda L^* L)^{-1} A^* \bm{y})^*
        (A^* A + \lambda L^* L)
        (\bm{x} - (A^* A + \lambda L^* L)^{-1} A^* \bm{y})
        - \bm{y}^* A (A^* A + \lambda L^* L)^{-1} A^* \bm{y}
        + \|\bm{y}\|_2^2
    \end{align}
    エルミート行列 $(A^* A + \lambda L^* L)$ が正定値であることから,
    この式が最小となるのは
    $(\bm{x} - (A^* A + \lambda L^* L)^{-1} A^* \bm{y})$
    が零ベクトルとなる場合,
    つまり,
    $\bm{x} = (A^* A + \lambda L^* L)^{-1} A^* \bm{y}$
    の場合である.
\end{proof}

\section{一般化特異値分解}

\index{いっぱんかとくいちぶんかい@一般化特異値分解}
$m \ge n \ge p$ の場合は
次のように表される一般化特異値分解を行うことができる \cite{Hansen1998}.

\begin{align}
    A       & =U
    \begin{pmatrix}
        \Sigma & O \\
        O      & I
    \end{pmatrix}
    W^{-1}, &
    L       & =V
    \begin{pmatrix}
        M & O
    \end{pmatrix}
    W^{-1}
\end{align}

ここで,
$U \in \setC^{m \times n}$,
$V \in \setC^{p \times p}$
はユニタリ行列で,
$W \in \setC^{n \times n}$
は正則行列とし,
$\Sigma = \diag(\sigma_1, \sigma_2, \ldots, \sigma_p)$,
$M = \diag(\mu_1, \mu_2, \ldots, \mu_p)$
は対角行列である.
$\sigma_i$ と $\mu_i$ については
$0 \le \sigma_1 \le \sigma_2 \le \ldots \le \sigma_p \le 1$,
$1 \ge \mu_1 \ge \mu_2 \ge \ldots \ge \mu_p > 0$,
$\sigma_i^2 + \mu_i^2 = 1$
を満たすものとし,
$\gamma_i = \sigma_i / \mu_i$
を一般化特異値と呼ぶ.
これを用いると,
次のように評価関数 \eqref{eq:regularization_gen-tikhonov_objective} を最小化する
$\bm{x}$ を表せる\cite{Hansen1998}.

\begin{equation}
    \bm{x}_\lambda =
    \sum_{i=1}^{p} \frac{\gamma_i / \mu_i}{\gamma_i^2+\lambda}
    (\bm{u}_i^*\bm{y}) \bm{w}_i
    +\sum_{i=p+1}^n (\bm{u}_i^*\bm{y}) \bm{w}_i
\end{equation}

$L=I$ のときは,
$p = n$, $\mu_i = 1$ となり,
Tikhonov 正則化の場合の
式 \eqref{eq:regularization_tikhonov_solution-by-svd} に一致する.

\section{単純な L2 ノルムによる Tikhonov 正則化への変換}

一般化 Tikhonov 正則化を通常の Tikhonov 正則化へ
変形することもできる \cite{Hansen1998}.
その変形では,次のように変数を置き換える.
\begin{align}
    \bm{x}      & = L_A^\dagger \bar{\bm{x}} + \bm{x}_0,                     &
    L_A^\dagger & = \left(I - \left(A P_L\right)^\dagger A\right) L^\dagger, &
    \bm{x}_0    & = \left(A P_L\right)^\dagger \bm{y}
    \label{eq:regularization_gen-tikhonov_change-of-variables}
\end{align}
ただし,$P_L = I - L^\dagger L$とする.
一般化 Tikhonov 正則化の問題を数値的に解く際は
この変換で通常の Tikhonov 正則化に直して解くと
安定かつ効率的に解が得られることが知られている \cite{Hansen1998}.
そこで,この変換について以下に示す.

まず,式 \eqref{eq:regularization_gen-tikhonov_change-of-variables} の変換は
次のような意味を持っている.

\begin{lemma}[{\cite[3 節]{Elden1982}}]\label{lemma:regularization_gen-tikhonov_meaning-of-change-of-variables}
    最適化問題
    \begin{align}
        \text{minimize} \hspace{1.5em} & \|A \bm{x} - \bm{y}\|_2 \notag \\
        \text{s.t.} \hspace{1.5em}     & L \bm{x} = \bar{\bm{x}} \notag
    \end{align}
    において,
    行列
    $A \in \setC^{m \times n}$,
    $L \in \setC^{p \times n}$
    ($m \ge n \ge p$)は
    核空間に $\bm{0}$ 以外の共通部分を持たないものとする.
    このとき,この問題の解は
    \begin{equation}
        \bm{x} = L_A^\dagger \bar{\bm{x}} + (A P_L)^\dagger\bm{y}
    \end{equation}
    である.
\end{lemma}
\begin{proof}
    まず,$L\bm{x}=\bar{\bm{x}}$ の一般解は次のように書ける
    (式 \eqref{eq:matrix-computation_moore-penrose_general-least-squares-solution}).
    \begin{equation}
        \bm{x} = L^\dagger \bar{\bm{x}} + P_L \bm{z}
    \end{equation}
    これを目的関数に代入すると
    \begin{equation}
        \|A\bm{x} - \bm{y}\|_2
        =\|A L^\dagger \bar{\bm{x}} + A P_L \bm{z} - \bm{y}\|_2
        =\|(A P_L) \bm{z} - (\bm{y} - A L^\dagger \bar{\bm{x}})\|_2
    \end{equation}
    のように変形でき,
    このノルムを最小化する$\bm{z}$は次のように書ける.
    \begin{equation}
        \bm{z} =
        (A P_L)^\dagger (\bm{y} - A L^\dagger \bar{\bm{x}})
        + (I - (A P_L)^\dagger (A P_L)) \bm{s}
    \end{equation}
    ここで,$\bm{s}$ は任意のベクトルである.
    ここで,$P_L$ は $L$ の核空間への射影演算子であり,
    $A$ と $L$ が核空間に $\bm{0}$ 以外の共通部分を持たないことから,
    $A P_L$ の核空間は $L$ の核空間の直交補空間である.
    任意のベクトル $\bm{a}$ について
    $(A P_L)^\dagger \bm{a}$ は $A P_L$ の核空間の成分を持たないため,
    $L$ の核空間に属していることが分かる.
    一方,
    $(I - (A P_L)^\dagger(A P_L)) \bm{s}$
    は $A P_L$ の核空間に属しており,
    $L$ の核空間に直交する.
    よって,
    \begin{equation}
        P_L \bm{z} =
        (A P_L)^\dagger (\bm{y} - A L^\dagger \bar{\bm{x}})
    \end{equation}
    となる.

    このときの $\bm{x}$ は
    \begin{align}
        \bm{x} & = L^\dagger \bar{\bm{x}} + P_L \bm{z} \notag                                               \\
               & = L^\dagger \bar{\bm{x}} +(A P_L)^\dagger (\bm{y} - A L^\dagger \bar{\bm{x}}) \notag       \\
               & = \left(I - (A P_L)^\dagger A \right)L^\dagger \bar{\bm{x}} + (AP_L)^\dagger \bm{y} \notag \\
               & = L_A^\dagger \bar{\bm{x}} + (A P_L)^\dagger \bm{y}
    \end{align}
    となり,解が得られる.
\end{proof}

この補題を用いることで,次のように評価関数の変換を示すことができる.

\begin{theorem}[{\cite{Hansen1998}}]
    $A$ と $L$ の核空間の共通部分が $\bm{0}$ のみである場合,
    変数変換 \eqref{eq:regularization_gen-tikhonov_change-of-variables} により
    評価関数 \eqref{eq:regularization_gen-tikhonov_objective} を次のように置き換えることができる.
    \begin{equation}
        E'(\bar{\bm{x}}) =
        \left\|\bar{A} \bar{\bm{x}} - \bar{\bm{y}}\right\|_2^2
        + \lambda \|\bar{\bm{x}}\|_2^2
    \end{equation}
    ただし,
    \begin{align}
        \bar{A}      & = A L_A^\dagger                     \\
        \bar{\bm{y}} & = \bm{y} - A \bm{x}_0               \\
        \bm{x}_0     & = \left(A P_L\right)^\dagger \bm{y}
    \end{align}
    である.
\end{theorem}
\begin{proof}
    補題 \ref{lemma:regularization_gen-tikhonov_meaning-of-change-of-variables} より,
    次のように変形できる.

    \begin{align}
        \min_{\bm{x}\in\setC^n} E(\bm{x})
         & = \min_{\bm{x}\in\setC^n} \left( \|A \bm{x} - \bm{y}\|_2^2 + \lambda \|L \bm{x}\|_2^2 \right)
        \notag                                                                                           \\
         & = \min_{\bar{\bm{x}} \in \setC^p}
        \min_{\bm{x} \in \{\bm{x} \in \setC^n \mid L \bm{x} = \bar{\bm{x}} \}}
        \left( \|A \bm{x} - \bm{y}\|_2^2 + \lambda \|L \bm{x}\|_2^2 \right)
        \notag                                                                                           \\
         & = \min_{\bar{\bm{x}} \in \setC^p}
        \left( \left\|A \left(L_A^\dagger \bar{\bm{x}} + \bm{x}_0\right) - \bm{y}\right\|_2^2
        + \lambda \left\|\bar{\bm{x}}\right\|_2^2 \right)
        \notag                                                                                           \\
         & = \min_{\bar{\bm{x}} \in \setC^p}
        \left( \left\|\bar{A} \bar{\bm{x}} - \bar{\bm{y}}\right\|_2^2
        + \lambda \left\|\bar{\bm{x}}\right\|_2^2 \right)
        \notag                                                                                           \\
         & = \min_{\bar{\bm{x}} \in \setC^p} E'(\bar{\bm{x}})
    \end{align}

    以上より,$E(\bm{x})$ の最小化は $E'(\bar{\bm{x}})$ の最小化へ置き換えることができる.
\end{proof}

最後にこの変換をコンピュータ上で行う手法について
文献 \cite{Elden1982, Hansen1994} の結果をまとめる.

\subsection{行列 L が行と同じ数のランクを持つ場合の計算方法}

ここでは,$L \in \setC^{p \times n}$ が $p < n$ でランク $p$ である場合を扱う.

まず,$L^*$ を QR 分解する.
\begin{equation}
    L^* = \begin{pmatrix}
        V_1 & V_2
    \end{pmatrix}\begin{pmatrix}
        R \\ O
    \end{pmatrix}
\end{equation}

ここで,
$V_1 \in \setC^{n \times p}$,
$V_2 \in \setC^{n \times (n-p)}$
は $V = (V_1, V_2)$ がユニタリ行列になるような
行列とし,
$R \in \setC^{p \times p}$ は
正則な上三角行列である.
このとき,$L^\dagger = V_1 R^{-*}$, $LV_2 = O$ となることと,
$A$ と $L$ の核空間の共通部分が $\bm{0}$ のみであることから,
$AV_2$ は列と同じ数のランクを持つ.
よって,次のように QR 分解できる.
\begin{equation}
    AV_2 = \begin{pmatrix}
        Q_1 & Q_2
    \end{pmatrix}\begin{pmatrix}
        U \\ O
    \end{pmatrix}
\end{equation}

これも同様に
$Q_1 \in \setC^{m \times (n-p)}$,
$Q_2 \in \setC^{m \times (m-n+p)}$
は$Q = (Q_1, Q_2)$ がユニタリ行列になるような行列とし,
$U \in \setC^{(n - p) \times (n - p)}$
は正則な上三角行列である.
このとき,
\begin{align}
    \bm{x}_0 & = \left(A (I - L^\dagger L)\right)^\dagger \bm{y} \notag \\
             & = \left(A V_2 V_2^*\right)^\dagger \bm{y}
\end{align}
となる.

ここで,$(AV_2V_2^*)^\dagger$ について定義に沿って考える.
最適化問題
\begin{align}
    \text{minimize} \hspace{2em} & \|\bm{x}\|_2 \notag                                                                    \\
    \text{s.t.} \hspace{2em}     & \|AV_2V_2^* \bm{x} - \bm{a}\|_2 = \min_{\bm{x}} \|AV_2V_2^* \bm{x} - \bm{a}\|_2 \notag
\end{align}
において,
$\bm{x} = V_1 \bm{s}_1 + V_2 \bm{s}_2$
とおくと,
$\|A V_2 V_2^* \bm{x} - \bm{a}\|_2 = \|A V_2 \bm{s}_2 - \bm{a}\|_2$
となる.
これを最小化すると$\bm{s}_2=(AV_2)^\dagger\bm{a}$となる.
また,
$\|\bm{x}\|_2^2 = \|\bm{s}_1\|_2^2 + \|\bm{s}_2\|_2^2$
となるから,
$\bm{s}_2 = (A V_2)^\dagger \bm{a}$
は固定された状態で
$\|\bm{x}\|_2$
を最小化すると
$\bm{s}_1=\bm{0}$
となる.
よって,この最適化問題の解は
$\bm{x} = V_2 (A V_2)^\dagger \bm{a}$
となり,
$(A V_2 V_2^*)^\dagger = V_2 (A V_2)^\dagger$
とわかる.

よって,
\begin{align}
    \bm{x}_0 & = \left(A (I - L^\dagger L)\right)^\dagger \bm{y} \notag \\
             & = V_2 \left(A V_2\right)^\dagger \bm{y} \notag           \\
             & = V_2 U^{-1} Q_1^* \bm{y}
\end{align}
となる.また,
\begin{align}
    \bar{\bm{y}} & = \bm{y} - A \bm{x}_0 \notag                \\
                 & = \bm{y} - A V_2 U^{-1} Q_1^* \bm{y} \notag \\
                 & = (I - Q_1 Q_1^*) \bm{y} \notag             \\
                 & = Q_2 Q_2^* \bm{y}
\end{align}
と書ける.さらに,
\begin{align}
    L_A^\dagger & = \left(I - \left(A (I - L^\dagger L)\right)^\dagger A\right)L^\dagger \notag \\
                & = (I - V_2 U^{-1} Q_1^* A) V_1 R^{-*}
\end{align}
となるから,
\begin{align}
    \bar{A} & = A L_A^\dagger \notag                         \\
            & = A (I - V_2 U^{-1} Q_1^* A) V_1 R^{-*} \notag \\
            & = (A - A V_2 U^{-1} Q_1^* A) V_1 R^{-*} \notag \\
            & = (I - A V_2 U^{-1} Q_1^*) A V_1 R^{-*} \notag \\
            & = (I - Q_1 U U^{-1} Q_1^*) A V_1 R^{-*} \notag \\
            & = (I - Q_1 Q_1^*) A V_1 R^{-*} \notag          \\
            & = Q_2 Q_2^* A V_1 R^{-*}
\end{align}
と変形できる.
そして,$\bar{\bm{x}}$ から $\bm{x}$への変換は
\begin{align}
    \bm{x} & = L_A^\dagger \bar{\bm{x}} + \bm{x}_0 \notag                                        \\
           & = (I - V_2 U^{-1} Q_1^* A) V_1 R^{-*} \bar{\bm{x}} + V_2 U^{-1} Q_1^* \bm{y} \notag \\
           & = (I - V_2 U^{-1} Q_1^* A) L^\dagger \bar{\bm{x}} + V_2 U^{-1} Q_1^* \bm{y} \notag  \\
           & = L^\dagger \bar{\bm{x}} + V_2 U^{-1} Q_1^* (\bm{y} - A L^\dagger \bar{\bm{x}})
\end{align}
でできる.
さらに,$Q_2^* Q_2 = I$を用いれば,
\begin{align}
    \left\|\bar{A} \bar{\bm{x}} - \bar{\bm{y}}\right\|_2
                   & = \left\|\tilde{A} \bar{\bm{x}} - \tilde{\bm{y}}\right\|_2, &
    \tilde{A}      & = Q_2^* A V_1 R^{-*},                                       &
    \tilde{\bm{y}} & = Q_2^* \bm{y}
\end{align}
のように書き換えることもできる.

\subsection{行列 L が正則な正方行列の場合の計算方法}

行列 $L$ が正則な場合,$L^\dagger = L^{-1}$ となるため,
式がかなり単純になる.

まず,$P_L$ は
\begin{equation}
    P_L = I - L^\dagger L = I - L^{-1} L = O
\end{equation}
となる.これを用いると,$L_A^\dagger$ は
\begin{equation}
    L_A^\dagger = \left(I - \left(A P_L\right)^\dagger A\right) L^\dagger
    = \left(I - O^{\dagger} A\right) L^{-1}
    = \left(I - O A\right) L^{-1}
    = L^{-1}
\end{equation}
となる.これらにより,変換の公式は以下のようになる.

\begin{align}
    \bm{x}       & = L^{-1} \bar{\bm{x}} \\
    \bar{A}      & = A L^{-1}            \\
    \bar{\bm{y}} & = \bm{y}
\end{align}

\subsection{行列 L が一般の行列の場合の計算方法}

まず,$L$ を特異値分解する.
\begin{equation}
    L = W
    \begin{pmatrix}
        \Omega & O \\
        O      & O
    \end{pmatrix}
    V^*
\end{equation}
ここで,$L$ はランク $r$ で
$W \in \setC^{p \times p}$,
$V \in \setC^{n \times n}$
はユニタリ行列,
$\Omega \in \setR^{r \times r}$
は正数の対角成分による対角行列とする.
このとき,$W = (W_1, W_2)$ なる
行列 $W_1 \in \setC^{p \times r}$ と $W_2 \in \setC^{p \times (p-r)}$,
$V = (V_1, V_2)$ なる
行列 $V_1 \in \setC^{n \times r}$ と $V_2 \in \setC^{n \times (n-r)}$
を定義する.

このとき,$L$ が列と同じ数のランクを持つ場合と同様にして
\begin{align}
    \bm{x}_0       & = V_2 U^{-1} Q_1^* \bm{y}                        \\
    \bar{\bm{y}}   & = Q_2 Q_2^* \bm{y}                               \\
    L_A^\dagger    & = (I - V_2 U^{-1} Q_1^* A) V_1 \Omega^{-1} W_1^* \\
    \bar{A}        & = Q_2 Q_2^* A V_1 \Omega^{-1} W_1^*              \\
    \tilde{A}      & = Q_2^* A V_1 \Omega^{-1} W_1^*                  \\
    \tilde{\bm{y}} & = Q_2^* \bm{y}
\end{align}
が得られる.
