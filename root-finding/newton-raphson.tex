% !TEX root = ../main.tex
%

\chapter{Newton-Raphson 法}

Newton-Raphson 法は、
次の式のような更新式で反復的に方程式 $f(x) = 0$ の根を求める手法である。
\begin{equation}
    x_{n+1} = x_n - \frac{f(x_n)}{f'(x_n)}
\end{equation}

\section{1 次元の方程式の場合}

1 次元の方程式 $f(x) = 0$ ($f: \setR \to \setR$) の場合、
1 次近似
\begin{equation}
    f(x_{n+1}) \approx f(x_n) + f'(x_n) (x_{n+1} - x_n)
\end{equation}
の根を求めると
\begin{equation}
    x_{n+1} = x_n - \frac{f(x_n)}{f'(x_n)}
    \label{eq:root-finding_newton-raphson_one-dim-update-law}
\end{equation}
と更新式が導かれる。

\subsection{平方根の算出}

ここでは、平方根の算出に Newton-Raphson 法を適用してみる。

$a>0$ について $\sqrt{a}$ を算出する場合、次の関数 $f$ の正の根を求めれば良い。
\begin{equation}
    f(x) = x^2 - a
\end{equation}
この関数を微分すると
\begin{equation}
    f'(x) = 2x
\end{equation}
となる。
よって、更新式は次のようになる。
\begin{align}
    x_{n+1} & = x_n - \frac{f(x_n)}{f'(x_n)}                \\
            & = x_n - \frac{x_n^2 - a}{2x_n}                \\
            & = \frac{1}{2}\left(x_n + \frac{a}{x_n}\right)
    \label{eq:root-finding_newton-raphson_sqrt-update-law}
\end{align}

\begin{theorem}
    式 \eqref{eq:root-finding_newton-raphson_sqrt-update-law} の更新式は、
    初期値 $x_0$ が $x_0 > 0$ を満たす場合、
    平方根 $\sqrt{a}$ に収束する。
\end{theorem}
\begin{proof}
    $x_0 = \sqrt{a}$ の場合は
    $k = 1, 2, \ldots$ において $x_k = \sqrt{a}$ が成り立つ。
    つまり、平方根 $\sqrt{a}$ に収束している。
    そこで、以下では $x_0 \neq \sqrt{a}$ とする。

    更新後の値 $x_{n+1}$ と平方根 $\sqrt{a}$ の差は次のようになる。
    \begin{align}
          & x_{n+1} - \sqrt{a}                                                      \\
        = & \frac{1}{2} (x_n - \sqrt{a}) + \frac{1}{2} \frac{a - \sqrt{a} x_n}{x_n} \\
        = & \frac{1}{2} (x_n - \sqrt{a}) \left(1 - \frac{\sqrt{a}}{x_n}\right)      \\
        = & \frac{1}{2x_n} (x_n - \sqrt{a})^2
        \label{eq:root-finding_newton-raphson_sqrt-error-update}
    \end{align}

    $x_0 > 0$ かつ $x_0 \neq \sqrt{a}$ の場合、
    式 \eqref{eq:root-finding_newton-raphson_sqrt-error-update} より
    $x_1 - \sqrt{a} > 0$ が成り立つ。
    さらに、$k = 2, 3, \ldots$ において $x_k - \sqrt{a} > 0$ であることも帰納的に成り立つ。
    よって、$k = 1, 2, \ldots$ において $x_k - \sqrt{a} > 0$ である。

    また、$a > 0$ より $\sqrt{a} > 0$ であることを用いると、
    \begin{align}
          & x_{n+1} - \sqrt{a}                           \\
        = & \frac{1}{2x_n} (x_n - \sqrt{a})^2            \\
        = & \frac{x_n - \sqrt{a}}{2x_n} (x_n - \sqrt{a}) \\
        < & \frac{x_n}{2x_n} (x_n - \sqrt{a})            \\
        = & \frac{1}{2} (x_n - \sqrt{a})
    \end{align}
    となる。

    以上から、
    \begin{equation}
        0 < x_{n+1} - \sqrt{a} < \frac{1}{2} (x_n - \sqrt{a})
    \end{equation}
    が成り立ち、$x_n$ の列が平方根 $\sqrt{a}$ へ収束する。
\end{proof}
