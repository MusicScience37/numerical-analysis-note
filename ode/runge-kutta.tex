% !TEX root = ../main.tex
%

\chapter{Runge-Kutta 法}

Runge-Kutta 法 (Runge-Kutta method) では次のような形式の初期値問題を数値的に解く
\cite{Mitsui1993}.
\begin{equation}
    \begin{cases}
        \dot{\bm{y}} = \bm{f}(t, \bm{y}) \\
        \bm{y}(0) = \bm{y}_0
    \end{cases}
\end{equation}

Runge-Kutta 法は,時刻 $t$ における変数 $\bm{y}(t)$ の値から,
次のような形式で時刻 $t + h$ における変数 $\bm{y}(t + h)$ の計算を行う.
\begin{align}
    \bm{k}_i      & = \bm{f}\left(t + c_i h, \bm{y}(t) + h \sum_{j = 1}^s a_{ij} \bm{k}_j \right)
                  & \text{for $i = 1, 2, \ldots, s$}
    \label{eq:ode_runge-kutta_k-law}                                                              \\
    \bm{y}(t + h) & = \bm{y}(t) + \sum_{i=1}^s b_i \bm{k}_i
    \label{eq:ode_runge-kutta_y-law}
\end{align}

ここで,時間の更新幅 $h$ はステップ幅と呼ばれる.
Runge-Kutta 法には,
整数 $s$ (段数と呼ばれる)と係数 $a_{ij}$, $b_i$, $c_i$ の異なる様々な公式が存在する.
係数 $a_{ij}$, $b_i$, $c_i$ は
表 \ref{table:ode_runge-kutta_butcher-array-general} のような
Butcher 配列と呼ばれる形式で記載される.

\begin{table}[bp]
    \caption{Butcher 配列における係数の並べ方}
    \label{table:ode_runge-kutta_butcher-array-general}
    \centering
    \begin{tabular}{c|ccccc}
        $c_1$    & $a_{11}$ & $a_{12}$ & $a_{13}$ & $\cdots$ & $a_{1s}$ \\
        $c_2$    & $a_{21}$ & $a_{22}$ & $a_{23}$ & $\cdots$ & $a_{2s}$ \\
        $c_3$    & $a_{31}$ & $a_{32}$ & $a_{33}$ & $\cdots$ & $a_{3s}$ \\
        $\vdots$ & $\vdots$ & $\vdots$ & $\vdots$ & $\ddots$ & $\vdots$ \\
        $c_s$    & $a_{s1}$ & $a_{s2}$ & $a_{s3}$ & $\cdots$ & $a_{ss}$ \\
        \hline
                 & $b_1$    & $b_2$    & $b_3$    & $\cdots$ & $b_s$
    \end{tabular}
\end{table}

Runge-Kutta 法の公式は次のように分類される.

\begin{description}
    \item[陽的 Runge-Kutta 法] $j \ge i$ について $a_{ij} = 0$ となっている場合,
          $\bm{k}i$ は $\bm{k}_1, \bm{k}_2, \ldots, \bm{k}_s$
          の順に式 \eqref{eq:ode_runge-kutta_k-law} の右辺を評価することで計算できる.
          このような場合は陽的 Runge-Kutta 法と呼ばれる.
    \item[半陰的 Runge-Kutta 法] $j > i$ について $a_{ij} = 0$ となっている場合,
          $\bm{k}i$ は $\bm{k}_1, \bm{k}_2, \ldots, \bm{k}_s$
          の順に式 \eqref{eq:ode_runge-kutta_k-law} を $\bm{k}_i$ について解くことで計算できる.
          このような場合は半陰的 Runge-Kutta 法と呼ばれる.
    \item[陰的 Runge-Kutta 法] $j > i$ でも $a_{ij} \neq 0$ となる係数が存在する場合,
          $\bm{k}i$ は $i = 1, 2, \ldots, s$ について連立した
          式 \eqref{eq:ode_runge-kutta_k-law} を $\bm{k}_i$ について解くことで計算する.
          このような場合は陰的 Runge-Kutta 法と呼ばれる.
\end{description}

陽的 Runge-Kutta 法の方が計算は単純だが,
陰的 Runge-Kutta 法は

\begin{itemize}
    \item 硬い系と呼ばれる比較的不安定な微分方程式で解が安定しやすい.
    \item 陽的 Runge-Kutta 法よりも少ない段数でより高い次数を出せる.
          (後述する公式の実例を見ると分かる.)
\end{itemize}

といった利点を持つ.
そのため,解が安定するようにステップ幅を調整した場合,
陰的 Runge-Kutta 法の方がステップ幅を大きくとることができ,
目的の時刻 $t$ までの解を得るために必要な計算時間は少なくなる場合がある.

また,Runge-Kutta 法の公式の精度を示す数値として次数が存在する.
変数の近似値 $\bm{y}(t + h)$ の精度が $h^p$ オーダーの場合,
その公式は $p$ 次といい,$p$ は次数と呼ばれる.

\section{埋め込み型公式}

Runge-Kutta 法の公式の中には,複数の $b_i$ の組を持つものがある
(表 \ref{table:ode_runge-kutta_butcher-array-rkf45} の例を参照).
そのような公式では,次のような 2 つの近似値を得ることができる.
\begin{align}
    \bm{y}(t + h)   & = \bm{y}(t) + \sum_{i=1}^s b_i \bm{k}_i     \\
    \bm{y}^*(t + h) & = \bm{y}^*(t) + \sum_{i=1}^s b_i^* \bm{k}_i
\end{align}
これらの差により誤差の近似値を求めることができる.
\begin{align}
    \bm{y}(t + h) - \bm{y}^*(t + h) & = \sum_{i=1}^s (b_i - b_i^*) \bm{k}_i
\end{align}

これを用いると,現在のステップ幅から次のステップ幅 $\hat{h}$ の適正値を推定できる.
まず,$b_i$ と $b_i^*$ のうち次数が低い方の次数を $p$ とすると,
\begin{align}
    \left| \bm{y}(t + h) - \bm{y}^*(t + h) \right| \approx |Ah^p|
\end{align}
のように書ける.
そこで,誤差の許容量を $\epsilon_{tol}$ としたとき,次の方程式が成り立つようにする.
\begin{equation}
    \frac{\hat{h}^p}{h^p} = \frac{\epsilon_{tol}}{\left| \bm{y}(t + h) - \bm{y}^*(t + h) \right|}
\end{equation}
これを $\hat{h}$ について解くと,次のようになる.
\begin{equation}
    \hat{h} = h \left(\frac{\epsilon_{tol}}{\left| \bm{y}(t + h) - \bm{y}^*(t + h) \right|}\right)^{\frac{1}{p}}
\end{equation}

\section{古典的 Runge-Kutta 法}

\begin{table}[bp]
    \caption{古典的 Runge-Kutta 法 (RK4 公式)の Butcher 配列}
    \label{table:ode_runge-kutta_butcher-array-rk4}
    \centering
    \begin{tabular}{c|cccc}
        $0$           &               &               &               &               \\
        $\frac{1}{2}$ & $\frac{1}{2}$ &               &               &               \\
        $\frac{1}{2}$ & $0$           & $\frac{1}{2}$ &               &               \\
        $1$           & $0$           & $0$           & $1$           &               \\
        \hline
                      & $\frac{1}{6}$ & $\frac{1}{3}$ & $\frac{1}{3}$ & $\frac{1}{6}$
    \end{tabular}
\end{table}

古典的 Runge-Kutta 法と呼ばれる公式では,
表 \ref{table:ode_runge-kutta_butcher-array-rk4} のような係数を用いる
\cite[3.3 節]{Mitsui1993}.
単純な係数で次数 4 を達成できる.

\section{RKF45 公式}

\begin{table}[bp]
    \caption{RKF45 公式の Butcher 配列}
    \label{table:ode_runge-kutta_butcher-array-rkf45}
    \centering
    \begin{tabular}{c|ccccccc}
        $0$             &                     &                      &                      &                       &                  &                &          \\
        $\frac{1}{4}$   & $\frac{1}{4}$       &                      &                      &                       &                  &                &          \\
        $\frac{3}{8}$   & $\frac{3}{32}$      & $\frac{9}{32}$       &                      &                       &                  &                &          \\
        $\frac{12}{13}$ & $\frac{1932}{2197}$ & $-\frac{7200}{2197}$ & $\frac{7296}{2197}$  &                       &                  &                &          \\
        $1$             & $\frac{439}{216}$   & $-8$                 & $\frac{3680}{513}$   & $-\frac{845}{4104}$   &                  &                &          \\
        $\frac{1}{2}$   & $-\frac{8}{27}$     & $2$                  & $-\frac{3544}{2565}$ & $\frac{1859}{4104}$   & $-\frac{11}{40}$ &                &          \\
        \hline
                        & $\frac{16}{135}$    & $0$                  & $\frac{6656}{12825}$ & $\frac{28561}{56430}$ & $-\frac{9}{50}$  & $\frac{2}{55}$ & (5 次) \\
                        & $\frac{25}{216}$    & $0$                  & $\frac{1408}{2565}$  & $\frac{2197}{4104}$   & $-\frac{1}{5}$   & $0$            & (4 次)
    \end{tabular}
\end{table}

RKF45 公式(RKF は Runge-Kutta-Fehlberg のこと)では,
表 \ref{table:ode_runge-kutta_butcher-array-rkf45} のような係数を用いる
\cite[4.1 節 (a)]{Mitsui1993}, \cite[Section 9.5]{Mathews2004}.
この埋め込み型公式では,$\bm{k}_i$ から $\bm{y}(t + h)$ を算出する係数に
5 次の精度を持つ組と 4 次の精度を持つ組が存在する
\footnote{挙げた 2 件の文献のうち,%
    文献 \cite{Mitsui1993} では係数が一カ所誤っていたため注意が必要.}.

\section{田中公式}

\begin{table}[bp]
    \caption{田中 Formula1 公式の Butcher 配列}
    \label{table:ode_runge-kutta_butcher-array-tanaka-formula1}
    \centering
    \begin{tabular}{c|ccc}
        $\frac{13}{20}$ & $\frac{13}{20}$    &                  &          \\
        $-\frac{1}{18}$ & $-\frac{127}{180}$ & $\frac{13}{20}$  &          \\
        \hline
                        & $\frac{100}{127}$  & $\frac{27}{127}$ & (3 次) \\
                        & $1$                &                  & (1 次)
    \end{tabular}
\end{table}

\begin{table}[bp]
    \caption{田中 Formula2 公式の Butcher 配列}
    \label{table:ode_runge-kutta_butcher-array-tanaka-formula2}
    \centering
    \begin{tabular}{c|cccc}
        $\frac{133}{100}$ & $\frac{133}{100}$     &                       &                      &          \\
        $\frac{1}{2}$     & $-\frac{5400}{18167}$ & $\frac{28967}{36334}$ &                      &          \\
        $-\frac{33}{100}$ & $\frac{133}{50}$      & $-\frac{108}{25}$     & $\frac{133}{100}$    &          \\
        \hline
                          & $\frac{1250}{20667}$  & $\frac{18167}{20667}$ & $\frac{1250}{20667}$ & (4 次) \\
                          & $0$                   & $1$                   &                      & (2 次)
    \end{tabular}
\end{table}

文献 \cite{Togawa2007} では,田中正次氏が考案した半陰的・陰的公式がいくつか示されている.
そのうち埋め込み型の半陰的公式を表
\ref{table:ode_runge-kutta_butcher-array-tanaka-formula1},
\ref{table:ode_runge-kutta_butcher-array-tanaka-formula2}
に示す
\footnote{文献 \cite{Togawa2007} には完全に陰的な 4 段 7 次の公式もあるが,%
    係数がかなり複雑なため省略した.}.

段数の多い陽的公式と同程度の精度を少ない段数で出すことができていることを確認できる.
また,埋め込み型の 2 つの係数のうち次数の低い方の係数が単純になっている
\footnote{利用する際には高い次数の係数との差を見るように実装するため,%
    次数の低い方の係数だけ単純でも計算コストへの影響はない.}.

\section{Euler 法}

\begin{table}[bp]
    \caption{Euler 法の Butcher 配列}
    \label{table:ode_runge-kutta_butcher-array-explicit-euler}
    \centering
    \begin{tabular}{c|c}
        $1$ &     \\
        \hline
            & $1$
    \end{tabular}
\end{table}

常微分方程式の解法としては最も基本的な Euler 法
\begin{equation}
    \bm{y}(t + h) \approx \bm{y}(t) + h \bm{f}(t, \bm{y}(t))
\end{equation}
は形式的に Runge-Kutta 法とみなすことができる.
Butcher 配列は表 \ref{table:ode_runge-kutta_butcher-array-explicit-euler} に示す通りである.
