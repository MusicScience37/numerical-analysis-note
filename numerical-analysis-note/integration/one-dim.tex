% !TEX root = ../main.tex
%

\chapter{1 次元の数値積分}

本章では,
1 次元における積分
\begin{equation}
    \int_{a}^{b} f(x) dx
\end{equation}
に対する数値積分の手法についてまとめる.

\section{Gauss 積分公式}

1 次元における積分に対する数値積分の手法の 1 つに
\index{Gaussせきぶんこうしき@Gauss 積分公式} Gauss 積分公式がある.
($a$, $b$ は $-\infty$, $\infty$ でも良い.)

Gauss 積分公式では,次のように $x_1, x_2, \ldots, x_n$ 上での関数値の重み付き平均を用いる
\cite{Mori1993}.
\begin{equation}
    \int_{a}^{b} f(x) w(x) dx \approx \sum_{k = 1}^n w_k f(x_k)
    \label{eq:integration_one-dim_gauss-general}
\end{equation}
分点 $x_1, x_2, \ldots, x_n$ と重み $w_1, w_2, \ldots, w_n$ は,
区間 $(a, b)$ と重み関数 $w(x)$ に応じて決まる.

\subsection{Gauss-Legendre 公式}

$a$, $b$ が有限な場合に利用できる
\index{GaussLegendreせきぶんこうしき@Gauss-Legendre 積分公式}
Gauss-Legendre 公式では,
Legendre 関数 $P_n(x)$ の $n$ 個の零点を分点 $x_k$ に用い,
重み関数は次のように算出する\cite{Mori1993}.
\begin{equation}
    w_k = \frac{2(1 - x_k^2)}{(n P_{n-1}(x_k))^2}
\end{equation}
分点 $x_k$ の算出には Newton-Raphson 法(\ref{chap:root-finding_newton-raphson} 章)を用いることができる.
初期値には
\begin{equation}
    x_k \approx \cos{\frac{\pi (k - 0.25)}{n + 0.5}}
\end{equation}
を使用すると良い\cite{Mori1993}.
また,$s_k = -x_{n-k}, w_k = w_{n-k}$ を用いて計算量を半減させることができる.

\subsection{Gauss-Kronrod 積分公式}\label{sec:integration_one-dim_gauss-kronrod}

数値積分した際に,計算結果の誤差を評価したい場合がある
\footnote{利用例は適応積分(\ref{chap:adaptive-integration} 章)を参照.}.
そこで,異なる近似値 2 つを用意し,その 2 つの差を誤差のオーダーの評価に用いる.
同じ積分公式で次数を変えた 2 種類の分点 $x_k$ と重み $w_k$ を用いても良いが,
計算効率を上げるため分点 $x_k$ を使い回せるように考えられた
\index{うめこみがたのこうしき@埋め込み型の公式}
「埋め込み型の公式」が存在する.
埋め込み型の公式では,同じ分点で次のような 2 種類の近似値を求めることができる.
\begin{align}
    I   & = \sum_{k = 1}^n w_k f(x_k)   \\
    I^* & = \sum_{k = 1}^n w_k^* f(x_k)
\end{align}
この 2 つの値の差を利用して誤差の評価を行う.

Gauss 積分公式において,追加の分点を加えることで追加の近似値を算出する手法として,
\index{GaussKronrodせきぶんこうしき@Gauss-Kronrod 積分公式}
Gauss-Kronrod 積分公式が考えられており,
行列演算により追加の分点と重みを算出する方法も考えられている
\cite{Laurie1997}.
細かいアルゴリズムについては文献 \cite{Laurie1997}
\footnote{文献 \cite{Laurie1997} の前提として文献 \cite{Golub1969} も参照しておくこと.%
    特に,Gauss-Legendre 積分公式から埋め込み型の公式を算出するには%
    文献 \cite{Golub1969} の内容が必要.}
や
Gauss-Legendre 公式に対する適応例 \cite{NumericalCollectionCpp} を参照.

\section{二重指数関数型公式}

1 次元の数値積分において,
Gauss 公式よりも困難な積分へ対応できる手法として,
\index{にじゅうしすうかんすうがたこうしき@二重指数関数型公式}
\index{Double Exponential Formula|see{二重指数関数型公式}}
二重指数関数型公式(Double Exponential Formula, DE Formula)が存在する
\cite[6.1 節 (b)]{Mori1993}, \cite[Section 4.5]{Press2007}.
二重指数関数型公式では,
区間 $(a, b)$ 上での積分において次の式による変数変換を行い,
区間 $(-\infty, \infty)$ 上での積分にする.
\begin{equation}
    x = \frac{1}{2}(a + b) + \frac{1}{2}(b - a) \tanh \left(\frac{\pi}{2} \sinh{t}\right)
\end{equation}
$\tanh$, $\sinh$ はどちらも指数関数で定義されるため,
名前通り二重の指数関数による積分となっている.

二重指数関数型公式は,
無限積分
\begin{equation}
    \int_{-\infty}^{\infty} f(x) dx
\end{equation}
が有限和
\begin{equation}
    h \sum_{k = -\infty}^{\infty} f(kh)
\end{equation}
でよく近似できるということに基づいている\cite[6.1 節 (b)]{Mori1993}.
二重指数関数型公式では,
積分領域の端で被積分関数が発散する
\begin{equation}
    \int_{-1}^{1} \frac{1}{\sqrt{1-x^2}} dx
\end{equation}
のような積分でも,
積分領域の端を無限遠へ移すことにより,
Gauss-Legendre 積分公式よりも安定して数値積分を行うことができる.
ただし,指数関数を何度も計算する必要があるため,
二重指数関数型公式の方が計算時間は長くなる可能性がある.

\subsection{有限区間上の積分}

1 次元の有限区間上の積分
\begin{equation}
    I = \int_{a}^{b} f(x) dx
\end{equation}
を考える.変数変換
\begin{equation}
    x = \phi(t) \equiv \frac{1}{2}(a + b) + \frac{1}{2}(b - a) \tanh \left(\frac{\pi}{2} \sinh{t}\right)
\end{equation}
を適用する場合,
\begin{equation}
    \frac{dx}{dt} = \phi'(t)
    = \frac{\pi}{4} (b - a) \frac{\cosh{t}}{\cosh^2 \left(\frac{\pi}{2} \sinh{t}\right)}
\end{equation}
を用いて
\begin{equation}
    I = \int_{a}^{b} f(x) dx
    = \int_{-\infty}^{\infty} f(\phi(t)) \phi'(t)
\end{equation}
のように近似でき,
\begin{equation}
    T_h = h \sum_{k = -N}^{N} f(\phi(kh)) \phi'(kh)
\end{equation}
で近似できる.
また,パラメータ $h$ の最適値は
\begin{equation}
    h \approx \frac{\log(2 \pi N)}{N}
\end{equation}
であり,この場合の数値積分の誤差のオーダーは
\begin{equation}
    |T_h - I| \approx \exp(-kN / \log{N})
\end{equation}
となる\cite[Section 4.5]{Press2007}.
点数 $N$ を 2 倍にすると有効桁数が約 2 倍になる.

なお,実装時はオーバーフローや桁落ちを防ぐため
\begin{align}
    b - x
     & = \frac{1}{2}(b - a) \left(1 - \tanh \left(\frac{\pi}{2} \sinh{t}\right) \right)
    \notag                                                                                 \\
     & = \frac{1}{2}(b - a) \left(1 -
    \frac{\exp\left(\frac{\pi}{2} \sinh{t}\right) - \exp\left(-\frac{\pi}{2} \sinh{t}\right)}%
    {\exp\left(\frac{\pi}{2} \sinh{t}\right) + \exp\left(-\frac{\pi}{2} \sinh{t}\right)}\right)
    \notag                                                                                 \\
     & = \frac{1}{2}(b - a) \left(1 -
    \frac{1 - \exp\left(-\pi \sinh{t}\right)}{1 + \exp\left(-\pi \sinh{t}\right)}\right)
    \notag                                                                                 \\
     & = (b - a) \frac{\exp\left(-\pi \sinh{t}\right)}{1 + \exp\left(-\pi \sinh{t}\right)}
\end{align}
\begin{align}
    \phi'(t)
     & = \frac{\pi}{4} (b - a) \frac{\cosh{t}}{\cosh^2 \left(\frac{\pi}{2} \sinh{t}\right)}
    \notag                                                                                  \\
     & = \frac{\pi}{4} (b - a) \frac{4 \cosh{t}}                                            %
    {(\exp\left(\frac{\pi}{2} \sinh{t}\right) + \exp\left(-\frac{\pi}{2} \sinh{t}\right))^2}
    \notag                                                                                  \\
     & = \pi (b - a) \frac{\cosh{t} \exp(-\pi \sinh{t})}{(1 + \exp(-\pi \sinh{t}))^2}
\end{align}
とすると良い\cite[Section 4.5.2]{Press2007}.

\subsection{半無限区間上の積分}

1 次元の半無限区間上の積分
\begin{equation}
    I = \int_{0}^{\infty} f(x) dx
\end{equation}
を考える.
これに対する二重指数関数型公式の変数変換は
\begin{equation}
    x = \phi(t) \equiv \exp(\pi \sinh{t})
\end{equation}
である\cite[Section 4.5.3]{Press2007}.
微分すると
\begin{equation}
    \frac{dx}{dt} = \phi'(t)
    = \pi \exp(\pi \sinh{t}) \cosh{t}
\end{equation}
となる.

半無限区間上の積分,および次節の無限区間上の積分においては,
変数変換後の変数 $t$ において区間 $[-4, 4]$ までの範囲で有限和による近似を行えば良い
\cite[4.5.3]{Press2007}.

\subsection{無限区間上の積分}

1 次元の無限区間上の積分
\begin{equation}
    I = \int_{-\infty}^{\infty} f(x) dx
\end{equation}
を考える.
これに対する二重指数関数型公式の変数変換は
\begin{equation}
    x = \phi(t) \equiv \sinh \left(\frac{\pi}{2} \sinh{t}\right)
\end{equation}
である\cite[Section 4.5.3]{Press2007}.
微分すると
\begin{equation}
    \frac{dx}{dt} = \phi'(t)
    = \frac{\pi}{2} \cosh \left(\frac{\pi}{2} \sinh{t}\right) \cosh{t}
\end{equation}
となる.

\section{適用例}

\subsection{有限区間上の積分}

以下の積分を数値積分で計算した.
\begin{gather}
    \int_{0}^{1} e^x dx = e - 1
    \label{eq:integration_one-dim_finite-exp-integration}
    \\
    \int_{-1}^{1} \sqrt{1-x^2} dx = \frac{\pi}{2}
    \label{eq:integration_one-dim_finite-sqrt-1mx2-integration}
    \\
    \int_{-1}^{1} \frac{1}{\sqrt{1-x^2}} dx = \pi
    \label{eq:integration_one-dim_finite-inv-sqrt-1mx2-integration}
\end{gather}
上から順に,
\begin{itemize}
    \item 滑らかな関数の積分(最も簡単な例)
    \item 積分区間の端で急変する関数(滑らかな関数よりも難易度が高い)
    \item 積分区間の端で発散する関数(特異積分)
\end{itemize}
となっている.

\begin{figure}[tp]
    \centering
    \includegraphics[width=0.99\linewidth]{plots/integ-gauss-legendre-errors.pdf}
    \caption{Legendre 関数の次数の異なる Gauss-Legendre 公式による数値積分の誤差}
    \label{fig:integration_one-dim_finite_gauss-legendre-errors}
\end{figure}

まず,Gauss-Legendre 公式で積分したときの誤差を
図 \ref{fig:integration_one-dim_finite_gauss-legendre-errors} に示す.
次数を増やすと精度が良くなる様子を確認できる.
また,簡単な式 \eqref{eq:integration_one-dim_finite-exp-integration} の積分では誤差が小さいのに対し,
積分区間の端で急変する式 \eqref{eq:integration_one-dim_finite-sqrt-1mx2-integration} の積分では誤差が大きくなり,
積分区間の端で発散する式 \eqref{eq:integration_one-dim_finite-inv-sqrt-1mx2-integration} の積分ではさらに誤差が大きくなっている.

\begin{figure}[tp]
    \centering
    \includegraphics[width=0.99\linewidth]{plots/integ-gauss-legendre-kronrod-time.pdf}
    \caption{Legendre 関数の次数の異なる Gauss-Legendre-Kronrod 公式による数値積分の計算時間}
    \label{fig:integration_one-dim_finite_gauss-legendre-kronrod-time}
\end{figure}

Gauss-Legendre-Kronrod 公式で適応積分したときの計算時間を
図 \ref{fig:integration_one-dim_finite_gauss-legendre-kronrod-time} に示す.
計算結果の誤差はどれも $10^{-7}$ 程度となっており,
その精度の値を計算するのにかかった時間が対象の積分や Legendre 関数の次数に応じて変化している.
まず,対象の積分の難易度が高くなると計算時間が長くなった.
積分区間を細かく分割して積分しないと十分な精度が出せないため,計算時間が長くなっていくものと考えられる.
また,Legendre 関数の次数が小さい場合も計算時間が長くなっている.
一回の積分公式の適用で得られる積分値の精度が低いため,
積分区間を細かく分割して積分することになり,計算時間が長くなっていくものと考えられる.
一方,次数を大きくしすぎた場合も計算時間が少し長くなっている.
分割した個々の区間に対する積分の計算に時間がかかるためと考えられる.

\begin{figure}[tp]
    \centering
    \includegraphics[width=0.99\linewidth]{plots/integ-de-finite-time.pdf}
    \caption{点数の異なる二重指数関数型公式による数値積分の計算時間}
    \label{fig:integration_one-dim_finite_de-errors}
\end{figure}

二重指数関数型公式で積分したときの誤差を
図 \ref{fig:integration_one-dim_finite_de-errors} に示す.
点数の多いときの結果がないのは,精度が倍精度浮動小数点数の桁数を超えてしまって
誤差の値が 0 となってしまったためである.
二重指数関数型公式は積分区間の端で被積分関数が急変したり発散したりするような難しい積分に強く,
どの積分でも計算に用いる点数を増やした場合は誤差が同程度に小さくなる.
