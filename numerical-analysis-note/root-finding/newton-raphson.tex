% !TEX root = ../main.tex
%

\chapter{Newton-Raphson 法}\label{chap:root-finding_newton-raphson}

\index{Newton-Raphson ほう@Newton-Raphson 法}
\index{Newtonほう@Newton 法}
Newton-Raphson 法(または単に「Newton 法」とも呼ばれる)は,
次の式のような更新式で反復的に方程式 $f(x) = 0$ の根を求める手法である.
\begin{equation}
    x_{n+1} = x_n - \frac{f(x_n)}{f'(x_n)}
\end{equation}

\section{1 次元の方程式の場合}

1 次元の方程式 $f(x) = 0$ ($f: \setR \to \setR$) の場合,
1 次近似
\begin{equation}
    f(x_{n+1}) \approx f(x_n) + f'(x_n) (x_{n+1} - x_n)
\end{equation}
の根を求めると
\begin{equation}
    x_{n+1} = x_n - \frac{f(x_n)}{f'(x_n)}
    \label{eq:root-finding_newton-raphson_one-dim-update-law}
\end{equation}
と更新式が導かれる.

\subsection{平方根の算出}

ここでは,平方根の算出に Newton-Raphson 法を適用してみる.

$a>0$ について $\sqrt{a}$ を算出する場合,次の関数 $f$ の正の根を求めれば良い.
\begin{equation}
    f(x) = x^2 - a
\end{equation}
この関数を微分すると
\begin{equation}
    f'(x) = 2x
\end{equation}
となる.
よって,更新式は次のようになる.
\begin{align}
    x_{n+1} & = x_n - \frac{f(x_n)}{f'(x_n)}          \notag \\
            & = x_n - \frac{x_n^2 - a}{2x_n}          \notag \\
            & = \frac{1}{2}\left(x_n + \frac{a}{x_n}\right)
    \label{eq:root-finding_newton-raphson_sqrt-update-law}
\end{align}

\begin{theorem}
    式 \eqref{eq:root-finding_newton-raphson_sqrt-update-law} の更新式は,
    初期値 $x_0$ が $x_0 > 0$ を満たす場合,
    平方根 $\sqrt{a}$ に収束する.
\end{theorem}
\begin{proof}
    $x_0 = \sqrt{a}$ の場合は
    $k = 1, 2, \ldots$ において $x_k = \sqrt{a}$ が成り立つ.
    つまり,平方根 $\sqrt{a}$ に収束している.
    そこで,以下では $x_0 \neq \sqrt{a}$ とする.

    更新後の値 $x_{n+1}$ と平方根 $\sqrt{a}$ の差は次のようになる.
    \begin{align}
          & x_{n+1} - \sqrt{a}                                                      \notag \\
        = & \frac{1}{2} (x_n - \sqrt{a}) + \frac{1}{2} \frac{a - \sqrt{a} x_n}{x_n} \notag \\
        = & \frac{1}{2} (x_n - \sqrt{a}) \left(1 - \frac{\sqrt{a}}{x_n}\right)      \notag \\
        = & \frac{1}{2x_n} (x_n - \sqrt{a})^2
        \label{eq:root-finding_newton-raphson_sqrt-error-update}
    \end{align}

    $x_0 > 0$ かつ $x_0 \neq \sqrt{a}$ の場合,
    式 \eqref{eq:root-finding_newton-raphson_sqrt-error-update} より
    $x_1 - \sqrt{a} > 0$ が成り立つ.
    さらに,$k = 2, 3, \ldots$ において $x_k - \sqrt{a} > 0$ であることも帰納的に成り立つ.
    よって,$k = 1, 2, \ldots$ において $x_k - \sqrt{a} > 0$ である.

    また,$a > 0$ より $\sqrt{a} > 0$ であることを用いると,
    \begin{align}
          & x_{n+1} - \sqrt{a}                           \notag \\
        = & \frac{1}{2x_n} (x_n - \sqrt{a})^2            \notag \\
        = & \frac{x_n - \sqrt{a}}{2x_n} (x_n - \sqrt{a}) \notag \\
        < & \frac{x_n}{2x_n} (x_n - \sqrt{a})            \notag \\
        = & \frac{1}{2} (x_n - \sqrt{a})
    \end{align}
    となる.

    以上から,
    \begin{equation}
        0 < x_{n+1} - \sqrt{a} < \frac{1}{2} (x_n - \sqrt{a})
    \end{equation}
    が成り立ち,$x_n$ の列が平方根 $\sqrt{a}$ へ収束する.
\end{proof}

\subsection{べき根の算出}

一般に $r$ 乗根($r = 2, 3, \ldots$)を算出することを考える.

$a>0$ について $\sqrt[r]{a}$ を算出する場合,次の関数 $f$ の正の根を求めれば良い.
\begin{equation}
    f(x) = x^r - a
\end{equation}
この関数を微分すると
\begin{equation}
    f'(x) = rx^{r-1}
\end{equation}
となる.
よって,更新式は次のようになる.
\begin{align}
    x_{n+1} & = x_n - \frac{f(x_n)}{f'(x_n)}             \notag \\
            & = x_n - \frac{x_n^r - a}{rx_n^{r-1}}       \notag \\
            & = \frac{r-1}{r} x_n + \frac{a}{rx_n^{r-1}}
    \label{eq:root-finding_newton-raphson_rth-root-update-law}
\end{align}

\begin{theorem}
    式 \eqref{eq:root-finding_newton-raphson_rth-root-update-law} の更新式は,
    初期値 $x_0$ が $x_0 > 0$ を満たす場合,
    べき根 $\sqrt[r]{a}$ に収束する.
\end{theorem}
\begin{proof}
    $x_0 = \sqrt[r]{a}$ の場合は
    $k = 1, 2, \ldots$ において $x_k = \sqrt[r]{a}$ が成り立つ.
    つまり,平方根 $\sqrt[r]{a}$ に収束している.
    そこで,以下では $x_0 \neq \sqrt[r]{a}$ とする.

    更新後の値 $x_{n+1}$ とべき根 $\sqrt[r]{a}$ の差は次のようになる.
    \begin{align}
          & x_{n+1} - \sqrt[r]{a}                                    \notag \\
        = & \frac{r-1}{r} \left(x_n - \sqrt[r]{a}\right)
        + \frac{1}{r} \left(\frac{a}{x_n^{r-1}} - \sqrt[r]{a}\right) \notag \\
        = & \frac{r-1}{r} \left(x_n - \sqrt[r]{a}\right)
        -\frac{\sqrt[r]{a}}{r} \left(1 - \left(\frac{\sqrt[r]{a}}{x_n}\right)^{r-1}\right)
    \end{align}
    これに等比級数の公式
    \begin{equation}
        \sum_{k = 1}^n x^{k-1} = \frac{1 - x^n}{1 - x}
    \end{equation}
    を適用すると,
    \begin{align}
          & x_{n+1} - \sqrt[r]{a}                                                \notag     \\
        = & \frac{r-1}{r} \left(x_n - \sqrt[r]{a}\right)
        -\frac{\sqrt[r]{a}}{r} \left(1 - \frac{\sqrt[r]{a}}{x_n}\right)
        \sum_{k = 1}^{r - 1} \left(\frac{\sqrt[r]{a}}{x_n}\right)^{k - 1}        \notag     \\
        = & \frac{r-1}{r} \left(x_n - \sqrt[r]{a}\right)
        -\frac{1}{r} \left(x_n - \sqrt[r]{a}\right)
        \sum_{k = 1}^{r - 1} \left(\frac{\sqrt[r]{a}}{x_n}\right)^k              \notag     \\
        = & \frac{1}{r} \left(x_n - \sqrt[r]{a}\right)
        \left(r - 1 - \sum_{k = 1}^{r - 1} \left(\frac{\sqrt[r]{a}}{x_n}\right)^k\right)
        \label{eq:root-finding_newton-raphson_rth-root-error-update-linear}                 \\
        = & \frac{1}{r} \left(x_n - \sqrt[r]{a}\right)
        \sum_{k = 1}^{r - 1} \left(1 - \left(\frac{\sqrt[r]{a}}{x_n}\right)^k\right) \notag \\
        = & \frac{1}{r} \left(x_n - \sqrt[r]{a}\right)
        \left(1 - \frac{\sqrt[r]{a}}{x_n}\right)
        \sum_{k = 1}^{r - 1} \sum_{l = 1}^{k}
        \left(\frac{\sqrt[r]{a}}{x_n}\right)^{l - 1}                                 \notag \\
        = & \frac{1}{rx_n} \left(x_n - \sqrt[r]{a}\right)^2
        \sum_{k = 1}^{r - 1} \sum_{l = 1}^{k}
        \left(\frac{\sqrt[r]{a}}{x_n}\right)^{l - 1}
        \label{eq:root-finding_newton-raphson_rth-root-error-update-quadratic}
    \end{align}
    となる.

    $x_0 > 0$ かつ $x_0 \neq \sqrt[r]{a}$ の場合,
    式 \eqref{eq:root-finding_newton-raphson_rth-root-error-update-quadratic} より
    $x_1 - \sqrt[r]{a} > 0$ が成り立つ.
    さらに,$k = 2, 3, \ldots$ において $x_k - \sqrt[r]{a} > 0$ であることも帰納的に成り立つ.
    よって,$k = 1, 2, \ldots$ において $x_k - \sqrt[r]{a} > 0$ である.

    さらに,$x_n > \sqrt[r]{a} > 0$ であることから
    $\sqrt[r]{a} / x_n > 0$ が成り立つから,
    式 \eqref{eq:root-finding_newton-raphson_rth-root-error-update-linear} より
    \begin{align}
          & x_{n+1} - \sqrt[r]{a}                          \notag \\
        < & \frac{r - 1}{r} \left(x_n - \sqrt[r]{a}\right)
    \end{align}
    となる.

    以上から,
    \begin{equation}
        0 < x_{n+1} - \sqrt[r]{a} < \frac{r - 1}{r} \left(x_n - \sqrt[r]{a}\right)
        \label{eq:root-finding_newton-raphson_rth-root-error-convergence}
    \end{equation}
    が成り立ち,$x_n$ の列がべき根 $\sqrt[r]{a}$ へ収束する.
\end{proof}

$r$ が奇数の場合,
負の数 $a < 0$ に対してべき根 $\sqrt[r]{a} < 0$ が存在する.
この場合についてもべき根の算出を考える.

\begin{theorem}
    $r = 3, 5, 7, \ldots$ かつ $a < 0$ の場合を考える.
    式 \eqref{eq:root-finding_newton-raphson_rth-root-update-law} の更新式は,
    初期値 $x_0$ が $x_0 < 0$ を満たす場合,
    べき根 $\sqrt[r]{a}$ に収束する.
\end{theorem}
\begin{proof}
    $x_0 = \sqrt[r]{a}$ の場合は
    $k = 1, 2, \ldots$ において $x_k = \sqrt[r]{a}$ が成り立つ.
    つまり,平方根 $\sqrt[r]{a}$ に収束している.
    そこで,以下では $x_0 \neq \sqrt[r]{a}$ とする.

    $x_0 < 0$ かつ $x_0 \neq \sqrt[r]{a}$ の場合,
    式 \eqref{eq:root-finding_newton-raphson_rth-root-error-update-quadratic} より
    $x_1 - \sqrt[r]{a} < 0$ が成り立つ.
    さらに,$k = 2, 3, \ldots$ において $x_k - \sqrt[r]{a} < 0$ であることも帰納的に成り立つ.
    よって,$k = 1, 2, \ldots$ において $x_k - \sqrt[r]{a} < 0$ である.

    さらに,$x_n < \sqrt[r]{a} < 0$ であることから
    $\sqrt[r]{a} / x_n > 0$ が成り立つから,
    式 \eqref{eq:root-finding_newton-raphson_rth-root-error-update-linear} より
    \begin{align}
          & x_{n+1} - \sqrt[r]{a}                          \notag \\
        > & \frac{r - 1}{r} \left(x_n - \sqrt[r]{a}\right)
    \end{align}
    となる.

    以上から,
    \begin{equation}
        \frac{r - 1}{r} \left(x_n - \sqrt[r]{a}\right) < x_{n+1} - \sqrt[r]{a} < 0
    \end{equation}
    が成り立ち,$x_n$ の列がべき根 $\sqrt[r]{a}$ へ収束する.
\end{proof}

\section{一般の次元の方程式の場合}

一般に $n$ 次元($n=1,2,\ldots$)の方程式
$\bm{f}(\bm{x}) = \bm{0}$ ($\bm{f}: \setR^n \to \setR^n$) の場合,
1 次近似
\begin{equation}
    \bm{f}(\bm{x}_{n+1}) \approx
    \bm{f}(\bm{x}_n)
    + \left.\frac{\partial \bm{f}}{\partial \bm{x}}\right|_{\bm{x}=\bm{x}_n}
    (\bm{x}_{n+1} - \bm{x}_n)
\end{equation}
をもとに更新式
\begin{equation}
    \bm{x}_{n+1} = \bm{x}_n
    - \left( \left.\frac{\partial \bm{f}}{\partial \bm{x}}\right|_{\bm{x}=\bm{x}_n} \right)^{-1}
    \bm{f}(\bm{x}_n)
    \label{eq:root-finding_newton-raphson_n-dim-update-law}
\end{equation}
が得られる.

\section{減速 Newton 法}

\index{げんそく Newton ほう@減速 Newton 法}
Newton-Raphson 法の更新式
\eqref{eq:root-finding_newton-raphson_n-dim-update-law}
は 1 次近似により導出しているため,
近似によるずれが大きい場合は更新により方程式の解から離れてしまう可能性がある.
そのような場合に対応するため,
\index{すてっぷはば@ステップ幅}
ステップ幅 $t_n > 0$ を加えた更新式
\begin{equation}
    \bm{x}_{n+1} = \bm{x}_n
    - t_n \left( \left.\frac{\partial \bm{f}}{\partial \bm{x}}\right|_{\bm{x}=\bm{x}_n} \right)^{-1}
    \bm{f}(\bm{x}_n)
    \label{eq:root-finding_deaccelerated-newton-raphson_n-dim-update-law}
\end{equation}
による減速 Newton 法が考えられている
\cite[6.6 節]{Togawa2007}.
ステップ幅 $t_n$ の決定法については,以下のようなものが考えられている.
\begin{itemize}
    \item $\bm{x}_n$ に関係ない $t_n = t$ に固定する
          \cite[6.6 節]{Togawa2007}.
    \item 動的に決定する.
          \begin{itemize}
              \item $\|\bm{f}(\bm{x}_{n+1})\|_2 < \|\bm{f}(\bm{x}_{n})\|_2$
                    となるまでステップ幅を縮小する
                    \cite[6.6 節]{Togawa2007}.
              \item 最適化の直線探索を適用する
                    (\ref{sec:root-finding_newton-raphson_relation-to-optimization} 節参照).
          \end{itemize}
\end{itemize}

これにより,ステップ幅のない更新式よりも数値解を求められる問題の幅が広がる.

\section{最適化との関係}
\label{sec:root-finding_newton-raphson_relation-to-optimization}

Newton-Raphson 法の更新式
\eqref{eq:root-finding_newton-raphson_n-dim-update-law} は,
最適化における勾配法
(\ref{chap:optimization_unconstrained_descent-methods} 章)
の枠組みでとらえることができる.
そのことを利用すると,
Backtracking Line Search
(Algorithm \ref{alg:optimization_unconstrained_descent-methods_BacktrackingLineSearch})
のような直線探索の手法を Newton-Raphson 法にも取り入れることができる.

$n$ 次元($n=1,2,\ldots$)の方程式
$\bm{f}(\bm{x}) = \bm{0}$ ($\bm{f}: \setR^n \to \setR^n$)
に対して最適化の目的関数
\begin{equation}
    g(\bm{x}) = \frac{1}{2} \| \bm{f}(\bm{x}) \|_2^2
\end{equation}
を考える.
このとき,勾配は
\begin{equation}
    \nabla g(\bm{x}) = \bm{f}(\bm{x})^\top \frac{\partial \bm{f}}{\partial \bm{x}}
\end{equation}
となるため,
Newton-Raphson 法の更新式における更新方向
\begin{equation}
    \bm{d} = - \left(\frac{\partial \bm{f}}{\partial \bm{x}}\right)^{-1} \bm{f}(\bm{x})
\end{equation}
は,$\bm{f}(\bm{x}) = 0$ でない限り
\begin{align}
    \nabla g(\bm{x})^\top \bm{d}
     & = - \bm{f}(\bm{x})^\top \frac{\partial \bm{f}}{\partial \bm{x}}
    \left(\frac{\partial \bm{f}}{\partial \bm{x}}\right)^{-1} \bm{f}(\bm{x})
    \notag                                                             \\
     & = - \bm{f}(\bm{x})^\top \bm{f}(\bm{x})
    \notag                                                             \\
     & = - \| \bm{f}(\bm{x}) \|_2^2
    \notag                                                             \\
     & < 0
\end{align}
のように目的関数の降下方向となっている.

以上から,目的関数 $g(\bm{x})$ の最適化を行うものとして
最適化の直線探索手法
(\ref{sec:optimization_unconstrained_descent-methods_line-search} 節)
を適用することができる.
