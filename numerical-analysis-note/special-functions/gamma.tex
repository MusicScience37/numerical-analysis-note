% !TEX root = ../main.tex
%

\section{Gamma 関数}\label{sec:special-function_gamma-function}

Gamma 関数は一部の特殊関数の定義で用いられる \cite{Kaneko1984} ほか,
統計学における確率密度関数の表現にも用いられる \cite{Berendsen2011}.

Gamma 関数は $x>0$ について次の式で表される
\cite[Section 4.5]{Morse1953}.
\begin{equation}
    \Gamma(x) = \int_0^\infty e^{-t} t^{x-1} dt
\end{equation}
これを複素平面全体に拡張して定義される.

Gamma 関数をプロットしたものを
図 \ref{fig:special-function_gamma-function} に示す.

\begin{figure}[tp]
    \centering
    \includegraphics[width=0.99\linewidth]{plots/gamma-function.pdf}
    \caption{Gamma 関数}
    \label{fig:special-function_gamma-function}
\end{figure}

Gamma 関数は
\begin{equation}
    \Gamma(x+1) = x \Gamma(x)
\end{equation}
のような漸化式を満たし,
特に $n = 1, 2, \ldots$ については以下が成り立つ
\cite[3 章]{Kaneko1984}.
\begin{align}
    \Gamma(n + 1)                      & = n!                                \\
    \Gamma\left(n + \frac{1}{2}\right) & = \frac{(2n - 1)!! \sqrt{\pi}}{2^n}
\end{align}
