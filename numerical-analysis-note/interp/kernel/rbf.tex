% !TEX root = ../../main.tex
%

\section{RBF 補間}\label{sec:interp_kernel_rbf}

\index{RBFほかん@RBF 補間}
\index{Radial Basis Function}
カーネル $K(\bm{r}, \bm{r}')$ の値が距離 $\|\bm{r} - \bm{r}'\|$ に依存する場合,
そのもとになる関数は Radial Basis Function (RBF) と呼ばれる.
RBF $\phi : [0, \infty) \to \setR$ によるカーネルは
\begin{equation}
    K(\bm{r}, \bm{r}') = \phi\left(\frac{\|\bm{r} - \bm{r}'\|}{c}\right)
    \label{eq:regularization_kernel_kernel-of-rbf}
\end{equation}
のように書ける.
RBF としては表
\ref{table:interp_kernel_example-rbfs}
のようなものが挙げられる
\cite{Brochu2010,Fornberg2015}.

\begin{table}[bp]
    \caption{RBF の例 \cite{Brochu2010,Fornberg2015}}
    \label{table:interp_kernel_example-rbfs}
    \centering
    \begin{tabular}{ll}
        名称                        & 関数                                           \\
        \hline
        Gaussian                  & $e^{-r^2}$                                   \\
        Multi-quadric             & $\sqrt{1 + r^2}$                             \\
        Inverse Multi-quadric     & $1/\sqrt{1 + r^2}$                           \\
        Inverse Quadric           & $1 / (1 + r^2)$                              \\
        Sech                      & $1 / \cosh{r}$                               \\
        Bessel ($d=1,2,\ldots$)   & $J_{d/2-1}(r) / r^{d/2-1}$                   \\
        Polyharmonic spline (PHS) & $r^{2k-1}$, $r^{2k}\log{r}$ ($k=1,2,\ldots$) \\
        Compactly Supported RBF   & (\ref{sec:interp_kernel_csrbf} 節参照)
    \end{tabular}
\end{table}

Polyharmonic spline は
\ref{sec:interp_kernel_thin-plate-spline} 節で後述する
thin plate spline におけるカーネルの係数を省略したような形になっている.
補間の計算をする上では係数に意味がないが,
thin plate spline では理論上 $r$ の何乗を選択するべきかが決定されているなどの違いがある.

\subsection{Wendland の Compactly Supported RBF}\label{sec:interp_kernel_csrbf}

\index{Compactly Supported Radial Basis Function}
\index{CSRBF}
RBF 補間を行うにあたって
式 \eqref{eq:interp_kernel_tikhonov_coeff_c} にあるような係数の計算が必要となるが,
表 \ref{table:interp_kernel_example-rbfs}
で挙げたような RBF を使用した場合,
係数行列 $K_m$ は $m \times m$ の密行列となるため,
補間に用いるサンプル点の数 $m$ の数が増加すると
計算に必要な時間が $m^3$ オーダーで増加し,
メモリは $m^2$ オーダーで増加する.
そこで,係数行列 $K_m$ が疎行列となるような RBF が考案された.
そのような RBF のうち,
ここでは Wendland の compactly supported RBF \cite{Wendland1995} について説明する.

Wendland の compactly supported RBF においては,
\begin{equation}
    x_+^l = \begin{cases}
        x^l & \text{for x > 0} \\
        0   & \text{otherwise}
    \end{cases}
\end{equation}
のような truncated power function を用いる.
また,サンプル点の座標は $d$ 次元の Euclid 空間にあるものとする.

まず,truncated power function を用いて定義される連続な RBF
\begin{equation}
    \phi_l(r) \equiv (1 - r)_+^l
\end{equation}
のパラメータ $l$ を $l \ge \lfloor d/2 \rfloor + 1$ となるように設定すると,
カーネルの行列 $K_m$ が正定値になる.
さらに,微分も可能な RBF を作るため,積分の作用素
\begin{equation}
    I(f)(r) \equiv \int_r^\infty sf(s) ds
\end{equation}
を定義し,その作用素を $\phi_l$ に適用した RBF
\begin{equation}
    \phi_{l,k} \equiv I^k \phi_l \label{eq:rbf_wendland-csrbf_definition}
\end{equation}
を考える.
このとき,$l = \lfloor d/2 \rfloor + k + 1$ とすると,
カーネルの行列 $K_m$ が正定値になる特性を持ったまま
$C^{2k}$ 級の関数になる \cite[Theorem 3.5]{Wendland1995}.
積分の作用素を適用した結果は部分積分を用いて計算でき,
$k = 0, 1, 2$ における RBF は以下のような式で書ける
(導出過程は後述する).
\begin{align}
    \phi_{l,0} & = (1 - r)_+^l
    \label{eq:rbf_wendland-csrbf_formula0}                                        \\
    \phi_{l,1} & = \frac{1}{(l+1)(l+2)} (1 - r)_+^{l+1} \left( (l+1)r + 1 \right)
    \label{eq:rbf_wendland-csrbf_formula1}                                        \\
    \phi_{l,2} & = \frac{1}{(l+1)(l+2)(l+3)(l+4)} (1 - r)_+^{l+2}
    \left( (l+1)(l+3)r^2 + 3(l+2)r + 3 \right)
    \label{eq:rbf_wendland-csrbf_formula2}
\end{align}
これらをプロットすると
図 \ref{fig:rbf_wendland-csrbf}
のようになる.

\begin{figure}[tp]
    \centering
    \includegraphics[width=0.99\linewidth]{plots/rbf-wendland-csrbf.pdf}
    \caption{Wendland の Compactly Supported RBF $\phi_{l,k}(r)$ \cite{Wendland1995} の例%
        (見やすいように $r = 0$ のときの値に対する比をプロットしている.)}
    \label{fig:rbf_wendland-csrbf}
\end{figure}

なお,微分の計算をするにあたっては,積分の作用素 $I$ の逆となる微分の作用素が
\begin{equation}
    D(f)(r) \equiv -\frac{1}{r} \frac{d}{dr} f(r)
\end{equation}
で定義されるということを利用すればよい.

\subsubsection{Wendland の CSRBF の計算式の導出}

ここで,
式 \eqref{eq:rbf_wendland-csrbf_formula1}, \eqref{eq:rbf_wendland-csrbf_formula2}
の導出過程について説明しておく.

まず,式 \eqref{eq:rbf_wendland-csrbf_formula1} を導出する.
式 \eqref{eq:rbf_wendland-csrbf_definition} の定義より,
\begin{align}
    \phi_{l,1} (r)
     & = I (\phi_l) (r)
    \notag                              \\
     & = \int_r^\infty s \phi_l(s) ds
    \notag                              \\
     & = \int_r^\infty s (1 - s)_+^l ds
\end{align}
と書ける.
ここで,truncated power function の定義より,
被積分関数は $s \ge 1$ において 0 となるため,
$r \ge 1$ において $\phi_{l,1} = 0$ となる.
そこで,$0 \le r < 1$ の場合について計算を続行する.
部分積分を用いることで,
\begin{align}
    \phi_{l,1} (r)
     & = \int_r^1 s (1 - s)^l ds
    \notag                                                              \\
     & = \left[ s \left( -\frac{1}{l+1} \right) (1-s)^{l+1} \right]_r^1
    - \int_r^1 \left( -\frac{1}{l+1} \right) (1-s)^{l+1} ds
    \notag                                                              \\
     & = \frac{1}{l+1} r (1-r)^{l+1}
    - \left[ \frac{1}{(l+1)(l+2)} (1-s)^{l+2} \right]_r^1
    \notag                                                              \\
     & = \frac{1}{l+1} r (1-r)^{l+1}
    + \frac{1}{(l+1)(l+2)} (1-r)^{l+2}
\end{align}
となる.式を整理すると,
\begin{align}
    \phi_{l,1} (r)
     & = \frac{1}{(l+1)(l+2)} (1-r)^{l+1}
    \left( (l+2)r + (1-r) \right)
    \notag                                                            \\
     & = \frac{1}{(l+1)(l+2)} (1 - r)^{l+1} \left( (l+1)r + 1 \right)
\end{align}
となる.
最後に $r \ge 1$ で $\phi_{l,1} = 0$ となるように truncated power function を用いて書き直すことで,
\begin{align}
    \phi_{l,1} (r)
     & = \frac{1}{(l+1)(l+2)} (1 - r)_+^{l+1} \left( (l+1)r + 1 \right)
\end{align}
が得られる.

続いて,式 \eqref{eq:rbf_wendland-csrbf_formula2} を導出する.
$\phi_{l,1}$ と同様に定義より
\begin{align}
    \phi_{l,2}
     & = I (\phi_{l,1}) (r)
    \notag                                                                                 \\
     & = \int_r^\infty s \phi_{l,1}(s) ds
    \notag                                                                                 \\
     & = \int_r^\infty s \frac{1}{(l+1)(l+2)} (1 - s)_+^{l+1} \left( (l+1)s + 1 \right) ds
    \notag                                                                                 \\
     & = \frac{1}{(l+1)(l+2)} \int_r^\infty s (1 - s)_+^{l+1} \left( (l+1)s + 1 \right) ds
\end{align}
となる.
$r \ge 1$ においては $\phi_{l,2} = 0$ となるため,
$0 \le r < 1$ とすると,
\begin{align}
    \phi_{l,2}
     & = \frac{1}{(l+1)(l+2)} \int_r^1 s (1 - s)^{l+1} \left( (l+1)s + 1 \right) ds
\end{align}
となる.部分積分を 2 回適用し,式を整理すると,
\begin{align}
    \phi_{l,2}
     & = \frac{1}{(l+1)(l+2)} \int_r^1 \left( (l+1)s^2 + s \right) (1 - s)^{l+1} ds
    \notag                                                                                                             \\
     & = \frac{1}{(l+1)(l+2)} \left[ \left( (l+1)s^2 + s \right) \left( -\frac{1}{l+2} \right) (1-s)^{l+2} \right]_r^1
    \notag                                                                                                             \\
     & \hspace{5em}
    - \frac{1}{(l+1)(l+2)} \int_r^1 \left( 2(l+1)s + 1 \right) \left( -\frac{1}{l+2} \right) (1-s)^{l+2} ds
    \notag                                                                                                             \\
     & = \frac{1}{(l+1)(l+2)^2} \left( (l+1)r^2 + r \right) (1-r)^{l+2}
    \notag                                                                                                             \\
     & \hspace{5em}
    - \frac{1}{(l+1)(l+2)} \left[ \left( 2(l+1)s + 1 \right) \frac{1}{(l+2)(l+3)} (1-s)^{l+3} \right]_r^1
    \notag                                                                                                             \\
     & \hspace{5em}
    + \frac{1}{(l+1)(l+2)} \int_r^1 2(l+1) \frac{1}{(l+2)(l+3)} (1-s)^{l+3} ds
    \notag                                                                                                             \\
     & = \frac{1}{(l+1)(l+2)^2} \left( (l+1)r^2 + r \right) (1-r)^{l+2}
    \notag                                                                                                             \\
     & \hspace{5em}
    + \frac{1}{(l+1)(l+2)^2(l+3)} \left( 2(l+1)r + 1 \right) (1-r)^{l+3}
    \notag                                                                                                             \\
     & \hspace{5em}
    + \frac{1}{(l+1)(l+2)} \left[ -\frac{2(l+1)}{(l+2)(l+3)(l+4)} (1-s)^{l+4} \right]_r^1
    \notag                                                                                                             \\
     & = \frac{1}{(l+1)(l+2)^2} \left( (l+1)r^2 + r \right) (1-r)^{l+2}
    \notag                                                                                                             \\
     & \hspace{5em}
    + \frac{1}{(l+1)(l+2)^2(l+3)} \left( 2(l+1)r + 1 \right) (1-r)^{l+3}
    \notag                                                                                                             \\
     & \hspace{5em}
    + \frac{2}{(l+2)^2(l+3)(l+4)} (1-r)^{l+4}
    \notag                                                                                                             \\
     & = \frac{1}{(l+1)(l+2)^2(l+3)(l+4)} (1-r)^{l+2}
    \notag                                                                                                             \\
     & \hspace{5em}
    \left( (l+3)(l+4) \left( (l+1)r^2 + r \right) + (l+4) \left( 2(l+1)r + 1 \right) (1-r)
    + 2(l+1) (1-r)^2 \right)
    \notag                                                                                                             \\
     & = \frac{1}{(l+1)(l+2)^2(l+3)(l+4)} (1-r)^{l+2}
    \notag                                                                                                             \\
     & \hspace{5em}
    \big( (l+1)(l+3)(l+4) r^2 + (l+3)(l+4)r
    \notag                                                                                                             \\
     & \hspace{5em}
    - 2(l+1)(l+4)r^2 + (2l+1)(l+4)r + (l+4)
    \notag                                                                                                             \\
     & \hspace{5em}
    + 2(l+1)r^2 - 4(l+1)r +2(l+1) \big)
    \notag                                                                                                             \\
     & = \frac{1}{(l+1)(l+2)^2(l+3)(l+4)} (1-r)^{l+2}
    \left( (l+1)(l+2)(l+3)r^2 + 3(l+2)^2r + 3(l+2) \right)
    \notag                                                                                                             \\
     & = \frac{1}{(l+1)(l+2)(l+3)(l+4)} (1-r)^{l+2}
    \left( (l+1)(l+3)r^2 + 3(l+2)r + 3 \right)
\end{align}
となる.
最後に $r \ge 1$ で $\phi_{l,2} = 0$ となるように truncated power function を用いて書き直すことで,
\begin{align}
    \phi_{l,2} & = \frac{1}{(l+1)(l+2)(l+3)(l+4)} (1 - r)_+^{l+2}
    \left( (l+1)(l+3)r^2 + 3(l+2)r + 3 \right)
\end{align}
が得られる.
