% !TEX root = ../../main.tex
%

\section{固有値分解による数値解法}\label{sec:interp_kernel_evd-computation}

密行列の固有値分解を用いてカーネル法の計算を行うことを考える.

カーネルの値を並べた行列 $K_m \in \setC^{m \times m}$ を次のように固有値分解する.
\begin{equation}
    K_m = V D V^*
\end{equation}
ここで,$V \in \setC^{m \times m}$ はエルミート行列であり,
$D \in \setR^{m \times m}$ は実数による対角行列である.
ここで,$K_m$ は半正定値とする
(このような性質を持つカーネルは正定値カーネルと呼ばれる \cite{Fukumizu2010}).
\ref{sec:interp_kernel_tikhonov} 節で導出したカーネルや
\ref{sec:interp_kernel_rbf} 節で示した Gaussian の RBF によるカーネルは
正定値カーネルである.
このとき,
\begin{equation}
    K_m + \lambda I = V (D + \lambda I) V^*
\end{equation}
が成り立つ.
$\lambda > 0$ であれば確実に $K_m + \lambda I$ が正定値となり,逆行列を持つ.
よって,
式 \eqref{eq:interp_kernel_tikhonov_coeff_c} で定めた係数 $\bm{c}$ は
\begin{align}
    \bm{c}
     & = (K_m + \lambda I)^{-1} \bm{y} \notag \\
     & = V (D + \lambda I)^{-1} V^* \bm{y}
\end{align}
と書ける.
さらに,
$V = (\bm{v}_1, \bm{v}_2, \ldots, \bm{v}_m)$,
$D = \diag(\lambda_1, \lambda_2, \ldots, \lambda_m)$
とおくと,
\begin{align}
    \bm{c}
     & = \sum_{i=1}^m \frac{1}{\lambda_i + \lambda} (\bm{v}_i^* \bm{y}) \bm{v}_i
\end{align}
のように表すこともできる.

\section{パラメータ推定}\label{sec:regularization_kernel_param-est}

式 \eqref{eq:regularization_kernel_kernel-of-rbf} の定数 $c$ のように
カーネルがパラメータを持っている場合がある.
そのようなパラメータの推定を
\cite[Remark 3 (Connection to spatial statistics)]{Scheuerer2011}
に沿った
\index{maximum likelihood estimation}
maximum likelihood estimation (MLE)
により考える.
なお,本節ではカーネルの値やデータの値がすべて実数になっているものとする.

Gaussian Process (\ref{sec:regularization_kernel_gaussian-process} 節)より,
データ $\bm{y}$ の確率密度関数は
\begin{equation}
    p(\bm{y}) = \frac{1}{\sqrt{(2\pi)^{m} \det(\tau K_m + \sigma^2 I)}}
    \exp\left(-\frac{1}{2} \bm{y}^T (\tau K_m + \sigma^2 I)^{-1} \bm{y} \right)
\end{equation}
であり,その対数を取ると
\begin{equation}
    \log{p(\bm{y})}
    = -\frac{m}{2}\log(2\pi)
    - \frac{1}{2} \log{\det(\tau K_m + \sigma^2 I)}
    - \frac{1}{2} \bm{y}^T (\tau K_m + \sigma^2 I)^{-1} \bm{y}
\end{equation}
となる.
これを最大化するようにパラメータを決定する.

$\lambda = \sigma^2 / \tau$ を固定すると,
\begin{equation}
    \log{p(\bm{y})}
    = -\frac{m}{2}\log(2\pi)
    - \frac{m}{2} \log{\tau}
    - \frac{1}{2} \log{\det(K_m + \lambda I)}
    - \frac{1}{2\tau} \bm{y}^T (K_m + \lambda I)^{-1} \bm{y}
\end{equation}
となる.
これを最大化するように $\tau$ を決定すると,
\begin{equation}
    \tau = \frac{1}{m} \bm{y}^T (K_m + \lambda I)^{-1} \bm{y}
    \label{eq:interp_kernel_param_coeff_tau}
\end{equation}
となり,そのときの $\log{p(\bm{y})}$ は
\begin{equation}
    \log{p(\bm{y})}
    = -\frac{m}{2}\log(2\pi)
    + \frac{m}{2} \log{m}
    - \frac{m}{2} \log(\bm{y}^T (K_m + \lambda I)^{-1} \bm{y})
    - \frac{1}{2} \log{\det(K_m + \lambda I)}
    - \frac{m}{2}
\end{equation}
となる.
$m$ はサンプルの数であり,定数だから,
パラメータ推定では
\begin{equation}
    m \log(\bm{y}^T (K_m + \lambda I)^{-1} \bm{y})
    + \log{\det(K_m + \lambda I)}
\end{equation}
を最小化すれば良い.

なお,
\ref{sec:interp_kernel_evd-computation} 節における固有値分解を用いることで,
パラメータ推定における
式 \eqref{eq:interp_kernel_param_coeff_tau} の $\tau$ は
\begin{align}
    \tau
     & = \frac{1}{m} \bm{y}^T (K_m + \lambda I)^{-1} \bm{y} \notag                    \\
     & = \frac{1}{m} \sum_{i=1}^m \frac{1}{\lambda_i + \lambda} (\bm{v}_i^T \bm{y})^2
\end{align}
のように書ける.
さらに,評価関数は
\begin{align}
     & \hphantom{=}
    m \log(\bm{y}^T (K_m + \lambda I)^{-1} \bm{y})
    + \log{\det(K_m + \lambda I)}
    \notag                                                                                   \\
     & = m \log\left(\sum_{i=1}^m \frac{1}{\lambda_i + \lambda} (\bm{v}_i^T \bm{y})^2\right)
    + \log\left(\prod_{i=1}^m (\lambda_i + \lambda)\right)
    \notag                                                                                   \\
     & = m \log\left(\sum_{i=1}^m \frac{1}{\lambda_i + \lambda} (\bm{v}_i^T \bm{y})^2\right)
    + \sum_{i=1}^m \log(\lambda_i + \lambda)
\end{align}
のように書ける.
