% !TEX root = ../../main.tex
%

\section{多項式などによる項の追加}\label{sec:interp_kernel_additional-terms}

本章の冒頭で概説したように,カーネル法で
\begin{equation}
    f(\bm{r}) = \sum_{i=1}^m c_i K(\bm{r}, \bm{r}_i) + \sum_{i=1}^M d_i p_i(\bm{r})
\end{equation}
のように追加の関数 $p_i : X \to \setR$ による項を追加することが考えられる.

例えば,RBF 補間においては追加の項が定数や多項式となるように,
$M = 1$ で $p_1(\bm{r}) = 1$ としたり,
$\bm{r} = (x, y)^\top$ のときに $M = 3$ で
$p_1(\bm{r}) = 1$, $p_2(\bm{r}) = x$, $p_3(\bm{r}) = y$ としたりする.
これにより,補間の精度が高くなるという
\cite[Section 3.1.3.5]{Fornberg2015}.
また,thin plate spline \cite{Ghosh2010}, spherical spline \cite{Wahba1981} のように,
追加の項が指定されているカーネルも存在する.

このような補間においては,係数 $c_i$, $d_i$ を以下の方程式により決定する.
\begin{equation}
    \begin{pmatrix}
        K_m + \lambda I & P \\
        P^\top          & O
    \end{pmatrix}
    \begin{pmatrix}
        \bm{c} \\ \bm{d}
    \end{pmatrix}
    =
    \begin{pmatrix}
        \bm{y} \\ \bm{0}
    \end{pmatrix}
\end{equation}
ここで,行列 $P \in \setR^{m \times M}$ の成分は $P_{ij} = p_j(\bm{r}_i)$ とする.
