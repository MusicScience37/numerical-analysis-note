% !TEX root = ../../main.tex
%

\chapter{カーネル法}

\index{かーねるほう@カーネル法}
本章では,カーネルを用いて補間を行うカーネル法についてまとめる.

関数 $f : X \to \setC$ を
サンプル点 $(\bm{r}_i, y_i) \in X \times \setC$ ($i = 1, 2, \ldots, m$) から補間することを考える.
カーネルを用いた補間では,カーネル $K : X \times X \to \setC$ を用いて
\begin{equation}
    f(\bm{r}) = \sum_{i=1}^m c_i K(\bm{r}, \bm{r}_i)
\end{equation}
のようにおき,$y_i = f(\bm{r}_i)$ となるように係数 $c_i$ を決める
\cite{Fukumizu2010}.

また,関数 $p_i : X \to \setR$ による項を追加した
\begin{equation}
    f(\bm{r}) = \sum_{i=1}^m c_i K(\bm{r}, \bm{r}_i) + \sum_{i=1}^M d_i p_i(\bm{r})
\end{equation}
を考えて係数 $c_i$, $d_i$ を求める方法も存在する
(\ref{sec:interp_kernel_additional-terms} 節にて説明する).

% !TEX root = ../../main.tex
%

\section{Tikhonov 正則化による導出}\label{sec:interp_kernel_tikhonov}

まず,カーネルによる補間の導出を行う.

関数 $f$ はある正規直交基底の存在する関数空間 $\mathcal{H}$ に属するものとし,
その関数空間 $\mathcal{H}$ には正規直交基底となる
関数 $\alpha_1(\bm{r}), \alpha_2(\bm{r}), \ldots, \alpha_N(\bm{r})$ が存在するものとする
\footnote{この導出では $N$ が有限であるとする.}.
それらの基底を用いて関数 $f$ を次のようにおく.
\begin{equation}
    f(\bm{r}) = \sum_{i = 1}^{N} x_i \alpha_i(\bm{r})
\end{equation}
ここで,
$A_{ij} = \alpha_j(\bm{r}_i)$ となる行列 $A \in \setC^{m \times N}$ を考えると,
最小二乗法の評価関数は
\begin{equation}
    \|A \bm{x} - \bm{y}\|_2
\end{equation}
となる.

これに Tikhonov 正則化(\ref{chap:regularization_tikhonov} 章)を適用する.
関数空間 $\mathcal{H}$ のノルムを $\|\cdot\|_{\mathcal{H}}$ とすると,
$\alpha_i$ が $\mathcal{H}$ の正規直交基底であることから,
\begin{align}
    \|f\|_{\mathcal{H}}^2
     & = \sum_{i = 1}^{N} |x_i|^2 \notag \\
     & = \|\bm{x}\|_2^2
\end{align}
となる.
よって,Tikhonov 正則化の評価関数は
式 \eqref{eq:regularization_tikhonov_objective} のように書ける.

式 \eqref{eq:regularization_tikhonov_exchange-mat} より最適解は
\begin{align}
    \bm{x} & = A^* (AA^* + \lambda I)^{-1} \bm{y}
\end{align}
となる.
ここで,行列 $AA^*$ の $(i, j)$ 要素は
\begin{equation}
    [AA^*]_{ij} = \sum_{k = 1}^{N} \alpha_k(\bm{r}_i) \overline{\alpha_k(\bm{r}_j)}
\end{equation}
となる.それをもとに次のように関数 $K : X \times X \to \setC$ を定義する.
\begin{equation}
    K(\bm{r}, \bm{r}') = \sum_{k = 1}^{N} \alpha_k(\bm{r}) \overline{\alpha_k(\bm{r}')}
\end{equation}
この関数がカーネルである.
また,行列 $K_m \in \setC^{m \times m}$ を
$K_m = AA^*$ のようにおく.
なお,この行列は半正定値エルミート行列である.

ここで,変数 $\bm{c} \in \setC^m$ を
\begin{equation}
    \bm{c} = (K_m + \lambda I)^{-1} \bm{y}
    \label{eq:interp_kernel_tikhonov_coeff_c}
\end{equation}
のようにおく.
このとき,$\bm{x} = A^* \bm{c}$ だから,
\begin{align}
    f(\bm{r})
     & = \sum_{i = 1}^{N} x_i \alpha_i(\bm{r}) \notag                                                \\
     & = \sum_{i = 1}^{N} \sum_{j = 1}^{m} \overline{A_{ji}} c_j \alpha_i(\bm{r}) \notag             \\
     & = \sum_{i = 1}^{N} \sum_{j = 1}^{m} \overline{\alpha_i(\bm{r}_j)} c_j \alpha_i(\bm{r}) \notag \\
     & = \sum_{j = 1}^{m} c_j K(\bm{r}, \bm{r}_j)
\end{align}
となる.

関数 $f$ を正規直交基底で表していたことから,
カーネル法は関数空間 $\mathcal{H}$ 全体の中で Tikhonov 正則化の評価関数
\begin{equation}
    \sum_{i=1}^m \left|y_i - f(\bm{r}_i)\right|^2
    + \|f\|_{\mathcal{H}}^2
\end{equation}
を最小化したものを求められるということが分かる.
また,ここでは導出において関数空間 $\mathcal{H}$ の次元を表す $N$ が有限であることを前提としていたが,
導出されたカーネル法で用いる行列は $m \times m$ の正方行列であるため,
$N \to \infty$ の場合にも適用できる.

\section{再生核ヒルベルト空間}\label{sec:interp_kernel_rkhs}

\index{さいせいかくひるべるとくうかん@再生核ヒルベルト空間}
\index{Reproducing Kernel Hilbert Space}
\index{RKHS|see(Reproducing Kernel Hilbert Space)}
本節では,
再生核ヒルベルト空間 (Reproducing Kernel Hilbert Space, RKHS)
について説明するとともに,
再生核ヒルベルト空間とカーネル法との関連について説明する.

再生核ヒルベルト空間は次のように定義される.

\begin{definition}[{\cite{Aronszajn1950}}]
    空間 $X$ 上で定義される関数のヒルベルト空間 $\mathcal{H}$ があり,
    関数 $f, g \in \mathcal{H}$ の内積は $(f, g)$ で表されるものとする.
    このとき,関数 $K : X \times X \to \setC$ が以下を満たすのであれば,
    関数 $K$ を再生核と呼び,$\mathcal{H}$ は再生核ヒルベルト空間と呼ぶ.
    \begin{itemize}
        \item 任意の $y \in X$ について $y$ を固定した $x$ についての関数 $K(x,y)$ は
              $\mathcal{H}$ に属する.
        \item 任意の $y \in X$ と $f \in \mathcal{X}$ について,
              \begin{equation}
                  f(y) = (f(x), K(x, y))_x
              \end{equation}
              が成り立つ.(添え字 $x$ は $x$ の関数として内積をとることを示す.)
    \end{itemize}
\end{definition}

上記の定義を満たすような再生核 $K$ をカーネル法におけるカーネルとして使用する場合,
点群 $\bm{r}_1, \bm{r}_2, \ldots, \bm{r}_m$ とカーネルによる行列
\begin{equation}
    K_m =
    \begin{pmatrix}
        K(\bm{r}_1, \bm{r}_1) & K(\bm{r}_1, \bm{r}_2) & \cdots & K(\bm{r}_1, \bm{r}_m) \\
        K(\bm{r}_2, \bm{r}_1) & K(\bm{r}_2, \bm{r}_2) & \cdots & K(\bm{r}_2, \bm{r}_m) \\
        \vdots                & \vdots                & \ddots & \vdots                \\
        K(\bm{r}_m, \bm{r}_1) & K(\bm{r}_m, \bm{r}_2) & \cdots & K(\bm{r}_m, \bm{r}_m)
    \end{pmatrix}
\end{equation}
は少なくとも半正定値になる
\cite{Aronszajn1950}.
点群を適切に選ぶことで行列 $K_m$ が正則になるようにした場合,
カーネル法による補間は以下の性質を満たすことが示せる
\footnote{文献 \cite{Kimeldorf1971,Wahba1981} の理論を利用している.}.

\begin{theorem}
    空間 $X$ 上で定義される関数の再生核ヒルベルト空間 $\mathcal{H}$ があり,
    関数 $f, g \in \mathcal{H}$ の内積は $(f, g)$ で表されるものとし,
    ノルムを $\|f\|$ とし,
    再生核は $K : X \times X \to \setC$ とする.
    また,$i = 1, 2, \ldots, m$ について
    点 $\bm{r}_i \in X$ と値 $f_i \in C$ を定義する.
    このとき,$f(\bm{r}_i) = f_i$ を満たす $f \in \mathcal{H}$ で
    ノルム $\|f\|$ が最小となるものは以下で与えられる.
    \begin{equation}
        f(\bm{r}) = \sum_{j = 1}^{m} c_j K(\bm{r}, \bm{r}_j)
    \end{equation}
    ここで,$c_j$ は以下を満たす係数である.
    \begin{equation}
        \begin{pmatrix}
            K(\bm{r}_1, \bm{r}_1) & K(\bm{r}_1, \bm{r}_2) & \cdots & K(\bm{r}_1, \bm{r}_m) \\
            K(\bm{r}_2, \bm{r}_1) & K(\bm{r}_2, \bm{r}_2) & \cdots & K(\bm{r}_2, \bm{r}_m) \\
            \vdots                & \vdots                & \ddots & \vdots                \\
            K(\bm{r}_m, \bm{r}_1) & K(\bm{r}_m, \bm{r}_2) & \cdots & K(\bm{r}_m, \bm{r}_m)
        \end{pmatrix}
        \begin{pmatrix}
            c_1 \\ c_2 \\ \vdots \\ c_m
        \end{pmatrix}
        =
        \begin{pmatrix}
            f_1 \\ f_2 \\ \vdots \\ f_m
        \end{pmatrix}
        \label{eq:interp_kernel_rkhs_exact-interp-coeff-equation}
    \end{equation}
\end{theorem}
\begin{proof}
    式 \eqref{eq:interp_kernel_rkhs_exact-interp-coeff-equation} を満たす係数 $c_i$ について,
    \begin{equation}
        f(\bm{r}) = \sum_{j = 1}^{m} c_j K(\bm{r}, \bm{r}_j) + g(\bm{r})
    \end{equation}
    となる関数 $g \in \mathcal{H}$ を考える.
    両辺について右から $K(\bm{r}, \bm{r}_i)$ との内積をとると,
    \begin{align}
        (f(\bm{r}), K(\bm{r}, \bm{r}_i))_{\bm{r}} & =
        \sum_{j = 1}^{m} c_j (K(\bm{r}, \bm{r}_j), K(\bm{r}, \bm{r}_i))_{\bm{r}}
        + (g(\bm{r}), K(\bm{r}, \bm{r}_i))_{\bm{r}}
    \end{align}
    となる.
    再生核の定義を用いると以下のように変形できる.
    \begin{align}
        f(\bm{r}_i)                               & =
        \sum_{j = 1}^{m} c_j K(\bm{r}_i, \bm{r}_j) + (g(\bm{r}), K(\bm{r}, \bm{r}_i)) \notag          \\
        f_i                                       & = f_i + (g(\bm{r}), K(\bm{r}, \bm{r}_i))   \notag \\
        (g(\bm{r}), K(\bm{r}, \bm{r}_i))_{\bm{r}} & = 0
    \end{align}
    この結果を利用すると,関数 $f$ のノルムは
    \begin{align}
        \|f\|^2
         & = (f, f)
        \notag                                                                                          \\
         & = \sum_{i = 1}^{m} \sum_{j = 1}^{m} c_i \bar{c_j} (K(\bm{r}, \bm{r}_i), K(\bm{r}, \bm{r}_j))
        + \sum_{j = 1}^{m} \bar{c_j} (g(\bm{r}), K(\bm{r}, \bm{r}_j))
        + \sum_{j = 1}^{m} c_j (K(\bm{r}, \bm{r}_j), g(\bm{r}))
        + \|g\|^2
        \notag                                                                                          \\
         & = \sum_{i = 1}^{m} \sum_{j = 1}^{m} c_i \bar{c_j} (K(\bm{r}, \bm{r}_i), K(\bm{r}, \bm{r}_j))
        + \|g\|^2
    \end{align}
    となる.
    これが最小となる $g \in \mathcal{H}$ は $\|g\| = 0$ となる $g(\bm{r}) = 0$ である.
    よって,$g(\bm{r}) = 0$ とした
    \begin{equation}
        f(\bm{r}) = \sum_{j = 1}^{m} c_j K(\bm{r}, \bm{r}_j)
    \end{equation}
    は $f(\bm{r}_i) = f_i$ となる $f \in \mathcal{H}$ の中で $\|f\|$ が最も小さいものとなっている.
\end{proof}

この定理の結果を利用し,例えば
ノルムを小さくすることで関数が滑らかになるように
再生核ヒルベルト空間 $\mathcal{H}$ を定義すると,
カーネル法により与えられた点を通る最も滑らかな関数を得ることができる.
実際に,そのような理論で作られたカーネルとして
thin plate spline \cite{Ghosh2010},
spherical spline \cite{Wahba1981}
が挙げられる.

\section{Gaussian Process}\label{sec:regularization_kernel_gaussian-process}

カーネルによる補間は,
\index{Gaussian Process}
Gaussian Process と呼ばれる確率分布から導くこともできる
\cite{Brochu2010}.
ただし,データとカーネル関数の値は実数とする.

平均 $\mu(\bm{r})$,分散 $K(\bm{r}, \bm{r}')$ の Gaussian Process に従う関数 $f$ は,
関数値のベクトル $(f(\bm{r}_1), f(\bm{r}_2), \ldots, f(\bm{r}_m))$ が
正規分布 $\mathcal{N}(\bm{\mu}_m, K_m)$ に従う.
ここで,
$\bm{\mu}_m$ は $[\bm{\mu}_m]_i = \mu(\bm{r}_i)$ なるベクトルであり,
$K_m$ は $[K_m]_{ij} = K(\bm{r}_i, \bm{r}_j)$ なる行列である.
Gaussian Process は関数における正規分布のようなものである.

関数 $f$ が平均 $0$,分散 $\tau K(\bm{r}, \bm{r}')$ の Gaussian Process に従っており,
関数 $f$ に対するノイズ入りのサンプルが $y_i = f(\bm{r}_i) + \epsilon_i$ のように得られているとする.
ただし,$\epsilon_i$ は独立に正規分布 $\mathcal{N}(0, \sigma^2)$ に従うものとする.
また,$\tau > 0$ は分散の大きさを示すパラメータである.
このとき,ベクトル $(y_1, y_2, \ldots, y_m, f(\bm{r}))$ は
次のような正規分布に従う.
\begin{equation}
    \mathcal{N}\left(\bm{0},
    \begin{pmatrix}
        \tau K_m + \sigma^2 I & \tau \bm{k}(\bm{r})    \\
        \tau \bm{k}(\bm{r})^T & \tau K(\bm{r}, \bm{r})
    \end{pmatrix}
    \right)
\end{equation}
このとき,確率密度関数 $p(y_1, y_2, \ldots, y_m, f(\bm{r}))$ は次のように書ける.
\begin{equation}
    p(y_1, y_2, \ldots, y_m, f(\bm{r}))
    = \frac{1}{\sqrt{(2\pi)^{m+1} \det{
                \begin{pmatrix}
                    \tau K_m + \sigma^2 I & \tau \bm{k}(\bm{r})    \\
                    \tau \bm{k}(\bm{r})^T & \tau K(\bm{r}, \bm{r})
                \end{pmatrix}
            }}}
    \exp\left(-\frac{1}{2}
    \begin{pmatrix}
        \bm{y} \\ f(\bm{r})
    \end{pmatrix}^T
    \begin{pmatrix}
        \tau K_m + \sigma^2 I & \tau \bm{k}(\bm{r})    \\
        \tau \bm{k}(\bm{r})^T & \tau K(\bm{r}, \bm{r})
    \end{pmatrix}^{-1}
    \begin{pmatrix}
        \bm{y} \\ f(\bm{r})
    \end{pmatrix}
    \right)
\end{equation}

ここで,
\begin{align}
    \sigma_K^2(\bm{r})
     & \equiv \tau K(\bm{r}, \bm{r})
    - \tau \bm{k}(\bm{r})^T (\tau K_m + \sigma^2 I)^{-1} \tau \bm{k}(\bm{r}) \\
     & = \tau K(\bm{r}, \bm{r})
    - \tau \bm{k}(\bm{r})^T (K_m + \sigma^2 I)^{-1} \bm{k}(\bm{r})
\end{align}
とおくと,
\begin{align}
    \det{
        \begin{pmatrix}
            \tau K_m + \sigma^2 I & \tau \bm{k}(\bm{r})    \\
            \tau \bm{k}(\bm{r})^T & \tau K(\bm{r}, \bm{r})
        \end{pmatrix}
    }
     & = \sigma_K^2(\bm{r}) \det(\tau K_m + \sigma^2 I)
\end{align}
となる.
また,
\begin{align}
     & \hphantom{=}
    \begin{pmatrix}
        \tau K_m + \sigma^2 I & \tau \bm{k}(\bm{r})    \\
        \tau \bm{k}(\bm{r})^T & \tau K(\bm{r}, \bm{r})
    \end{pmatrix}^{-1}
    \notag          \\
     & =
    \begin{pmatrix}
        (\tau K_m + \sigma^2 I)^{-1}
        + (\tau K_m + \sigma^2 I)^{-1} \tau \bm{k}(\bm{r}) \sigma_K^2(\bm{r})^{-1} \tau \bm{k}(\bm{r})^T (\tau K_m + \sigma^2 I)^{-1}
         &
        - (\tau K_m + \sigma^2 I)^{-1} \tau \bm{k}(\bm{r}) \sigma_K^2(\bm{r})^{-1}
        \\
        - \sigma_K^2(\bm{r})^{-1} \tau \bm{k}(\bm{r})^T (\tau K_m + \sigma^2 I)^{-1}
         &
        \sigma_K^2(\bm{r})^{-1}
    \end{pmatrix}
\end{align}
であり,
\begin{align}
     & \hphantom{=}
    \begin{pmatrix}
        \bm{y} \\ f(\bm{r})
    \end{pmatrix}^T
    \begin{pmatrix}
        \tau K_m + \sigma^2 I & \tau \bm{k}(\bm{r})    \\
        \tau \bm{k}(\bm{r})^T & \tau K(\bm{r}, \bm{r})
    \end{pmatrix}^{-1}
    \begin{pmatrix}
        \bm{y} \\ f(\bm{r})
    \end{pmatrix}
    \notag                                            \\
     & = \bm{y}^T (\tau K_m + \sigma^2 I)^{-1} \bm{y}
    + \bm{y}^T (\tau K_m + \sigma^2 I)^{-1} \tau \bm{k}(\bm{r}) \sigma_K^2(\bm{r})^{-1} \tau \bm{k}(\bm{r})^T (\tau K_m + \sigma^2 I)^{-1} \bm{y}
    \notag                                            \\ & \hspace{2em}
    - f(\bm{r}) \sigma_K^2(\bm{r})^{-1} \tau \bm{k}(\bm{r})^T (\tau K_m + \sigma^2 I)^{-1} \bm{y}
    - \bm{y}^T (\tau K_m + \sigma^2 I)^{-1} \tau \bm{k}(\bm{r}) \sigma_K^2(\bm{r})^{-1} f(\bm{r})
    \notag                                            \\ & \hspace{2em}
    + f(\bm{r}) \sigma_K^2(\bm{r})^{-1} f(\bm{r})
    \notag                                            \\
     & = \bm{y}^T (\tau K_m + \sigma^2 I)^{-1} \bm{y}
    + \sigma_K^2(\bm{r})^{-1} \left(f(\bm{r}) - \tau \bm{k}(\bm{r})^T (\tau K_m + \sigma^2 I)^{-1} \bm{y}\right)^2
\end{align}
となる.
よって,確率密度関数は次のように分割できる.
\begin{align}
    p(y_1, y_2, \ldots, y_m, f(\bm{r}))
     & = p_{\bm{y}}(\bm{y}) p_f(f(\bm{r}))
    \\
    p_{\bm{y}}(\bm{y})
     & = \frac{1}{\sqrt{(2\pi)^{m} \det(\tau K_m + \sigma^2 I)}}
    \exp\left(-\frac{1}{2} \bm{y}^T (\tau K_m + \sigma^2 I)^{-1} \bm{y} \right)
    \\
    p_f(f(\bm{r}))
     & = \frac{1}{\sqrt{(2\pi)^{m} \sigma_K^2(\bm{r})}}
    \exp\left(-\frac{1}{2} \sigma_K^2(\bm{r})^{-1} \left(f(\bm{r}) - \mu_K(\bm{r})\right)^2\right)
    \\
    \mu_K(\bm{r})
     & = \tau \bm{k}(\bm{r})^T (\tau K_m + \sigma^2 I)^{-1} \bm{y}
\end{align}
分割したあとの
$p_{\bm{y}}$ は $\bm{y}$ に関する正規分布 $\mathcal{N}(\bm{0}, \tau K_m + \sigma^2 I)$ であり,
$p_f$ は $f(\bm{r})$ に関する正規分布 $\mathcal{N}(\mu_K(\bm{r}), \sigma_K^2(\bm{r}))$ である.
$\lambda = \sigma^2 / \tau$ とおくと,
\begin{align}
    \mu_K(\bm{r})
     & = \tau \bm{k}(\bm{r})^T (\tau K_m + \sigma^2 I)^{-1} \bm{y}
    \notag                                                                               \\
     & = \bm{k}(\bm{r})^T (K_m + \lambda I)^{-1} \bm{y}
    \notag                                                                               \\
     & = \sum_{i = 1}^m K(\bm{r}, \bm{r}_i) \left[(K_m + \lambda I)^{-1} \bm{y}\right]_i
\end{align}
のように前節と同様の公式が得られる.

Gaussian Process を用いた表現では分散 $\sigma_K^2(\bm{r})$ も得られるため,
補間された関数のどの部分に誤差が多い可能性が高いか評価するのにも役立つ.
また,パラメータ推定において確率の最大化を行うといった応用もある
(\ref{sec:regularization_kernel_param-est} 節).

% !TEX root = ../../main.tex
%

\section{RBF 補間}\label{sec:interp_kernel_rbf}

\index{RBFほかん@RBF 補間}
\index{Radial Basis Function}
カーネル $K(\bm{r}, \bm{r}')$ の値が距離 $\|\bm{r} - \bm{r}'\|$ に依存する場合,
そのもとになる関数は Radial Basis Function (RBF) と呼ばれる.
RBF $\phi : [0, \infty) \to \setR$ によるカーネルは
\begin{equation}
    K(\bm{r}, \bm{r}') = \phi\left(\frac{\|\bm{r} - \bm{r}'\|}{c}\right)
    \label{eq:regularization_kernel_kernel-of-rbf}
\end{equation}
のように書ける.
RBF としては表
\ref{table:interp_kernel_example-rbfs}
のようなものが挙げられる
\cite{Brochu2010,Fornberg2015}.

\begin{table}[bp]
    \caption{RBF の例 \cite{Brochu2010,Fornberg2015}}
    \label{table:interp_kernel_example-rbfs}
    \centering
    \begin{tabular}{ll}
        名称                      & 関数                                  \\
        \hline
        Gaussian                & $e^{-r^2}$                          \\
        Multi-quadric           & $\sqrt{1 + r^2}$                    \\
        Inverse Multi-quadric   & $1/\sqrt{1 + r^2}$                  \\
        Inverse Quadric         & $1 / (1 + r^2)$                     \\
        Sech                    & $1 / \cosh{r}$                      \\
        Bessel ($d=1,2,\ldots$) & $J_{d/2-1}(r) / r^{d/2-1}$          \\
        Compactly Supported RBF & (\ref{sec:interp_kernel_csrbf} 節参照)
    \end{tabular}
\end{table}

\subsection{Wendland の Compactly Supported RBF}\label{sec:interp_kernel_csrbf}

\index{Compactly Supported Radial Basis Function}
\index{CSRBF}
RBF 補間を行うにあたって
式 \eqref{eq:interp_kernel_tikhonov_coeff_c} にあるような係数の計算が必要となるが,
表 \ref{table:interp_kernel_example-rbfs}
で挙げたような RBF を使用した場合,
係数行列 $K_m$ は $m \times m$ の密行列となるため,
補間に用いるサンプル点の数 $m$ の数が増加すると
計算に必要な時間が $m^3$ オーダーで増加し,
メモリは $m^2$ オーダーで増加する.
そこで,係数行列 $K_m$ が疎行列となるような RBF が考案された.
そのような RBF のうち,
ここでは Wendland の compactly supported RBF \cite{Wendland1995} について説明する.

Wendland の compactly supported RBF においては,
\begin{equation}
    x_+^l = \begin{cases}
        x^l & \text{for x > 0} \\
        0   & \text{otherwise}
    \end{cases}
\end{equation}
のような truncated power function を用いる.
また,サンプル点の座標は $d$ 次元の Euclid 空間にあるものとする.

まず,truncated power function を用いて定義される連続な RBF
\begin{equation}
    \phi_l(r) \equiv (1 - r)_+^l
\end{equation}
のパラメータ $l$ を $l \ge \lfloor d/2 \rfloor + 1$ となるように設定すると,
カーネルの行列 $K_m$ が正定値になる.
さらに,微分も可能な RBF を作るため,積分の作用素
\begin{equation}
    I(f)(r) \equiv \int_r^\infty sf(s) ds
\end{equation}
を定義し,その作用素を $\phi_l$ に適用した RBF
\begin{equation}
    \phi_{l,k} \equiv I^k \phi_l \label{eq:rbf_wendland-csrbf_definition}
\end{equation}
を考える.
このとき,$l = \lfloor d/2 \rfloor + k + 1$ とすると,
カーネルの行列 $K_m$ が正定値になる特性を持ったまま
$C^{2k}$ 級の関数になる \cite[Theorem 3.5]{Wendland1995}.
積分の作用素を適用した結果は部分積分を用いて計算でき,
$k = 0, 1, 2$ における RBF は以下のような式で書ける
(導出過程は後述する).
\begin{align}
    \phi_{l,0} & = (1 - r)_+^l
    \label{eq:rbf_wendland-csrbf_formula0}                                        \\
    \phi_{l,1} & = \frac{1}{(l+1)(l+2)} (1 - r)_+^{l+1} \left( (l+1)r + 1 \right)
    \label{eq:rbf_wendland-csrbf_formula1}                                        \\
    \phi_{l,2} & = \frac{1}{(l+1)(l+2)(l+3)(l+4)} (1 - r)_+^{l+2}
    \left( (l+1)(l+3)r^2 + 3(l+2)r + 3 \right)
    \label{eq:rbf_wendland-csrbf_formula2}
\end{align}
これらをプロットすると
図 \ref{fig:rbf_wendland-csrbf}
のようになる.

\begin{figure}[tp]
    \centering
    \includegraphics[width=0.99\linewidth]{plots/rbf-wendland-csrbf.pdf}
    \caption{Wendland の Compactly Supported RBF $\phi_{l,k}(r)$ \cite{Wendland1995} の例%
        (見やすいように $r = 0$ のときの値に対する比をプロットしている.)}
    \label{fig:rbf_wendland-csrbf}
\end{figure}

なお,微分の計算をするにあたっては,積分の作用素 $I$ の逆となる微分の作用素が
\begin{equation}
    D(f)(r) \equiv -\frac{1}{r} \frac{d}{dr} f(r)
\end{equation}
で定義されるということを利用すればよい.

\subsubsection{Wendland の CSRBF の計算式の導出}

ここで,
式 \eqref{eq:rbf_wendland-csrbf_formula1}, \eqref{eq:rbf_wendland-csrbf_formula2}
の導出過程について説明しておく.

まず,式 \eqref{eq:rbf_wendland-csrbf_formula1} を導出する.
式 \eqref{eq:rbf_wendland-csrbf_definition} の定義より,
\begin{align}
    \phi_{l,1} (r)
     & = I (\phi_l) (r)
    \notag                              \\
     & = \int_r^\infty s \phi_l(s) ds
    \notag                              \\
     & = \int_r^\infty s (1 - s)_+^l ds
\end{align}
と書ける.
ここで,truncated power function の定義より,
被積分関数は $s \ge 1$ において 0 となるため,
$r \ge 1$ において $\phi_{l,1} = 0$ となる.
そこで,$0 \le r < 1$ の場合について計算を続行する.
部分積分を用いることで,
\begin{align}
    \phi_{l,1} (r)
     & = \int_r^1 s (1 - s)^l ds
    \notag                                                              \\
     & = \left[ s \left( -\frac{1}{l+1} \right) (1-s)^{l+1} \right]_r^1
    - \int_r^1 \left( -\frac{1}{l+1} \right) (1-s)^{l+1} ds
    \notag                                                              \\
     & = \frac{1}{l+1} r (1-r)^{l+1}
    - \left[ \frac{1}{(l+1)(l+2)} (1-s)^{l+2} \right]_r^1
    \notag                                                              \\
     & = \frac{1}{l+1} r (1-r)^{l+1}
    + \frac{1}{(l+1)(l+2)} (1-r)^{l+2}
\end{align}
となる.式を整理すると,
\begin{align}
    \phi_{l,1} (r)
     & = \frac{1}{(l+1)(l+2)} (1-r)^{l+1}
    \left( (l+2)r + (1-r) \right)
    \notag                                                            \\
     & = \frac{1}{(l+1)(l+2)} (1 - r)^{l+1} \left( (l+1)r + 1 \right)
\end{align}
となる.
最後に $r \ge 1$ で $\phi_{l,1} = 0$ となるように truncated power function を用いて書き直すことで,
\begin{align}
    \phi_{l,1} (r)
     & = \frac{1}{(l+1)(l+2)} (1 - r)_+^{l+1} \left( (l+1)r + 1 \right)
\end{align}
が得られる.

続いて,式 \eqref{eq:rbf_wendland-csrbf_formula2} を導出する.
$\phi_{l,1}$ と同様に定義より
\begin{align}
    \phi_{l,2}
     & = I (\phi_{l,1}) (r)
    \notag                                                                                 \\
     & = \int_r^\infty s \phi_{l,1}(s) ds
    \notag                                                                                 \\
     & = \int_r^\infty s \frac{1}{(l+1)(l+2)} (1 - s)_+^{l+1} \left( (l+1)s + 1 \right) ds
    \notag                                                                                 \\
     & = \frac{1}{(l+1)(l+2)} \int_r^\infty s (1 - s)_+^{l+1} \left( (l+1)s + 1 \right) ds
\end{align}
となる.
$r \ge 1$ においては $\phi_{l,2} = 0$ となるため,
$0 \le r < 1$ とすると,
\begin{align}
    \phi_{l,2}
     & = \frac{1}{(l+1)(l+2)} \int_r^1 s (1 - s)^{l+1} \left( (l+1)s + 1 \right) ds
\end{align}
となる.部分積分を 2 回適用し,式を整理すると,
\begin{align}
    \phi_{l,2}
     & = \frac{1}{(l+1)(l+2)} \int_r^1 \left( (l+1)s^2 + s \right) (1 - s)^{l+1} ds
    \notag                                                                                                             \\
     & = \frac{1}{(l+1)(l+2)} \left[ \left( (l+1)s^2 + s \right) \left( -\frac{1}{l+2} \right) (1-s)^{l+2} \right]_r^1
    \notag                                                                                                             \\
     & \hspace{5em}
    - \frac{1}{(l+1)(l+2)} \int_r^1 \left( 2(l+1)s + 1 \right) \left( -\frac{1}{l+2} \right) (1-s)^{l+2} ds
    \notag                                                                                                             \\
     & = \frac{1}{(l+1)(l+2)^2} \left( (l+1)r^2 + r \right) (1-r)^{l+2}
    \notag                                                                                                             \\
     & \hspace{5em}
    - \frac{1}{(l+1)(l+2)} \left[ \left( 2(l+1)s + 1 \right) \frac{1}{(l+2)(l+3)} (1-s)^{l+3} \right]_r^1
    \notag                                                                                                             \\
     & \hspace{5em}
    + \frac{1}{(l+1)(l+2)} \int_r^1 2(l+1) \frac{1}{(l+2)(l+3)} (1-s)^{l+3} ds
    \notag                                                                                                             \\
     & = \frac{1}{(l+1)(l+2)^2} \left( (l+1)r^2 + r \right) (1-r)^{l+2}
    \notag                                                                                                             \\
     & \hspace{5em}
    + \frac{1}{(l+1)(l+2)^2(l+3)} \left( 2(l+1)r + 1 \right) (1-r)^{l+3}
    \notag                                                                                                             \\
     & \hspace{5em}
    + \frac{1}{(l+1)(l+2)} \left[ -\frac{2(l+1)}{(l+2)(l+3)(l+4)} (1-s)^{l+4} \right]_r^1
    \notag                                                                                                             \\
     & = \frac{1}{(l+1)(l+2)^2} \left( (l+1)r^2 + r \right) (1-r)^{l+2}
    \notag                                                                                                             \\
     & \hspace{5em}
    + \frac{1}{(l+1)(l+2)^2(l+3)} \left( 2(l+1)r + 1 \right) (1-r)^{l+3}
    \notag                                                                                                             \\
     & \hspace{5em}
    + \frac{2}{(l+2)^2(l+3)(l+4)} (1-r)^{l+4}
    \notag                                                                                                             \\
     & = \frac{1}{(l+1)(l+2)^2(l+3)(l+4)} (1-r)^{l+2}
    \notag                                                                                                             \\
     & \hspace{5em}
    \left( (l+3)(l+4) \left( (l+1)r^2 + r \right) + (l+4) \left( 2(l+1)r + 1 \right) (1-r)
    + 2(l+1) (1-r)^2 \right)
    \notag                                                                                                             \\
     & = \frac{1}{(l+1)(l+2)^2(l+3)(l+4)} (1-r)^{l+2}
    \notag                                                                                                             \\
     & \hspace{5em}
    \big( (l+1)(l+3)(l+4) r^2 + (l+3)(l+4)r
    \notag                                                                                                             \\
     & \hspace{5em}
    - 2(l+1)(l+4)r^2 + (2l+1)(l+4)r + (l+4)
    \notag                                                                                                             \\
     & \hspace{5em}
    + 2(l+1)r^2 - 4(l+1)r +2(l+1) \big)
    \notag                                                                                                             \\
     & = \frac{1}{(l+1)(l+2)^2(l+3)(l+4)} (1-r)^{l+2}
    \left( (l+1)(l+2)(l+3)r^2 + 3(l+2)^2r + 3(l+2) \right)
    \notag                                                                                                             \\
     & = \frac{1}{(l+1)(l+2)(l+3)(l+4)} (1-r)^{l+2}
    \left( (l+1)(l+3)r^2 + 3(l+2)r + 3 \right)
\end{align}
となる.
最後に $r \ge 1$ で $\phi_{l,2} = 0$ となるように truncated power function を用いて書き直すことで,
\begin{align}
    \phi_{l,2} & = \frac{1}{(l+1)(l+2)(l+3)(l+4)} (1 - r)_+^{l+2}
    \left( (l+1)(l+3)r^2 + 3(l+2)r + 3 \right)
\end{align}
が得られる.

% !TEX root = ../../main.tex
%

\section{多項式などによる項の追加}\label{sec:interp_kernel_additional-terms}

本章の冒頭で概説したように,カーネル法で
\begin{equation}
    f(\bm{r}) = \sum_{i=1}^m c_i K(\bm{r}, \bm{r}_i) + \sum_{i=1}^M d_i p_i(\bm{r})
\end{equation}
のように追加の関数 $p_i : X \to \setR$ による項を追加することが考えられる.

例えば,RBF 補間においては追加の項が定数や多項式となるように,
$M = 1$ で $p_1(\bm{r}) = 1$ としたり,
$\bm{r} = (x, y)^\top$ のときに $M = 3$ で
$p_1(\bm{r}) = 1$, $p_2(\bm{r}) = x$, $p_3(\bm{r}) = y$ としたりする.
これにより,補間の精度が高くなるという
\cite[Section 3.1.3.5]{Fornberg2015}.
また,thin plate spline \cite{Ghosh2010}, spherical spline \cite{Wahba1981} のように,
追加の項が指定されているカーネルも存在する.

このような補間においては,係数 $c_i$, $d_i$ を以下の方程式により決定する.
\begin{equation}
    \begin{pmatrix}
        K_m + \lambda I & P \\
        P^\top          & O
    \end{pmatrix}
    \begin{pmatrix}
        \bm{c} \\ \bm{d}
    \end{pmatrix}
    =
    \begin{pmatrix}
        \bm{y} \\ \bm{0}
    \end{pmatrix}
\end{equation}
ここで,行列 $P \in \setR^{m \times M}$ の成分は $P_{ij} = p_j(\bm{r}_i)$ とする.

% !TEX root = ../../main.tex
%

\section{パラメータ推定}\label{sec:regularization_kernel_param-est}

式 \eqref{eq:regularization_kernel_kernel-of-rbf} の定数 $c$ のように
カーネルがパラメータを持っている場合がある.
そのようなパラメータの推定を
\cite[Remark 3 (Connection to spatial statistics)]{Scheuerer2011}
に沿った
\index{maximum likelihood estimation}
maximum likelihood estimation (MLE)
により考える.

Gaussian Process (\ref{sec:regularization_kernel_gaussian-process} 節)より,
データ $\bm{y}$ の確率密度関数は
\begin{equation}
    p(\bm{y}) = \frac{1}{\sqrt{(2\pi)^{m} \det(\tau K_m + \sigma^2 I)}}
    \exp\left(-\frac{1}{2} \bm{y}^T (\tau K_m + \sigma^2 I)^{-1} \bm{y} \right)
\end{equation}
であり,その対数を取ると
\begin{equation}
    \log{p(\bm{y})}
    = -\frac{m}{2}\log(2\pi)
    - \frac{1}{2} \log{\det(\tau K_m + \sigma^2 I)}
    - \frac{1}{2} \bm{y}^T (\tau K_m + \sigma^2 I)^{-1} \bm{y}
\end{equation}
となる.
これを最大化するようにパラメータを決定する.

$\lambda = \sigma^2 / \tau$ を固定すると,
\begin{equation}
    \log{p(\bm{y})}
    = -\frac{m}{2}\log(2\pi)
    - \frac{m}{2} \log{\tau}
    - \frac{1}{2} \log{\det(K_m + \lambda I)}
    - \frac{1}{2\tau} \bm{y}^T (K_m + \lambda I)^{-1} \bm{y}
\end{equation}
となる.
これを最大化するように $\tau$ を決定すると,
\begin{equation}
    \tau = \frac{1}{m} \bm{y}^T (K_m + \lambda I)^{-1} \bm{y}
    \label{eq:interp_kernel_param_coeff_tau}
\end{equation}
となり,そのときの $\log{p(\bm{y})}$ は
\begin{equation}
    \log{p(\bm{y})}
    = -\frac{m}{2}\log(2\pi)
    + \frac{m}{2} \log{m}
    - \frac{m}{2} \log(\bm{y}^T (K_m + \lambda I)^{-1} \bm{y})
    - \frac{1}{2} \log{\det(K_m + \lambda I)}
    - \frac{m}{2}
\end{equation}
となる.
$m$ はサンプルの数であり,定数だから,
パラメータ推定では
\begin{equation}
    m \log(\bm{y}^T (K_m + \lambda I)^{-1} \bm{y})
    + \log{\det(K_m + \lambda I)}
\end{equation}
を最小化すれば良い.

\section{固有値分解による数値解法}

密行列の固有値分解を用いてカーネル法の計算を行うことを考える.

カーネルの値を並べた行列 $K_m \in \setC^{m \times m}$ を次のように固有値分解する.
\begin{equation}
    K_m = V D V^*
\end{equation}
ここで,$V \in \setC^{m \times m}$ はエルミート行列であり,
$D \in \setR^{m \times m}$ は実数による対角行列である.
ここで,$K_m$ は半正定値とする
(このような性質を持つカーネルは正定値カーネルと呼ばれる \cite{Fukumizu2010}).
\ref{sec:interp_kernel_tikhonov} 節で導出したカーネルや
\ref{sec:interp_kernel_rbf} 節で示した Gaussian の RBF によるカーネルは
正定値カーネルである.
このとき,
\begin{equation}
    K_m + \lambda I = V (D + \lambda I) V^*
\end{equation}
が成り立つ.
$\lambda > 0$ であれば確実に $K_m + \lambda I$ が正定値となり,逆行列を持つ.
よって,
式 \eqref{eq:interp_kernel_tikhonov_coeff_c} で定めた係数 $\bm{c}$ は
\begin{align}
    \bm{c}
     & = (K_m + \lambda I)^{-1} \bm{y} \notag \\
     & = V (D + \lambda I)^{-1} V^* \bm{y}
\end{align}
と書ける.
さらに,
$V = (\bm{v}_1, \bm{v}_2, \ldots, \bm{v}_m)$,
$D = \diag(\lambda_1, \lambda_2, \ldots, \lambda_m)$
とおくと,
\begin{align}
    \bm{c}
     & = \sum_{i=1}^m \frac{1}{\lambda_i + \lambda} (\bm{v}_i^* \bm{y}) \bm{v}_i
\end{align}
のように表すこともできる.

なお,$K_m$, $\bm{y}$ が実数の行列とベクトルである場合,
パラメータ推定における
式 \eqref{eq:interp_kernel_param_coeff_tau} の $\tau$ は
\begin{align}
    \tau
     & = \frac{1}{m} \bm{y}^T (K_m + \lambda I)^{-1} \bm{y} \notag                    \\
     & = \frac{1}{m} \sum_{i=1}^m \frac{1}{\lambda_i + \lambda} (\bm{v}_i^T \bm{y})^2
\end{align}
のように書ける.
さらに,評価関数は
\begin{align}
     & \hphantom{=}
    m \log(\bm{y}^T (K_m + \lambda I)^{-1} \bm{y})
    + \log{\det(K_m + \lambda I)}
    \notag                                                                                   \\
     & = m \log\left(\sum_{i=1}^m \frac{1}{\lambda_i + \lambda} (\bm{v}_i^T \bm{y})^2\right)
    + \log\left(\prod_{i=1}^m (\lambda_i + \lambda)\right)
    \notag                                                                                   \\
     & = m \log\left(\sum_{i=1}^m \frac{1}{\lambda_i + \lambda} (\bm{v}_i^T \bm{y})^2\right)
    + \sum_{i=1}^m \log(\lambda_i + \lambda)
\end{align}
のように書ける.

% !TEX root = ../../main.tex
%

\section{大域最適解への応用}\label{sec:interp_kernel_global-optimization}

Gaussian Process (\ref{sec:regularization_kernel_gaussian-process} 節)
を利用して大域最適解を行う
Gaussian Process 最適化
\index{Gaussian Process さいてきか@Gaussian Process 最適化}
が考案されている
\cite{Srinivas2010}.

領域 $D \in \setR^d$ 上で関数 $f : D \to \setR$ を最小化する
制約なし最適化の問題を考える
\footnote{文献 \cite{Srinivas2010} では最大化を前提としているが,%
    ここでは \ref{part:optimization} 部に合わせて最小化を考える.}.
文献 \cite{Srinivas2010} のアルゴリズムにおいては,
各反復ごとにこれまでに算出したサンプル点
$(\bm{x}_i, f(\bm{x}_i))$ ($i = 1,2,\ldots,T$)
に対して Gaussian Process による補間を行い,
Gaussian Process の平均 $\mu_T(\bm{x})$ と標準偏差 $\sigma_T(\bm{x})$ を算出できるようにする.
そして,反復ごとに変化する係数 $\beta_t$ とパラメータ $\nu$ を用いて定義される目的関数
\begin{equation}
    F_T(\bm{x}) = \mu_T(\bm{x}) - \sqrt{\nu \beta_{T+1}} \sigma_T(\bm{x})
\end{equation}
を最小化するような $\bm{x}$ を次のサンプル点として選択する.
係数 $\nu$, $\beta_t$ は
\begin{itemize}
    \item $\nu = 1/5$, $\beta_t = 2 \log(|D| t^2 \pi^2 / 6 \delta)$,
          $\delta \in (0, 1)$
          \cite{Srinivas2010}
          \footnote{文献 \cite{Srinivas2010} 内の定理では $\nu = 1$ の場合を扱っていたが,%
              数値実験において $\nu = 1/5$ とすると結果が良くなったという記述がある.%
              また,文献 \cite{Srinivas2010} の数値実験では $\delta = 0.1$ が選択されている.}
    \item $\nu=1$, $\beta_t = 2 \log(t^{d/2+2} \pi^2 /3 \delta)$,
          $\delta \in (0, 1)$
          \cite{Brochu2010}
\end{itemize}
のように与えられる.
関数 $F_T$ を最小化するために別の大域最適解のアルゴリズムが必要となるが,
元の目的関数 $f$ の値を得るのに時間がかかる場合には有効なアルゴリズムである.

