% !TEX root = ../../main.tex
%

\section{微分などの線型作用素の適用}\label{sec:interp_kernel_operators}

カーネル法による補間
\begin{equation}
    f(\bm{r}) = \sum_{i=1}^m c_i K(\bm{r}, \bm{r}_i)
\end{equation}
を利用して線型作用素 $L$ を適用した結果を
\begin{equation}
    L f(\bm{r}) = \sum_{i=1}^m c_i L K(\bm{r}, \bm{r}_i)
\end{equation}
のように推定できる.

\subsection{RBF 補間の微分}\label{sec:interp_kernel_operators_rbf-differentiation}

RBF 補間に Laplacian などの微分の作用素を適用することで部分を表現したものは
偏微分方程式の解法 \cite{Fornberg2015,Chu2023} などに応用されている.

RBF の微分を行う作用素を Wendland の Compactly Supported RBF のように
式 \eqref{eq:rbf_wendland-csrbf_differentiation-operator} で定義すると,
2-ノルムを用いた RBF 補間におけるカーネル
\begin{equation}
    K(\bm{r}, \bm{r}') = \phi\left(\frac{\|\bm{r} - \bm{r}'\|_2}{c}\right)
\end{equation}
の $\bm{r}$ に対する勾配は以下のように書ける
\footnote{$\nabla$ を適用して式を整理することで導出できるため,導出の過程は省略している.}.
\begin{equation}
    \nabla K(\bm{r}, \bm{r}') =
    - (D \phi)\left(\frac{\|\bm{r} - \bm{r}'\|_2}{c}\right) \frac{\bm{r} - \bm{r}'}{c^2}
\end{equation}

同様に $\bm{r}$ に対する Laplacian は $\bm{r}$ の次元 $d$ を用いて以下のように書ける.
\begin{equation}
    \triangle K(\bm{r}, \bm{r}') =
    (D^2 \phi)\left(\frac{\|\bm{r} - \bm{r}'\|_2}{c}\right) \frac{\|\bm{r} - \bm{r}'\|_2^2}{c^4}
    -(D \phi)\left(\frac{\|\bm{r} - \bm{r}'\|_2}{c}\right) \frac{d}{c^2}
\end{equation}

$\bm{r}$ に対する Hessian は以下のように書ける.
\begin{equation}
    \nabla^2 K(\bm{r}, \bm{r}') =
    (D^2 \phi)\left(\frac{\|\bm{r} - \bm{r}'\|_2}{c}\right) \frac{(\bm{r} - \bm{r}')(\bm{r} - \bm{r}')^\top}{c^4}
    -(D \phi)\left(\frac{\|\bm{r} - \bm{r}'\|_2}{c}\right) \frac{I}{c^2}
\end{equation}

更に,$\bm{r}$ に対する 3 階,4 階の微分も以下のように書ける.
\begin{align}
    \nabla \triangle K(\bm{r}, \bm{r}') & =
    -(D^3 \phi)\left(\frac{\|\bm{r} - \bm{r}'\|_2}{c}\right)
    \frac{\|\bm{r} - \bm{r}'\|_2^2 (\bm{r} - \bm{r}')}{c^6}
    +(D^2 \phi)\left(\frac{\|\bm{r} - \bm{r}'\|_2}{c}\right)
    \frac{(d+2)(\bm{r} - \bm{r}')}{c^4}
    \\
    \triangle^2 K(\bm{r}, \bm{r}')      & =
    (D^4 \phi)\left(\frac{\|\bm{r} - \bm{r}'\|_2}{c}\right)
    \frac{\|\bm{r} - \bm{r}'\|_2^4}{c^8}
    -(D^3 \phi)\left(\frac{\|\bm{r} - \bm{r}'\|_2}{c}\right)
    \frac{2(d+2)\|\bm{r} - \bm{r}'\|_2^2}{c^6}
    \notag                                  \\ &\hspace{4em}
    +(D^2 \phi)\left(\frac{\|\bm{r} - \bm{r}'\|_2}{c}\right)
    \frac{d(d+2)}{c^4}
\end{align}

表 \ref{table:interp_kernel_example-rbfs} で示していた RBF の一部に対して作用素 $D$ を適用した例を
表 \ref{table:interp_kernel_rbf-derivatives} に示す.

\begin{table}[bp]
    \caption{RBF の微分の例}
    \label{table:interp_kernel_rbf-derivatives}
    \centering
    \begin{tabular}{lll}
        名称                    & 元の関数               & 微分                                             \\
        \hline
        Gaussian              & $e^{-r^2}$         & $D^n e^{-r^2} = 2^n e^{-r^2}$                  \\
        Inverse Multi-quadric & $1/\sqrt{1 + r^2}$ & $D (1/\sqrt{1 + r^2}) = 1/(1 + r^2)^{3/2}$     \\
        Inverse Quadric       & $1 / (1 + r^2)$    & $D (1 / (1 + r^2)) = 2/(1 + r^2)^2$            \\
        Sech                  & $1 / \cosh{r}$     & $D (1 / \cosh{r}) = \sinh{r} / (r \cosh^2{r})$
    \end{tabular}
\end{table}
