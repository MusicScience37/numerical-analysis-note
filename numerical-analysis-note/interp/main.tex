% !TEX root = ../main.tex
%

\part{補間}

\chapter{導入}

\index{ほかん@補間}
この部では,補間の手法をまとめる.
補間では,
サンプル点 $(\bm{r}_i, y_i) \in X \times \setC$ ($i = 1, 2, \ldots, m$) について
$f(\bm{r}_i) = y_i$
となるような関数 $f$ を算出する.
データがノイズを含んでいる場合は
正則化(\ref{part:regularization} 部)を利用して
$f(\bm{r}_i) \approx y_i$
となるような関数 $f$ を求める場合もある.

\RequirePackage{plautopatch}
\documentclass[a4j,dvipdfmx,
%,draft%
]{jsbook}
\setlength{\textwidth}{\fullwidth}
\setlength{\evensidemargin}{\oddsidemargin}

%% 表題の設定
\makeatletter
% 1ページ用
\newcommand\ktitle{数値解析ノート}
% 途中で表示するとき用(1行)
\newcommand\ktitles{\ktitle}
\makeatother

% !TEX root = main.tex
%

% チェック
\RequirePackage[l2tabu, orthodox]{nag}

%% パッケージ
%図
\usepackage[hiresbb]{graphicx}
\usepackage{caption}
\usepackage[subrefformat=parens]{subcaption}
%数式
\usepackage{amsmath}
\usepackage[all, warning]{onlyamsmath}
\allowdisplaybreaks
\def\equation{\gather}%%hyperref対応
\def\endequation{\endgather}%%hyperref対応
\usepackage{amssymb}
\usepackage{amsthm}
\newcommand{\hmmax}{0}
\newcommand{\bmmax}{0}
\usepackage{bm}
%SI単位系
\usepackage{siunitx}
%表
\usepackage{longtable}
%アルゴリズム
\usepackage[chapter]{algorithm}
\usepackage{algorithmicx}
\usepackage{algpseudocode}
%ソースコード
\usepackage{listings,extern/plistings/plistings}
\lstset{
    numbers=left,
    numberstyle=\footnotesize,
    basicstyle=\footnotesize,
    breaklines=true,
    frame={tb},
    flexiblecolumns=true
}
%URL
\usepackage{url}
\urlstyle{rm}
%しおり
\usepackage{bookmark}
%引用
\usepackage{cite}%下の3行がない限りhyperrefとは同時に使わない.
\makeatletter
\def\NAT@parse{\typeout{This is a fake Natbib command to fool Hyperref.}}
\makeatother
%ハイパーリンク
\usepackage{hyperref}%なるべく後
\usepackage{pxjahyper}%日本語対応
\renewcommand\UrlFont{\rmfamily}
%フォント(※順番を変えない.)
\usepackage[nomath]{lmodern}
\usepackage[T1]{fontenc}
\usepackage{textcomp}
\usepackage{newtxmath}
%自分のパッケージ(他のパッケージに依存するから最後にする)
\usepackage[book,amsthm,many]{kmath}
\usepackage{kmacro}

%タイトル
\title{\ktitle}
\author{椛島 健太}
\西暦
%hypersetup用の設定
\hypersetup{%
    bookmarksnumbered=true,%
    setpagesize=false,%
    pdftitle={\ktitles},%
    pdfauthor={椛島 健太},%
    pdfsubject={},%
    pdfkeywords={},%
    hidelinks,%
    pdfstartview=FitH,%
    pdfremotestartview=FitH%
}

%参考文献
\def\bibname{参考文献}

%数式の番号(章番号を使うようにする.)
\makeatletter
\renewcommand{\theequation}{%
\arabic{chapter}.\arabic{equation}%
}
\@removefromreset{equation}{section}
\@addtoreset{equation}{chapter}
\makeatother
 % 設定を読みこむ

\begin{document}

% タイトルページ
\maketitle

% copyright ページ
\thispagestyle{empty}
\null\vfill % 文章をページの下に寄せる
\noindent
\copyright \ 2021, Kenta Kabashima.\\
This work is licensed under the Creative Commons Attribution-ShareAlike 4.0 International License. To view a copy of this license, visit
\url{http://creativecommons.org/licenses/by-sa/4.0/}.

% 目次ページ
\setcounter{tocdepth}{2}
\tableofcontents

% !TEX root = main.tex
%

\chapter{本書について}

本書では,私が数値解析関係で調査したことをまとめる.
基本的には,実装のためにアルゴリズムやその理論を確認した結果をまとめることを目的としており,
C++ で実装したものは Git リポジトリ \cite{NumericalCollectionCpp} にて公開している.

数値解析の分野も幅広く,
扱う対象によって独特の記号が定義されるケースが少なくない,
そこで,本書では部または章ごとに記号の定義を記載している.

% !TEX root = ../main.tex
%

\chapter{特殊関数}

この章では,数値計算でしばしば利用される特殊関数について,
定義や基本的な性質,計算方法などを示す.

% !TEX root = ../main.tex
%

\section{Legendre 関数}\label{sec:special-function_legendre-function}

Legendre 関数は,$n=0,1,\ldots$ について次の式で表される
\cite[Section 5.2]{Morse1953}.
\begin{equation}
    P_n(x) = \frac{1}{2^n n!} \frac{d^n}{dx^n} (x - 1)^n
\end{equation}

Legendre 関数は $n$ 次多項式で,区間 $[-1, 1]$ において次のような直交性を持つ
\cite[Section 6.3]{Morse1953}.
\begin{equation}
    \int_{-1}^{1} P_n(x) P_m(x) = \frac{2}{2n + 1} \delta_{nm}
\end{equation}
さらに,$n$ 次の Legendre 関数は $n-1$ 次以下の任意の多項式 $f(x)$ と次のように直交する.
\begin{equation}
    \int_{-1}^{1} P_n(x) f(x) = 0
\end{equation}

計算には次のような公式を用いる\cite[Section6.3]{Morse1953}.
\begin{gather}
    P_{n+1}(x) = \frac{2n+1}{n+1} x P_n(x) - \frac{n}{n+1} P_{n-1}(x) \\
    (1-x^2) P_n'(x) = n(P_{n-1}(x) - x P_n(x))
\end{gather}


% !TEX root = ../main.tex
%

\part{数値積分}

\chapter{導入}

この部では,数値積分の手法をまとめる.

手計算で解析的に積分するのが困難な場合や,
解析的な積分結果だけでは具体的な数値が不明な場合など,
数値的に積分を行う必要がある場合は多々ある.
何次元のどのような形状の領域でどのような関数を積分するかに応じて
様々な積分手法が存在する.

\chapter{記号}

\begin{explainlist}
    $P_n(x)$ & Legendre 関数(\ref{sec:special-function_legendre-function} 節) \\
\end{explainlist}

% !TEX root = ../main.tex
%

\chapter{1 次元の数値積分}

本章では,
1 次元における積分
\begin{equation}
    \int_{a}^{b} f(x) dx
\end{equation}
に対する数値積分の手法についてまとめる.

\section{Gauss 積分公式}

1 次元における積分に対する数値積分の手法の 1 つに Gauss 積分公式がある.
($a$, $b$ は $-\infty$, $\infty$ でも良い.)

Gauss 積分公式では,次のように $x_1, x_2, \ldots, x_n$ 上での関数値の重み付き平均を用いる
\cite{Mori1993}.
\begin{equation}
    \int_{a}^{b} f(x) w(x) dx = \sum_{k = 1}^n w_k f(x_k)
\end{equation}
分点 $x_1, x_2, \ldots, x_n$ と重み $w_1, w_2, \ldots, w_n$ は,
区間 $(a, b)$ と重み関数 $w(x)$ に応じて決まる.

\subsection{Gauss-Legendre 公式}

$a$, $b$ が有限な場合に利用できる Gauss-Legendre 公式では,
Legendre 関数 $P_n(x)$ の $n$ 個の零点を分点 $x_k$ に用い,
重み関数は次のように算出する\cite{Mori1993}.
\begin{equation}
    w_k = \frac{2(1 - x_k^2)}{(n P_{n-1}(x_k))^2}
\end{equation}
分点 $x_k$ の算出には Newton-Raphson 法(\ref{chap:root-finding_newton-raphson} 章)を用いることができる.
初期値には
\begin{equation}
    x_k \approx \cos{\frac{\pi (k - 0.25)}{n + 0.5}}
\end{equation}
を使用すると良い\cite{Mori1993}.
また,$s_k = -x_{n-k}, w_k = w_{n-k}$ を用いて計算量を半減させることができる.

\section{二重指数関数型公式}

1 次元の数値積分において,
Gauss 公式よりも困難な積分へ対応できる手法として,
二重指数関数型公式(Double Exponential Formula, DE Formula)が存在する
\cite[6.1 節 (b)]{Mori1993}, \cite[Section 4.5]{Press2007}.
二重指数関数型公式では,
区間 $(a, b)$ 上での積分において次の式による変数変換を行い,
区間 $(-\infty, \infty)$ 上での積分にする.
\begin{equation}
    x = \frac{1}{2}(a + b) + \frac{1}{2}(b - a) \tanh \left(\frac{\pi}{2} \sinh{t}\right)
\end{equation}
$\tanh$, $\sinh$ はどちらも指数関数で定義されるため,
名前通り二重の指数関数による積分となっている.

二重指数関数型公式は,
無限積分
\begin{equation}
    \int_{-\infty}^{\infty} f(x) dx
\end{equation}
が有限和
\begin{equation}
    h \sum_{k = -\infty}^{\infty} f(kh)
\end{equation}
でよく近似できるということに基づいている\cite[6.1 節 (b)]{Mori1993}.
二重指数関数型公式では,
積分領域の端で被積分関数が発散する
\begin{equation}
    \int_{-1}^{1} \frac{1}{\sqrt{1-x^2}} dx
\end{equation}
のような積分でも,
積分領域の端を無限遠へ移すことにより,
Gauss-Legendre 積分公式よりも安定して数値積分を行うことができる.
ただし,指数関数を何度も計算する必要があるため,
二重指数関数型公式の方が計算時間は長くなる可能性がある.

\subsection{有限区間上の積分}

1 次元の有限区間上の積分
\begin{equation}
    I = \int_{a}^{b} f(x) dx
\end{equation}
を考える.変数変換
\begin{equation}
    x = \phi(t) \equiv \frac{1}{2}(a + b) + \frac{1}{2}(b - a) \tanh \left(\frac{\pi}{2} \sinh{t}\right)
\end{equation}
を適用する場合,
\begin{equation}
    \frac{dx}{dt} = \phi'(t)
    = \frac{\pi}{4} (b - a) \frac{\cosh{t}}{\cosh^2 \left(\frac{\pi}{2} \sinh{t}\right)}
\end{equation}
を用いて
\begin{equation}
    I = \int_{a}^{b} f(x) dx
    = \int_{-\infty}^{\infty} f(\phi(t)) \phi'(t)
\end{equation}
のように近似でき,
\begin{equation}
    T_h = h \sum_{k = -N}^{N} f(\phi(kh)) \phi'(kh)
\end{equation}
で近似できる.
ここで,パラメータ $c$ は $1$ か $\pi/2$ とする場合が多い\cite[Section 4.5]{Press2007}.
また,パラメータ $h$ の最適値は
\begin{equation}
    h \approx \frac{\log(2 \pi N)}{N}
\end{equation}
であり,この場合の数値積分の誤差のオーダーは
\begin{equation}
    |T_h - I| \approx \exp(-kN / \log{N})
\end{equation}
となる\cite[Section 4.5]{Press2007}.
点数 $N$ を 2 倍にすると有効桁数が約 2 倍になる.

なお,実装時はオーバーフローしないように
\begin{align}
    \phi'(t)
     & = \frac{\pi}{4} (b - a) \frac{\cosh{t}}{\cosh^2 \left(\frac{\pi}{2} \sinh{t}\right)}                                              \\
     & = \frac{\pi}{4} (b - a) \frac{4 \cosh{t}}{(\exp\left(\frac{\pi}{2} \sinh{t}\right) + \exp\left(-\frac{\pi}{2} \sinh{t}\right))^2} \\
     & = \pi (b - a) \frac{\cosh{t} \exp(-\pi \sinh{t})}{(1 + \exp(-\pi \sinh{t}))^2}
\end{align}
とすると良い\cite[Section 4.5.2]{Press2007}.

\subsection{半無限区間上の積分}

1 次元の半無限区間上の積分
\begin{equation}
    I = \int_{0}^{\infty} f(x) dx
\end{equation}
を考える.
これに対する二重指数関数型公式の変数変換は
\begin{equation}
    x = \phi(t) \equiv \exp(\pi \sinh{t})
\end{equation}
である\cite[Section 4.5.3]{Press2007}.
微分すると
\begin{equation}
    \frac{dx}{dt} = \phi'(t)
    = \pi \exp(\pi \sinh{t}) \cosh{t}
\end{equation}
となる.

半無限区間上の積分,および次節の無限区間上の積分においては,
変数変換後の変数 $t$ において区間 $[-4, 4]$ までの範囲で有限和による近似を行えば良い
\cite[4.5.3]{Press2007}.

\subsection{無限区間上の積分}

1 次元の無限区間上の積分
\begin{equation}
    I = \int_{-\infty}^{\infty} f(x) dx
\end{equation}
を考える.
これに対する二重指数関数型公式の変数変換は
\begin{equation}
    x = \phi(t) \equiv \sinh \left(\frac{\pi}{2} \sinh{t}\right)
\end{equation}
である\cite[Section 4.5.3]{Press2007}.
微分すると
\begin{equation}
    \frac{dx}{dt} = \phi'(t)
    = \frac{\pi}{2} \cosh \left(\frac{\pi}{2} \sinh{t}\right) \cosh{t}
\end{equation}
となる.

% !TEX root = ../main.tex
%

\chapter{多次元の数値積分}

\section{三角形上の数値積分}

2 次元平面上のメッシュや,
3 次元空間内の曲面を表すメッシュなど,
三角形が必要になる場面はしばしば存在する.
そのような三角形での使用を目的とした数値積分手法の 1 つを紹介する.

三点 $\bm{a}, \bm{b}, \bm{c}$ からなる三角形
\footnote{何次元空間の三角形かは問わない.}
において,
三角形上の点 $\bm{p}$ を重心 $\bm{g}$ を用いて,
\begin{equation}
    \bm{p} = \xi_1 (\bm{a} - \bm{g}) + \xi_2 (\bm{b} - \bm{g}) + \xi_3 (\bm{c} - \bm{g}) + \bm{g}
\end{equation}
のように
$\xi_1 \ge 0$,
$\xi_2 \ge 0$,
$\xi_3 \ge 0$,
$\xi_1 + \xi_2 + \xi_3 \le 1$
なる $\xi_1$, $\xi_2$, $\xi_3$ の組で示すことができる.
この原点を中心とした座標系は barycentric coordinates と呼ばれる.
文献 \cite{Laurie1982} では,
この barycentric coordinates を用いた対照的な積分公式を算出している.

文献 \cite{Laurie1982} における積分公式は次のように計算する.
\begin{align}
    I_5[f]     & = \Delta \sum_{i=0}^2 w_5^{(i)} I^{(i)}[f] \\
    I_8[f]     & = \Delta \sum_{i=0}^5 w_8^{(i)} I^{(i)}[f] \\
    I^{(i)}[f] & = \frac{1}{6} \sum_{\substack{k = 1, 2, 3  \\ l = 1, 2, 3 \\ k \neq l}}
    f(\xi_k^{(i)}, \xi_l^{(i)})
\end{align}
ここで,
$I_5[f]$, $I_8[f]$ はそれぞれ関数 $f$ の 5, 8 次精度の積分となっている.
$\Delta$ は三角形の面積であり,
パラメータ $w_5^{(i)}$, $w_8^{(i)}$, $\xi_k^{(i)}$ は
表 \ref{table:integration_multi-dim_cubtri-parameters} のようになっている.

\begin{table}[bp]
    \caption{CUBTRI \cite{Laurie1982} のパラメータ($\phi = \sqrt{15}$, $\sigma = \sqrt{7}$)}
    \label{table:integration_multi-dim_cubtri-parameters}
    \centering
    \begin{tabular}{c|ccc}
        $i$                                                   &
        $\xi_1^{(i)}$                                         &
        $\xi_2^{(i)}$                                         &
        $\xi_3^{(i)}$                                           \\
        \hline
        0                                                     &
        $1 / 3$                                               &
        $1 / 3$                                               &
        $1 / 3$                                                 \\
        1                                                     &
        $3 / 7 + 2 \phi / 21$                                 &
        $2 / 7 - \phi / 21$                                   &
        $2 / 7 - \phi / 21$                                     \\
        2                                                     &
        $3 / 7 - 2 \phi / 21$                                 &
        $2 / 7 + \phi / 21$                                   &
        $2 / 7 + \phi / 21$                                     \\
        3                                                     &
        $4 / 9 + \phi / 9 + \sigma / 9 - \sigma \phi / 45$    &
        $5 / 18 - \phi / 18 - \sigma / 18 + \sigma \phi / 90$ &
        $5 / 18 - \phi / 18 - \sigma / 18 + \sigma \phi / 90$   \\
        4                                                     &
        $4 / 9 - \phi / 9 + \sigma / 9 + \sigma \phi / 45$    &
        $5 / 18 + \phi / 18 - \sigma / 18 - \sigma \phi / 90$ &
        $5 / 18 + \phi / 18 - \sigma / 18 - \sigma \phi / 90$   \\
        5                                                     &
        $5 / 18 - \phi / 18 - \sigma / 18 + \sigma \phi / 90$ &
        $5 / 18 + \phi / 18 - \sigma / 18 - \sigma \phi / 90$ &
        $4 / 9 + \sigma / 9$
    \end{tabular}
    \begin{tabular}{c|c}
        $i$ &
        $w_5^{(i)}$            \\
        \hline
        0   &
        $9 / 40$               \\
        1   &
        $31 / 80 - \phi / 400$ \\
        2   &
        $31 / 80 + \phi / 400$ \\
    \end{tabular}
    \begin{tabular}{c|c}
        $i$ &
        $w_8^{(i)}$                                                        \\
        \hline
        0   &
        $7137 / 62720 - 45 \sigma / 1568$                                  \\
        1   &
        $-9301697 / 4695040 - 13517313 \phi / 23475200
            + 764885 \sigma / 939008 + 198763 \phi \sigma / 939008$        \\
        2   &
        $-9301697 / 4695040 + 13517313 \phi / 23475200
            + 764885 \sigma / 939008 - 198763 \phi \sigma / 939008$        \\
        3   &
        $102791225 / 59157504 + 23876225 \phi / 59157504
            - 34500875 \sigma / 59157504 - 9914825 \phi \sigma / 59157504$ \\
        4   &
        $102791225 / 59157504 - 23876225 \phi / 59157504
            - 34500875 \sigma / 59157504 + 9914825 \phi \sigma / 59157504$ \\
        5   &
        $11075 / 8064 - 125 \sigma / 288$                                  \\
    \end{tabular}
\end{table}


% !TEX root = ../main.tex
%

\part{求根アルゴリズム}

\chapter{導入}

求根アルゴリズム (root-finding algorithm)
は,微分などの演算を含まない通常の方程式
$f(\bm{x})=\bm{0}$ ($f : \setR^n \to \setR^m$)
の解を求めるためのアルゴリズムである.
関数 $f$ の種類に依り様々な手法が存在する.

\chapter{記号}

本部で使用する記号を以下に示す.

\begin{explainlist}
    $\setR$ & 実数の集合 \\
    $f'(x)$ & 1 変数関数 $f : \setR \to \setR$ の微分係数 \\
    $\nabla f$ & 関数 $f : \setR^n \to \setR$ の勾配 \\
    $\frac{\partial \bm{f}}{\partial \bm{x}}$ & 関数 $\bm{f} : \setR^n \to \setR^n$ の Jacobian \\
    $\|\bm{x}\|_2$ & ベクトル $\bm{x}$ の 2-ノルム \\
\end{explainlist}

% !TEX root = ../main.tex
%

\chapter{Newton-Raphson 法}\label{chap:root-finding_newton-raphson}

Newton-Raphson 法は,
次の式のような更新式で反復的に方程式 $f(x) = 0$ の根を求める手法である.
\begin{equation}
    x_{n+1} = x_n - \frac{f(x_n)}{f'(x_n)}
\end{equation}

\section{1 次元の方程式の場合}

1 次元の方程式 $f(x) = 0$ ($f: \setR \to \setR$) の場合,
1 次近似
\begin{equation}
    f(x_{n+1}) \approx f(x_n) + f'(x_n) (x_{n+1} - x_n)
\end{equation}
の根を求めると
\begin{equation}
    x_{n+1} = x_n - \frac{f(x_n)}{f'(x_n)}
    \label{eq:root-finding_newton-raphson_one-dim-update-law}
\end{equation}
と更新式が導かれる.

\subsection{平方根の算出}

ここでは,平方根の算出に Newton-Raphson 法を適用してみる.

$a>0$ について $\sqrt{a}$ を算出する場合,次の関数 $f$ の正の根を求めれば良い.
\begin{equation}
    f(x) = x^2 - a
\end{equation}
この関数を微分すると
\begin{equation}
    f'(x) = 2x
\end{equation}
となる.
よって,更新式は次のようになる.
\begin{align}
    x_{n+1} & = x_n - \frac{f(x_n)}{f'(x_n)}          \notag \\
            & = x_n - \frac{x_n^2 - a}{2x_n}          \notag \\
            & = \frac{1}{2}\left(x_n + \frac{a}{x_n}\right)
    \label{eq:root-finding_newton-raphson_sqrt-update-law}
\end{align}

\begin{theorem}
    式 \eqref{eq:root-finding_newton-raphson_sqrt-update-law} の更新式は,
    初期値 $x_0$ が $x_0 > 0$ を満たす場合,
    平方根 $\sqrt{a}$ に収束する.
\end{theorem}
\begin{proof}
    $x_0 = \sqrt{a}$ の場合は
    $k = 1, 2, \ldots$ において $x_k = \sqrt{a}$ が成り立つ.
    つまり,平方根 $\sqrt{a}$ に収束している.
    そこで,以下では $x_0 \neq \sqrt{a}$ とする.

    更新後の値 $x_{n+1}$ と平方根 $\sqrt{a}$ の差は次のようになる.
    \begin{align}
          & x_{n+1} - \sqrt{a}                                                      \notag \\
        = & \frac{1}{2} (x_n - \sqrt{a}) + \frac{1}{2} \frac{a - \sqrt{a} x_n}{x_n} \notag \\
        = & \frac{1}{2} (x_n - \sqrt{a}) \left(1 - \frac{\sqrt{a}}{x_n}\right)      \notag \\
        = & \frac{1}{2x_n} (x_n - \sqrt{a})^2
        \label{eq:root-finding_newton-raphson_sqrt-error-update}
    \end{align}

    $x_0 > 0$ かつ $x_0 \neq \sqrt{a}$ の場合,
    式 \eqref{eq:root-finding_newton-raphson_sqrt-error-update} より
    $x_1 - \sqrt{a} > 0$ が成り立つ.
    さらに,$k = 2, 3, \ldots$ において $x_k - \sqrt{a} > 0$ であることも帰納的に成り立つ.
    よって,$k = 1, 2, \ldots$ において $x_k - \sqrt{a} > 0$ である.

    また,$a > 0$ より $\sqrt{a} > 0$ であることを用いると,
    \begin{align}
          & x_{n+1} - \sqrt{a}                           \notag \\
        = & \frac{1}{2x_n} (x_n - \sqrt{a})^2            \notag \\
        = & \frac{x_n - \sqrt{a}}{2x_n} (x_n - \sqrt{a}) \notag \\
        < & \frac{x_n}{2x_n} (x_n - \sqrt{a})            \notag \\
        = & \frac{1}{2} (x_n - \sqrt{a})
    \end{align}
    となる.

    以上から,
    \begin{equation}
        0 < x_{n+1} - \sqrt{a} < \frac{1}{2} (x_n - \sqrt{a})
    \end{equation}
    が成り立ち,$x_n$ の列が平方根 $\sqrt{a}$ へ収束する.
\end{proof}

\subsection{べき根の算出}

一般に $r$ 乗根($r = 2, 3, \ldots$)を算出することを考える.

$a>0$ について $\sqrt[r]{a}$ を算出する場合,次の関数 $f$ の正の根を求めれば良い.
\begin{equation}
    f(x) = x^r - a
\end{equation}
この関数を微分すると
\begin{equation}
    f'(x) = rx^{r-1}
\end{equation}
となる.
よって,更新式は次のようになる.
\begin{align}
    x_{n+1} & = x_n - \frac{f(x_n)}{f'(x_n)}             \notag \\
            & = x_n - \frac{x_n^r - a}{rx_n^{r-1}}       \notag \\
            & = \frac{r-1}{r} x_n + \frac{a}{rx_n^{r-1}}
    \label{eq:root-finding_newton-raphson_rth-root-update-law}
\end{align}

\begin{theorem}
    式 \eqref{eq:root-finding_newton-raphson_rth-root-update-law} の更新式は,
    初期値 $x_0$ が $x_0 > 0$ を満たす場合,
    べき根 $\sqrt[r]{a}$ に収束する.
\end{theorem}
\begin{proof}
    $x_0 = \sqrt[r]{a}$ の場合は
    $k = 1, 2, \ldots$ において $x_k = \sqrt[r]{a}$ が成り立つ.
    つまり,平方根 $\sqrt[r]{a}$ に収束している.
    そこで,以下では $x_0 \neq \sqrt[r]{a}$ とする.

    更新後の値 $x_{n+1}$ とべき根 $\sqrt[r]{a}$ の差は次のようになる.
    \begin{align}
          & x_{n+1} - \sqrt[r]{a}                                    \notag \\
        = & \frac{r-1}{r} \left(x_n - \sqrt[r]{a}\right)
        + \frac{1}{r} \left(\frac{a}{x_n^{r-1}} - \sqrt[r]{a}\right) \notag \\
        = & \frac{r-1}{r} \left(x_n - \sqrt[r]{a}\right)
        -\frac{\sqrt[r]{a}}{r} \left(1 - \left(\frac{\sqrt[r]{a}}{x_n}\right)^{r-1}\right)
    \end{align}
    これに等比級数の公式
    \begin{equation}
        \sum_{k = 1}^n x^{k-1} = \frac{1 - x^n}{1 - x}
    \end{equation}
    を適用すると,
    \begin{align}
          & x_{n+1} - \sqrt[r]{a}                                                \notag     \\
        = & \frac{r-1}{r} \left(x_n - \sqrt[r]{a}\right)
        -\frac{\sqrt[r]{a}}{r} \left(1 - \frac{\sqrt[r]{a}}{x_n}\right)
        \sum_{k = 1}^{r - 1} \left(\frac{\sqrt[r]{a}}{x_n}\right)^{k - 1}        \notag     \\
        = & \frac{r-1}{r} \left(x_n - \sqrt[r]{a}\right)
        -\frac{1}{r} \left(x_n - \sqrt[r]{a}\right)
        \sum_{k = 1}^{r - 1} \left(\frac{\sqrt[r]{a}}{x_n}\right)^k              \notag     \\
        = & \frac{1}{r} \left(x_n - \sqrt[r]{a}\right)
        \left(r - 1 - \sum_{k = 1}^{r - 1} \left(\frac{\sqrt[r]{a}}{x_n}\right)^k\right)
        \label{eq:root-finding_newton-raphson_rth-root-error-update-linear}                 \\
        = & \frac{1}{r} \left(x_n - \sqrt[r]{a}\right)
        \sum_{k = 1}^{r - 1} \left(1 - \left(\frac{\sqrt[r]{a}}{x_n}\right)^k\right) \notag \\
        = & \frac{1}{r} \left(x_n - \sqrt[r]{a}\right)
        \left(1 - \frac{\sqrt[r]{a}}{x_n}\right)
        \sum_{k = 1}^{r - 1} \sum_{l = 1}^{k}
        \left(\frac{\sqrt[r]{a}}{x_n}\right)^{l - 1}                                 \notag \\
        = & \frac{1}{rx_n} \left(x_n - \sqrt[r]{a}\right)^2
        \sum_{k = 1}^{r - 1} \sum_{l = 1}^{k}
        \left(\frac{\sqrt[r]{a}}{x_n}\right)^{l - 1}
        \label{eq:root-finding_newton-raphson_rth-root-error-update-quadratic}
    \end{align}
    となる.

    $x_0 > 0$ かつ $x_0 \neq \sqrt[r]{a}$ の場合,
    式 \eqref{eq:root-finding_newton-raphson_rth-root-error-update-quadratic} より
    $x_1 - \sqrt[r]{a} > 0$ が成り立つ.
    さらに,$k = 2, 3, \ldots$ において $x_k - \sqrt[r]{a} > 0$ であることも帰納的に成り立つ.
    よって,$k = 1, 2, \ldots$ において $x_k - \sqrt[r]{a} > 0$ である.

    さらに,$x_n > \sqrt[r]{a} > 0$ であることから
    $\sqrt[r]{a} / x_n > 0$ が成り立つから,
    式 \eqref{eq:root-finding_newton-raphson_rth-root-error-update-linear} より
    \begin{align}
          & x_{n+1} - \sqrt[r]{a}                          \notag \\
        < & \frac{r - 1}{r} \left(x_n - \sqrt[r]{a}\right)
    \end{align}
    となる.

    以上から,
    \begin{equation}
        0 < x_{n+1} - \sqrt[r]{a} < \frac{r - 1}{r} \left(x_n - \sqrt[r]{a}\right)
        \label{eq:root-finding_newton-raphson_rth-root-error-convergence}
    \end{equation}
    が成り立ち,$x_n$ の列がべき根 $\sqrt[r]{a}$ へ収束する.
\end{proof}

$r$ が奇数の場合,
負の数 $a < 0$ に対してべき根 $\sqrt[r]{a} < 0$ が存在する.
この場合についてもべき根の算出を考える.

\begin{theorem}
    $r = 3, 5, 7, \ldots$ かつ $a < 0$ の場合を考える.
    式 \eqref{eq:root-finding_newton-raphson_rth-root-update-law} の更新式は,
    初期値 $x_0$ が $x_0 < 0$ を満たす場合,
    べき根 $\sqrt[r]{a}$ に収束する.
\end{theorem}
\begin{proof}
    $x_0 = \sqrt[r]{a}$ の場合は
    $k = 1, 2, \ldots$ において $x_k = \sqrt[r]{a}$ が成り立つ.
    つまり,平方根 $\sqrt[r]{a}$ に収束している.
    そこで,以下では $x_0 \neq \sqrt[r]{a}$ とする.

    $x_0 < 0$ かつ $x_0 \neq \sqrt[r]{a}$ の場合,
    式 \eqref{eq:root-finding_newton-raphson_rth-root-error-update-quadratic} より
    $x_1 - \sqrt[r]{a} < 0$ が成り立つ.
    さらに,$k = 2, 3, \ldots$ において $x_k - \sqrt[r]{a} < 0$ であることも帰納的に成り立つ.
    よって,$k = 1, 2, \ldots$ において $x_k - \sqrt[r]{a} < 0$ である.

    さらに,$x_n < \sqrt[r]{a} < 0$ であることから
    $\sqrt[r]{a} / x_n > 0$ が成り立つから,
    式 \eqref{eq:root-finding_newton-raphson_rth-root-error-update-linear} より
    \begin{align}
          & x_{n+1} - \sqrt[r]{a}                          \notag \\
        > & \frac{r - 1}{r} \left(x_n - \sqrt[r]{a}\right)
    \end{align}
    となる.

    以上から,
    \begin{equation}
        \frac{r - 1}{r} \left(x_n - \sqrt[r]{a}\right) < x_{n+1} - \sqrt[r]{a} < 0
    \end{equation}
    が成り立ち,$x_n$ の列がべき根 $\sqrt[r]{a}$ へ収束する.
\end{proof}

\section{2 次元以上の方程式の場合}

一般に $n$ 次元($n=1,2,\ldots$)の方程式
$\bm{f}(\bm{x}) = \bm{0}$ ($\bm{f}: \setR^n \to \setR^n$) の場合,
1 次近似
\begin{equation}
    \bm{f}(\bm{x}_{n+1}) \approx
    \bm{f}(\bm{x}_n) + \frac{\partial \bm{f}}{\partial \bm{x}} (\bm{x}_{n+1} - \bm{x}_n)
\end{equation}
をもとに更新式
\begin{equation}
    \bm{x}_{n+1} = \bm{x}_n - \left(\frac{\partial \bm{f}}{\partial \bm{x}}\right)^{-1} \bm{f}(\bm{x}_n)
\end{equation}
が得られる.


% !TEX root = ../main.tex
%

\part{最適化}

\chapter{最適化問題}

この部では,最適化のアルゴリズムをまとめる.

最適化問題は一般に次のように書ける.

\begin{align}\label{optimization_general_problem}
    \text{minimize} \hspace{1em} & f(\bm{x})                 \\
    \text{s.t.} \hspace{1em}     & \bm{g}(\bm{x}) \le \bm{0} \\
                                 & \bm{h}(\bm{x}) = \bm{0}
\end{align}

ここで,
$\bm{x} \in \setR^n$,
$f : \setR^n \to \setR$,
$\bm{g} : \setR^n \to \setR^m$,
$\bm{h} : \setR^n \to \setR^r$
とする.
また,ベクトル同士の「$\le$」による比較は
全ての要素において「$\le$」の関係が成り立つことを意味する.

問題\eqref{optimization_general_problem}において,
関数$f$は目的関数,
$\bm{g}(\bm{x}) \le \bm{0}$, $\bm{h}(\bm{x}) = \bm{0}$は制約条件と呼ばれ,
制約条件を満たす中で目的関数を最小化する問題を示している.
最小化した結果の変数 $\bm{x}$ は最適解と呼ばれ,
目的関数の値は最適値と呼ばれる.

なお,最小化でなく最大化で定式化する場合もあるが,
ここでは最小化に統一して説明する.

\chapter{記号}

本部で使用する記号を以下に示す.

\begin{explainlist}
    $\setR$ & 実数の集合 \\
    $\nabla f$ & 関数 $f : \setR^n \to \setR$ の勾配 \\
    $\nabla^2 f$ & 関数 $f : \setR^n \to \setR$ の Hessian \\
    $O$ & 零行列 \\
    $I$ & 単位行列 \\
    $A \succ O$ & 正方行列 $A$ は正定値である \\
    $A \succeq O$ & 正方行列 $A$ は半正定値である \\
    $A \succeq B$ & $A - B \succeq O$ となる \\
    $\|\bm{x}\|_2$ & ベクトル $\bm{x}$ の 2-ノルム \\
\end{explainlist}

\chapter{基本的な定義と性質}

\begin{definition}[{\cite[Section 6.4]{Luenberger2003}},{\cite[Section 3.1.1]{Boyd2004}}]
    関数 $f : \setR^n \to \setR$ が
    $\forall \bm{x}, \bm{y} \in \setR^n$, $\forall \alpha \in [0, 1]$ に対して
    \begin{equation}
        f\left(\alpha \bm{x} + (1-\alpha) \bm{y}\right)
        \le \alpha f(\bm{x}) + (1-\alpha) f(\bm{y})
    \end{equation}
    を満たす場合,関数 $f$ を凸関数(convex function)と呼ぶ.
\end{definition}

\begin{theorem}[{\cite[Section 6.4]{Luenberger2003}},{\cite[Section 3.1.3]{Boyd2004}}]
    $C^1$ 級関数 $f : \setR^n \to \setR$ が凸関数であるための必要十分条件は
    $\forall \bm{x}, \bm{y} \in \setR^n$ に対して
    \begin{equation}
        f(\bm{y}) \ge f(\bm{x}) + \nabla f(\bm{x})^\top (\bm{y} - \bm{x})
    \end{equation}
    が成り立つことである.
\end{theorem}

\begin{theorem}[{\cite[Section 6.4]{Luenberger2003}},{\cite[Section 3.1.3]{Boyd2004}}]
    $C^2$ 級関数 $f : \setR^n \to \setR$ が凸関数であるための必要十分条件は
    $\forall \bm{x} \in \setR^n$ に対して
    \begin{equation}
        \nabla^2 f(\bm{x}) \succeq O
    \end{equation}
    が成り立つことである.
\end{theorem}

\begin{definition}[{\cite[Section 6.4]{Luenberger2003}},{\cite[Section 3.1.1]{Boyd2004}}]
    関数 $f : \setR^n \to \setR$ が
    $\forall \bm{x}, \bm{y} \in \setR^n$, $\forall \alpha \in [0, 1]$ に対して
    \begin{equation}
        f\left(\alpha \bm{x} + (1-\alpha) \bm{y}\right)
        < \alpha f(\bm{x}) + (1-\alpha) f(\bm{y})
    \end{equation}
    を満たす場合,関数 $f$ は狭義凸関数(strictly convex function)であるという.
\end{definition}

\begin{definition}[{\cite[Section 9.1.2]{Boyd2004}}]
    $C^2$ 級関数 $f : \setR^n \to \setR$ が
    ある定数 $m > 0$ を持ち,
    $\forall \bm{x} \in \setR^n$ に対して
    \begin{equation}
        \nabla^2 f(\bm{x}) \succeq m I
    \end{equation}
    を満たす場合,関数 $f$ は強凸関数(strongly convex function)であるという.
\end{definition}

\begin{theorem}[{\cite[Section 9.1.2]{Boyd2004}}]
    $C^2$ 級関数 $f : \setR^n \to \setR$ が強凸関数である場合,
    $\forall \bm{x}, \bm{y} \in \setR^n$ に対して
    \begin{equation}
        f(\bm{y}) \ge f(\bm{x}) + \nabla f(\bm{x})^\top (\bm{y} - \bm{x})
        + \frac{m}{2} \| \bm{y} - \bm{x} \|_2^2
    \end{equation}
    が成り立つ.
\end{theorem}

\begin{theorem}[{\cite[Section 9.1.2]{Boyd2004}}]
    $C^2$ 級関数 $f : \setR^n \to \setR$ が強凸関数である場合,
    関数 $f$ が最小となる $\bm{x} \in \setR^n$ は一意的に定まる.
\end{theorem}

% !TEX root = ../main.tex
%

\chapter{制約なし最適化}

ここでは,次のような制約のない最適化問題の解法をまとめる.

\begin{align}
    \text{minimize} \hspace{1em}& f(\bm{x}) \\
    \text{s.t.} \hspace{1em}& \bm{x} \in \setR^n
\end{align}

\input{optimization/one-dim-section-search.tex}
\input{optimization/descent-methods.tex}


% !TEX root = ../main.tex
%

\part{常微分方程式の数値解法}

\chapter{導入}

\index{じょうびぶんほうていしき@常微分方程式}
\index{ordinary differential equation|see{常微分方程式}}
この部では,常微分方程式 (ordinary differential equation, ODE) の数値解法をまとめる.

関数 $\bm{y}(t)$ の常微分方程式は一般に
\begin{equation}
    \bm{f}(t, \bm{y}, \dot{\bm{y}}, \ddot{\bm{y}}, \dddot{\bm{y}}, \ddddot{\bm{y}}, \ldots) = \bm{0}
\end{equation}
のように書ける.

% !TEX root = ../main.tex
%

\chapter{Runge-Kutta 法}

Runge-Kutta 法 (Runge-Kutta method) では次のような形式の初期値問題を数値的に解く
\cite{Mitsui1993}.
\begin{equation}
    \begin{cases}
        \dot{\bm{y}} = \bm{f}(t, \bm{y}) \\
        \bm{y}(0) = \bm{y}_0
    \end{cases}
\end{equation}

Runge-Kutta 法は,時刻 $t$ における変数 $\bm{y}(t)$ の値から,
次のような形式で時刻 $t + h$ における変数 $\bm{y}(t + h)$ の計算を行う.
\begin{align}
    \bm{k}_i      & = \bm{f}\left(t + c_i h, \bm{y}(t) + h \sum_{j = 1}^s a_{ij} \bm{k}_j \right)
                  & \text{for $i = 1, 2, \ldots, s$}
    \label{eq:ode_runge-kutta_k-law}                                                              \\
    \bm{y}(t + h) & = \bm{y}(t) + \sum_{i=1}^s b_i \bm{k}_i
    \label{eq:ode_runge-kutta_y-law}
\end{align}

ここで,時間の更新幅 $h$ はステップ幅と呼ばれる.
Runge-Kutta 法には,
整数 $s$ (段数と呼ばれる)と係数 $a_{ij}$, $b_i$, $c_i$ の異なる様々な公式が存在する.
係数 $a_{ij}$, $b_i$, $c_i$ は
表 \ref{table:ode_runge-kutta_butcher-array-general} のような
Butcher 配列と呼ばれる形式で記載される.

\begin{table}[bp]
    \caption{Butcher 配列における係数の並べ方}
    \label{table:ode_runge-kutta_butcher-array-general}
    \centering
    \begin{tabular}{c|ccccc}
        $c_1$    & $a_{11}$ & $a_{12}$ & $a_{13}$ & $\cdots$ & $a_{1s}$ \\
        $c_2$    & $a_{21}$ & $a_{22}$ & $a_{23}$ & $\cdots$ & $a_{2s}$ \\
        $c_3$    & $a_{31}$ & $a_{32}$ & $a_{33}$ & $\cdots$ & $a_{3s}$ \\
        $\vdots$ & $\vdots$ & $\vdots$ & $\vdots$ & $\ddots$ & $\vdots$ \\
        $c_s$    & $a_{s1}$ & $a_{s2}$ & $a_{s3}$ & $\cdots$ & $a_{ss}$ \\
        \hline
                 & $b_1$    & $b_2$    & $b_3$    & $\cdots$ & $b_s$
    \end{tabular}
\end{table}

Runge-Kutta 法の公式は次のように分類される.

\begin{description}
    \item[陽的 Runge-Kutta 法] $j \ge i$ について $a_{ij} = 0$ となっている場合,
          $\bm{k}i$ は $\bm{k}_1, \bm{k}_2, \ldots, \bm{k}_s$
          の順に式 \eqref{eq:ode_runge-kutta_k-law} の右辺を評価することで計算できる.
          このような場合は陽的 Runge-Kutta 法と呼ばれる.
    \item[半陰的 Runge-Kutta 法] $j > i$ について $a_{ij} = 0$ となっている場合,
          $\bm{k}i$ は $\bm{k}_1, \bm{k}_2, \ldots, \bm{k}_s$
          の順に式 \eqref{eq:ode_runge-kutta_k-law} を $\bm{k}_i$ について解くことで計算できる.
          このような場合は半陰的 Runge-Kutta 法と呼ばれる.
    \item[陰的 Runge-Kutta 法] $j > i$ でも $a_{ij} \neq 0$ となる係数が存在する場合,
          $\bm{k}i$ は $i = 1, 2, \ldots, s$ について連立した
          式 \eqref{eq:ode_runge-kutta_k-law} を $\bm{k}_i$ について解くことで計算する.
          このような場合は陰的 Runge-Kutta 法と呼ばれる.
\end{description}

陽的 Runge-Kutta 法の方が計算は単純だが,
陰的 Runge-Kutta 法は

\begin{itemize}
    \item 硬い系と呼ばれる比較的不安定な微分方程式で解が安定しやすい.
    \item 陽的 Runge-Kutta 法よりも少ない段数でより高い次数を出せる.
          (後述する公式の実例を見ると分かる.)
\end{itemize}

といった利点を持つ.
そのため,解が安定するようにステップ幅を調整した場合,
陰的 Runge-Kutta 法の方がステップ幅を大きくとることができ,
目的の時刻 $t$ までの解を得るために必要な計算時間は少なくなる場合がある.

また,Runge-Kutta 法の公式の精度を示す数値として次数が存在する.
変数の近似値 $\bm{y}(t + h)$ の精度が $h^p$ オーダーの場合,
その公式は $p$ 次といい,$p$ は次数と呼ばれる.

\section{埋め込み型公式}

Runge-Kutta 法の公式の中には,複数の $b_i$ の組を持つものがある
(表 \ref{table:ode_runge-kutta_butcher-array-rkf45} の例を参照).
そのような公式では,次のような 2 つの近似値を得ることができる.
\begin{align}
    \bm{y}(t + h)   & = \bm{y}(t) + \sum_{i=1}^s b_i \bm{k}_i     \\
    \bm{y}^*(t + h) & = \bm{y}^*(t) + \sum_{i=1}^s b_i^* \bm{k}_i
\end{align}
これらの差により誤差の近似値を求めることができる.
\begin{align}
    \bm{y}(t + h) - \bm{y}^*(t + h) & = \sum_{i=1}^s (b_i - b_i^*) \bm{k}_i
\end{align}

これを用いると,現在のステップ幅から次のステップ幅 $\hat{h}$ の適正値を推定できる.
まず,$b_i$ と $b_i^*$ のうち次数が低い方の次数を $p$ とすると,
\begin{align}
    \left| \bm{y}(t + h) - \bm{y}^*(t + h) \right| \approx |Ah^p|
\end{align}
のように書ける.
そこで,誤差の許容量を $\epsilon_{tol}$ としたとき,次の方程式が成り立つようにする.
\begin{equation}
    \frac{\hat{h}^p}{h^p} = \frac{\epsilon_{tol}}{\left| \bm{y}(t + h) - \bm{y}^*(t + h) \right|}
\end{equation}
これを $\hat{h}$ について解くと,次のようになる.
\begin{equation}
    \hat{h} = h \left(\frac{\epsilon_{tol}}{\left| \bm{y}(t + h) - \bm{y}^*(t + h) \right|}\right)^{\frac{1}{p}}
\end{equation}

\section{古典的 Runge-Kutta 法}

\begin{table}[bp]
    \caption{古典的 Runge-Kutta 法 (RK4 公式)の Butcher 配列}
    \label{table:ode_runge-kutta_butcher-array-rk4}
    \centering
    \begin{tabular}{c|cccc}
        $0$           &               &               &               &               \\
        $\frac{1}{2}$ & $\frac{1}{2}$ &               &               &               \\
        $\frac{1}{2}$ & $0$           & $\frac{1}{2}$ &               &               \\
        $1$           & $0$           & $0$           & $1$           &               \\
        \hline
                      & $\frac{1}{6}$ & $\frac{1}{3}$ & $\frac{1}{3}$ & $\frac{1}{6}$
    \end{tabular}
\end{table}

古典的 Runge-Kutta 法と呼ばれる公式では,
表 \ref{table:ode_runge-kutta_butcher-array-rk4} のような係数を用いる
\cite[3.3 節]{Mitsui1993}.
単純な係数で次数 4 を達成できる.

\section{RKF45 公式}

\begin{table}[bp]
    \caption{RKF45 公式の Butcher 配列}
    \label{table:ode_runge-kutta_butcher-array-rkf45}
    \centering
    \begin{tabular}{c|ccccccc}
        $0$             &                     &                      &                      &                       &                  &                &          \\
        $\frac{1}{4}$   & $\frac{1}{4}$       &                      &                      &                       &                  &                &          \\
        $\frac{3}{8}$   & $\frac{3}{32}$      & $\frac{9}{32}$       &                      &                       &                  &                &          \\
        $\frac{12}{13}$ & $\frac{1932}{2197}$ & $-\frac{7200}{2197}$ & $\frac{7296}{2197}$  &                       &                  &                &          \\
        $1$             & $\frac{439}{216}$   & $-8$                 & $\frac{3680}{513}$   & $-\frac{845}{4104}$   &                  &                &          \\
        $\frac{1}{2}$   & $-\frac{8}{27}$     & $2$                  & $-\frac{3544}{2565}$ & $\frac{1859}{4104}$   & $-\frac{11}{40}$ &                &          \\
        \hline
                        & $\frac{16}{135}$    & $0$                  & $\frac{6656}{12825}$ & $\frac{28561}{56430}$ & $-\frac{9}{50}$  & $\frac{2}{55}$ & (5 次) \\
                        & $\frac{25}{216}$    & $0$                  & $\frac{1408}{2565}$  & $\frac{2197}{4104}$   & $-\frac{1}{5}$   & $0$            & (4 次)
    \end{tabular}
\end{table}

RKF45 公式(RKF は Runge-Kutta-Fehlberg のこと)では,
表 \ref{table:ode_runge-kutta_butcher-array-rkf45} のような係数を用いる
\cite[4.1 節 (a)]{Mitsui1993}, \cite[Section 9.5]{Mathews2004}.
この埋め込み型公式では,$\bm{k}_i$ から $\bm{y}(t + h)$ を算出する係数に
5 次の精度を持つ組と 4 次の精度を持つ組が存在する
\footnote{挙げた 2 件の文献のうち,%
    文献 \cite{Mitsui1993} では係数が一カ所誤っていたため注意が必要.}.

\section{田中公式}

\begin{table}[bp]
    \caption{田中 Formula1 公式の Butcher 配列}
    \label{table:ode_runge-kutta_butcher-array-tanaka-formula1}
    \centering
    \begin{tabular}{c|ccc}
        $\frac{13}{20}$ & $\frac{13}{20}$    &                  &          \\
        $-\frac{1}{18}$ & $-\frac{127}{180}$ & $\frac{13}{20}$  &          \\
        \hline
                        & $\frac{100}{127}$  & $\frac{27}{127}$ & (3 次) \\
                        & $1$                &                  & (1 次)
    \end{tabular}
\end{table}

\begin{table}[bp]
    \caption{田中 Formula2 公式の Butcher 配列}
    \label{table:ode_runge-kutta_butcher-array-tanaka-formula2}
    \centering
    \begin{tabular}{c|cccc}
        $\frac{133}{100}$ & $\frac{133}{100}$     &                       &                      &          \\
        $\frac{1}{2}$     & $-\frac{5400}{18167}$ & $\frac{28967}{36334}$ &                      &          \\
        $-\frac{33}{100}$ & $\frac{133}{50}$      & $-\frac{108}{25}$     & $\frac{133}{100}$    &          \\
        \hline
                          & $\frac{1250}{20667}$  & $\frac{18167}{20667}$ & $\frac{1250}{20667}$ & (4 次) \\
                          & $0$                   & $1$                   &                      & (2 次)
    \end{tabular}
\end{table}

文献 \cite{Togawa2007} では,田中正次氏が考案した半陰的・陰的公式がいくつか示されている.
そのうち埋め込み型の半陰的公式を表
\ref{table:ode_runge-kutta_butcher-array-tanaka-formula1},
\ref{table:ode_runge-kutta_butcher-array-tanaka-formula2}
に示す
\footnote{文献 \cite{Togawa2007} には完全に陰的な 4 段 7 次の公式もあるが,%
    係数がかなり複雑なため省略した.}.

段数の多い陽的公式と同程度の精度を少ない段数で出すことができていることを確認できる.
また,埋め込み型の 2 つの係数のうち次数の低い方の係数が単純になっている
\footnote{利用する際には高い次数の係数との差を見るように実装するため,%
    次数の低い方の係数だけ単純でも計算コストへの影響はない.}.

\section{Butcher-Kuntzmann 公式}

\begin{table}[bp]
    \caption{2 段 Butcher-Kuntzmann 公式の Butcher 配列}
    \label{table:ode_runge-kutta_butcher-array-2stage-butcher-kuntzmann}
    \centering
    \begin{tabular}{c|cc}
        $\frac{1}{2} + \frac{\sqrt{3}}{6}$ & $\frac{1}{4}$                      & $\frac{1}{4} + \frac{\sqrt{3}}{6}$ \\
        $\frac{1}{2} - \frac{\sqrt{3}}{6}$ & $\frac{1}{4} - \frac{\sqrt{3}}{6}$ & $\frac{1}{4}$                      \\
        \hline
                                           & $\frac{1}{2}$                      & $\frac{1}{2}$
    \end{tabular}
\end{table}

文献 \cite[5.2 節 (b)]{Mitsui1993} によると,
陰的 Runge-Kutta 法では,$s$ 段公式で $2s$ 次を超えることができないと証明されており,
その限界の $2s$ 次の公式が存在することも証明されている.
特に Legendre 関数の零点を用いて係数を決める種類の公式は,
$s$ 段 Butcher-Kuntzmann 公式と呼ばれる.
2 段 Butcher-Kuntzmann 公式を表
\ref{table:ode_runge-kutta_butcher-array-2stage-butcher-kuntzmann}
に示す(前述の通り 4 次の精度を持つ).

\section{Euler 法}

\begin{table}[bp]
    \caption{Euler 法の Butcher 配列}
    \label{table:ode_runge-kutta_butcher-array-explicit-euler}
    \centering
    \begin{tabular}{c|c}
        $1$ &     \\
        \hline
            & $1$
    \end{tabular}
\end{table}

常微分方程式の解法としては最も基本的な Euler 法
\begin{equation}
    \bm{y}(t + h) \approx \bm{y}(t) + h \bm{f}(t, \bm{y}(t))
\end{equation}
は形式的に Runge-Kutta 法とみなすことができる.
Butcher 配列は表 \ref{table:ode_runge-kutta_butcher-array-explicit-euler} に示す通りである.

% !TEX root = ../main.tex
%

\chapter{平均ベクトル場法}

エネルギー保存則を満たす系の運動方程式を解くことを目的とした手法の 1 つとして,
平均ベクトル場法(Average vector field method, AVF method) \cite{Quispel2008} がある.

座標 $\bm{q}$ に対して,
運動エネルギーが $T = \bm{p}^\top \dot{\bm{q}} / 2$ で表現されるようにモーメント $\bm{p}$ を定める.
さらに位置エネルギー $V$ を座標 $\bm{q}$ で表現することで,
全エネルギーの和を表現した関数 $H(\bm{q}, \bm{p}) = T + V$ を
ハミルトニアン(Hamiltonian)と呼ぶ
\cite[Section 3.2]{Morse1953}.
このとき,運動方程式は次の式で表現される.
\begin{align}
    \dot{\bm{q}} & = \frac{\partial H}{\partial \bm{p}}  \\
    \dot{\bm{p}} & = -\frac{\partial H}{\partial \bm{q}}
\end{align}
この式は変数 $\bm{y} = (\bm{q}^\top, \bm{p}^\top)^\top$ を用いて次のように書くこともできる.
\begin{equation}
    \dot{\bm{y}} = \bm{f}(\bm{y}) \equiv S \nabla H(\bm{y})
\end{equation}
文献 \cite{Quispel2008} はこの形式の微分方程式を前提として
次のような数値解法を提案している.
\begin{equation}
    \frac{\bm{y}_{n+1} - \bm{y}_n}{h}
    = \int_{0}^{1} \bm{f}((1 - \xi) \bm{y}_n + \xi \bm{y}_{n+1}) d \xi
    \label{eq:ode_average-vector-field_update-2-order}
\end{equation}

この手法がエネルギーを保存することを示す.
まず,$\bm{f}(\bm{y}) \equiv S \nabla H(\bm{y})$ より
\begin{equation}
    \frac{\bm{y}_{n+1} - \bm{y}_n}{h}
    = S \int_{0}^{1} \nabla H((1 - \xi) \bm{y}_n + \xi \bm{y}_{n+1}) d \xi
    \label{eq:ode_average-vector-field_update-with-hamiltonian}
\end{equation}
である.ここで,$S$ は
\begin{equation}
    S =
    \begin{pmatrix}
        O  & I \\
        -I & O
    \end{pmatrix}
\end{equation}
であるため
\footnote{このような行列は歪対称行列と呼ばれる.}
,任意のベクトル $\bm{y} = (\bm{q}^\top, \bm{p}^\top)^\top$ において,
\begin{equation}
    \bm{y}^\top S \bm{y} =
    \begin{pmatrix}
        \bm{q}^\top & \bm{p}^\top
    \end{pmatrix}
    \begin{pmatrix}
        O  & I \\
        -I & O
    \end{pmatrix}
    \begin{pmatrix}
        \bm{q} \\ \bm{p}
    \end{pmatrix}
    =
    \begin{pmatrix}
        \bm{q}^\top & \bm{p}^\top
    \end{pmatrix}
    \begin{pmatrix}
        \bm{p} \\ -\bm{q}
    \end{pmatrix}
    = \bm{0}
\end{equation}
となる.よって,
式 \eqref{eq:ode_average-vector-field_update-with-hamiltonian}
の両辺と
$\int_{0}^{1} \nabla H((1 - \xi) \bm{y}_n + \xi \bm{y}_{n+1}) d \xi$ の内積をとると
\begin{equation}
    \frac{\bm{y}_{n+1} - \bm{y}_n}{h}
    \int_{0}^{1} \nabla H((1 - \xi) \bm{y}_n + \xi \bm{y}_{n+1}) d \xi
    = 0
\end{equation}
となる.さらに左辺は
\begin{align}
      & \frac{\bm{y}_{n+1} - \bm{y}_n}{h}
    \int_{0}^{1} \nabla H((1 - \xi) \bm{y}_n + \xi \bm{y}_{n+1}) d \xi         \\
    = & \frac{1}{h}
    \int_{0}^{1} \frac{d}{d\xi} H((1 - \xi) \bm{y}_n + \xi \bm{y}_{n+1}) d \xi \\
    = & \frac{H(\bm{y}_{n+1}) - H()\bm{y}_{n})}{h}
\end{align}
と変形できるから,次のようにエネルギーの保存が示される.
\begin{equation}
    H(\bm{y}_{n+1}) - H(\bm{y}_{n}) = 0
\end{equation}

式 \eqref{eq:ode_average-vector-field_update-2-order} は 2 次の精度だが,
3 次と 4 次の公式も文献 \cite{Quispel2008} で示されている.
それらは次の式で与えられる.
\begin{equation}
    \frac{\bm{y}_{n+1} - \bm{y}_n}{h}
    = \left(I + \alpha h^2
    \left(\left. \frac{\partial \bm{f}}{\partial \bm{y}} \right|_{\bm{y} = \hat{\bm{y}}}\right)^2
    \right)
    \int_{0}^{1} \bm{f}((1 - \xi) \bm{y}_n + \xi \bm{y}_{n+1}) d \xi
\end{equation}
パラメータと次数は次のようになる.
\begin{itemize}
    \item $\alpha = 0$ とすると 2 次精度の更新式
          \eqref{eq:ode_average-vector-field_update-2-order}
          が得られる.
    \item $\alpha = -1/12$, $\hat{\bm{y}} = \bm{y}_n$ とすると
          3 次精度の更新式が得られる.
    \item $\alpha = -1/12$, $\hat{\bm{y}} = (\bm{y}_n + \bm{y}_{n+1})/2$ とすると
          4 次精度の更新式が得られる.
\end{itemize}

この手法の更新式は次数に依らず陰的で,
次数が上がるごとに計算が複雑になっていく.
ここで,次数 2 のときの方程式
\begin{equation}
    F(\bm{y}_{n+1}) \equiv
    \frac{\bm{y}_{n+1} - \bm{y}_n}{h} -
    \int_{0}^{1} \bm{f}((1 - \xi) \bm{y}_n + \xi \bm{y}_{n+1}) d \xi
    = 0
\end{equation}
を考える.
これを Newton-Raphson 法(\ref{chap:root-finding_newton-raphson} 章)で解くには,
Jacobian が必要となる.
Jacobian は次のようになる.
\begin{equation}
    \frac{\partial F}{\partial \bm{y}_{n+1}} =
    \frac{1}{h} I -
    \int_{0}^{1} \xi \left. \frac{\partial \bm{f}}{\partial \bm{y}} \right|_{\bm{y} = (1 - \xi) \bm{y}_n + \xi \bm{y}_{n+1}} d \xi
\end{equation}
さらに,Jacobian の変化が十分小さいとすると,
\begin{align}
            & \frac{\partial F}{\partial \bm{y}_{n+1}}                                                \\
    \approx & \frac{1}{h} I -
    \int_{0}^{1} \xi \left. \frac{\partial \bm{f}}{\partial \bm{y}} \right|_{\bm{y} = \bm{y}_n} d \xi \\
    =       & \frac{1}{h} I - \frac{1}{2}
    \left. \frac{\partial \bm{f}}{\partial \bm{y}} \right|_{\bm{y} = \bm{y}_n}
    \label{eq:ode_average-vector-field_approx-jacobian}
\end{align}
のように近似できる.
3 次と 4 次の更新式でも,Newton-Raphson 法の Jacobian においては $h^2$ の項を無視して
式 \eqref{eq:ode_average-vector-field_approx-jacobian} で
近似することができる.


% !TEX root = ../main.tex
%

\part{高精度演算}

% !TEX root = ../main.tex
%

\chapter{倍精度浮動小数点数による多倍長浮動小数点数}

倍精度浮動小数点数を用いて四倍精度,八倍精度といったより高い精度の演算を行う手法が考えられている
\cite{Hirayama2014,Hida2001}.
それぞれ 2 つ,4 つの倍精度浮動小数点数の和で小数を表現し,
仮数の桁が被らないように演算することで高い精度を実現する.

\section{記号}

本章で使用する記号を以下に示す.

\begin{explainlist}
    $\oplus$ & 倍精度浮動小数点数の加算 \\
    $\ominus$ & 倍精度浮動小数点数の減算 \\
    $\otimes$ & 倍精度浮動小数点数の乗算 \\
    $\oslash$ & 倍精度浮動小数点数の除算 \\
\end{explainlist}

\section{基本演算}

倍精度浮動小数点数による四倍精度,八倍精度演算の既存文献\cite{Hida2001}で使用される
基本的な演算を以下にまとめる.

まず,$|a| \ge |b|$ となる倍精度浮動小数点数 $a$, $b$ について
$s = a \oplus b$, $a + b = s + e$ が成り立つ倍精度浮動小数点数 $s$, $e$ を
算出するアルゴリズムを Algorithm \ref{more-precision_multi-double_algo_QuickTwoSum} に示す.
なお,$a$, $b$ の大小が不明な場合は
Algorithm \ref{more-precision_multi-double_algo_TwoSum} を使用する.

\begin{algorithm}[tp]
    \caption{大小の明確な倍精度浮動小数点数の加算と誤差計算\cite[Algorithm 3]{Hida2001}}
    \label{more-precision_multi-double_algo_QuickTwoSum}
    \begin{algorithmic}[1]
        \Procedure{QuickTwoSum}{$a, b$}
        \State $s \gets a \oplus b$
        \State $e \gets b \ominus (s \ominus a)$
        \State \Return $(s, e)$
        \EndProcedure
    \end{algorithmic}
\end{algorithm}

\begin{algorithm}[tp]
    \caption{大小の不明な倍精度浮動小数点数の加算と誤差計算\cite[Algorithm 4]{Hida2001}}
    \label{more-precision_multi-double_algo_TwoSum}
    \begin{algorithmic}[1]
        \Procedure{TwoSum}{$a, b$}
        \State $s \gets a \oplus b$
        \State $v \gets s \ominus a$
        \State $e \gets (a \ominus (s \ominus v)) \oplus (b \ominus v)$
        \State \Return $(s, e)$
        \EndProcedure
    \end{algorithmic}
\end{algorithm}

また,乗算についても
$s = a \otimes b$, $a \times b = s + e$ となる
倍精度浮動小数点数 $s$, $e$ を算出する
Algorithm \ref{more-precision_multi-double_algo_TwoProd} が存在する.
ただし,fused multiply-add (FMA) 命令が存在する CPU では,
Algorithm \ref{more-precision_multi-double_algo_TwoProdFMA} により高速化が期待できる.

\begin{algorithm}[tp]
    \caption{倍精度浮動小数点数の乗算と誤差計算\cite[Algorithm 5, 6]{Hida2001}}
    \label{more-precision_multi-double_algo_TwoProd}
    \begin{algorithmic}[1]
        \Procedure{TwoProd}{$a, b$}
        \State $p \gets a \otimes b$
        \State $(a_h, a_l) \gets \text{Split}(a)$
        \State $(b_h, b_l) \gets \text{Split}(b)$
        \State $e \gets ((a_h \otimes b_h \ominus p) \oplus a_h \otimes b_l \oplus a_l \otimes b_h) \oplus a_l \otimes b_l$
        \State \Return $(p, e)$
        \EndProcedure
    \end{algorithmic}
    \begin{algorithmic}[1]
        \Procedure{Split}{$a$}
        \State $t \gets (2^{27} + 1) \otimes a$
        \State $a_h \gets t \ominus (t \ominus a)$
        \State $a_l \gets a \ominus a_h$
        \State \Return $(a_h, a_l)$
        \EndProcedure
    \end{algorithmic}
\end{algorithm}

\begin{algorithm}[tp]
    \caption{倍精度浮動小数点数の乗算と誤差計算(FMA 命令を使用する場合)\cite[Algorithm 7]{Hida2001}}
    \label{more-precision_multi-double_algo_TwoProdFMA}
    \begin{algorithmic}[1]
        \Procedure{TwoProdFMA}{$a, b$}
        \State $s \gets a \otimes b$
        \State $e \gets \text{FMA}(a, b, -p)$
        \Comment{$\text{FMA}(a, b, c)$ は $a \times b + c$ を $a \times b$ を途中で丸めずに計算する.}
        \State \Return $(s, e)$
        \EndProcedure
    \end{algorithmic}
\end{algorithm}

\clearpage

\section{倍精度浮動小数点数による四倍精度演算}

四倍精度浮動小数点数を $a = a_h + a_l$, $|a_l| \le (1/2) \ulp(|a_h|)$ となる
倍精度浮動小数点数 $a_h$, $a_l$ で表現する.

\subsection{四則演算}

このとき,加算は
Algorithm \ref{more-precision_multi-double_algo_QuadAddPrecisely},
\ref{more-precision_multi-double_algo_QuadAddSimple}
のようなアルゴリズムで行うことができる.
Algorithm \ref{more-precision_multi-double_algo_QuadAddSimple} の方が
簡潔な演算手法だが,
倍精度浮動小数点数の 2 倍の 104 ビットの精度で演算できることが
示されている\cite{Naoya2012}.
減算も加算と同様にして行うことができる.

\begin{algorithm}[tp]
    \caption{四倍精度の加算(正確な演算)\cite{Hisashi2006}}
    \label{more-precision_multi-double_algo_QuadAddPrecisely}
    \begin{algorithmic}[1]
        \Procedure{QuadAddPrecisely}{$(a_h, a_l), (b_h, b_l)$}
        \State $(x_h, x_l) \gets \text{TwoSum}(a_h, b_h)$
        \State $(y_h, y_l) \gets \text{TwoSum}(a_l, b_l)$
        \State $x_l \gets x_l \oplus y_h$
        \State $(x_h, x_l) \gets \text{QuickTwoSum}(x_h, x_l)$
        \State $x_l \gets x_l \oplus y_l$
        \State $(x_h, x_l) \gets \text{QuickTwoSum}(x_h, x_l)$
        \State \Return $(x_h, x_l)$
        \EndProcedure
    \end{algorithmic}
\end{algorithm}

\begin{algorithm}[tp]
    \caption{四倍精度の加算(簡潔な演算)\cite{Naoya2012,Hirayama2014}}
    \label{more-precision_multi-double_algo_QuadAddSimple}
    \begin{algorithmic}[1]
        \Procedure{QuadAddSimple}{$(a_h, a_l), (b_h, b_l)$}
        \State $(x_h, x_l) \gets \text{TwoSum}(a_h, b_h)$
        \State $x_l \gets x_l \oplus a_l \oplus b_l$
        \State $(x_h, x_l) \gets \text{QuickTwoSum}(x_h, x_l)$
        \State \Return $(x_h, x_l)$
        \EndProcedure
    \end{algorithmic}
\end{algorithm}

また,乗算は
Algorithm \ref{more-precision_multi-double_algo_QuadMultiply} のようにすることで
102 ビットの精度で算出できることが示されている\cite{Naoya2012}.

\begin{algorithm}[tp]
    \caption{四倍精度の乗算\cite{Hisashi2006,Naoya2012}}
    \label{more-precision_multi-double_algo_QuadMultiply}
    \begin{algorithmic}[1]
        \Procedure{QuadMultiply}{$(a_h, a_l), (b_h, b_l)$}
        \State $(x_h, x_l) \gets \text{TwoProd}(a_h, b_h)$
        \State $x_l \gets x_l \oplus a_h \otimes b_l \oplus a_l \otimes b_h$
        \State $(x_h, x_l) \gets \text{QuickTwoSum}(x_h, x_l)$
        \State \Return $(x_h, x_l)$
        \EndProcedure
    \end{algorithmic}
\end{algorithm}

残りの四則演算である除算は
Algorithm \ref{more-precision_multi-double_algo_QuadDivide} のようにすることで
算出できる\cite{Naoya2012s}.

\begin{algorithm}[tp]
    \caption{四倍精度の除算\cite{Naoya2012s}}
    \label{more-precision_multi-double_algo_QuadDivide}
    \begin{algorithmic}[1]
        \Procedure{QuadDivide}{$(a_h, a_l), (b_h, b_l)$}
        \State $c \gets 1 \oslash b_h$
        \State $d \gets b_l \otimes c$
        \State $x_h \gets a_h \otimes c$
        \State $(r_1, r_2) \gets \text{TwoProd}(x_h, b_h)$
        \State $x_l \gets ((a_h \ominus r_1) \ominus r_2) \otimes c$
        \State $x_l \gets x_l \oplus x_h \otimes ((a_l \oslash a_h) \ominus d)$
        \State $(x_h, x_l) \gets \text{QuickTwoSum}(x_h, x_l)$
        \State \Return $(x_h, x_l)$
        \EndProcedure
    \end{algorithmic}
\end{algorithm}



%\section{テスト}
%
%テスト
%
%\begin{equation}
%    e^{ix} = \cos{x} + i \sin{x}
%\end{equation}
%
%\begin{lstlisting}[caption=テスト, language={C++}]
%int main() {
%    return 0; // 日本語コメント
%}
%\end{lstlisting}
%
%\begin{algorithm}
%    \caption{テスト}
%    \begin{algorithmic}
%        \Procedure{Test}{$x, y$}\Comment{テスト}
%        \State $x = y$
%        \EndProcedure
%    \end{algorithmic}
%\end{algorithm}
%
%\cite{Hirayama2014}

\bibliographystyle{junsrt}
\bibliography{articles.bib}

\end{document}

