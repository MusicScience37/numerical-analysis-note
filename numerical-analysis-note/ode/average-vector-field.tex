% !TEX root = ../main.tex
%

\chapter{平均ベクトル場法}\label{chap:ode_avf}

\index{へいきんべくとるばほう@平均ベクトル場法}
\index{Average vector field method|see{平均ベクトル場法}}
エネルギー保存則を満たす系の運動方程式を解くことを目的とした手法の 1 つとして,
平均ベクトル場法(Average vector field method, AVF method) \cite{Quispel2008} がある.

\ref{chap:lagrangian-and-hamiltonian} 章で示しているように,
座標 $\bm{q} \in \setR^d$ とモーメント $\bm{p} \in \setR^d$ による
ハミルトニアン $H(\bm{q}, \bm{p})$ を用いた次の正準方程式を考える.
\index{せいじゅんほうていしき@正準方程式}
\begin{align}
    \dot{\bm{q}} & = \frac{\partial H}{\partial \bm{p}}  \\
    \dot{\bm{p}} & = -\frac{\partial H}{\partial \bm{q}}
\end{align}
この式は変数 $\bm{y} = (\bm{q}^\top, \bm{p}^\top)^\top$ を用いて次のように書くこともできる.
\begin{equation}
    \dot{\bm{y}} = \bm{f}(\bm{y}) \equiv S \nabla H(\bm{y})
\end{equation}
文献 \cite{Quispel2008} はこの形式の微分方程式を前提として
次のような数値解法を提案している.
\begin{equation}
    \frac{\bm{y}_{n+1} - \bm{y}_n}{h}
    = \int_{0}^{1} \bm{f}((1 - \xi) \bm{y}_n + \xi \bm{y}_{n+1}) d \xi
    \label{eq:ode_average-vector-field_update-2-order}
\end{equation}

この手法がエネルギーを保存することを示す.
まず,$\bm{f}(\bm{y}) \equiv S \nabla H(\bm{y})$ より
\begin{equation}
    \frac{\bm{y}_{n+1} - \bm{y}_n}{h}
    = S \int_{0}^{1} \nabla H((1 - \xi) \bm{y}_n + \xi \bm{y}_{n+1}) d \xi
    \label{eq:ode_average-vector-field_update-with-hamiltonian}
\end{equation}
である.ここで,$S$ は
\begin{equation}
    S =
    \begin{pmatrix}
        O  & I \\
        -I & O
    \end{pmatrix}
\end{equation}
であるため
\footnote{このような行列は\index{わいたいしょうぎょうれつ@歪対称行列}歪対称行列と呼ばれる.}
,任意のベクトル $\bm{y} = (\bm{q}^\top, \bm{p}^\top)^\top$ において,
\begin{equation}
    \bm{y}^\top S \bm{y} =
    \begin{pmatrix}
        \bm{q}^\top & \bm{p}^\top
    \end{pmatrix}
    \begin{pmatrix}
        O  & I \\
        -I & O
    \end{pmatrix}
    \begin{pmatrix}
        \bm{q} \\ \bm{p}
    \end{pmatrix}
    =
    \begin{pmatrix}
        \bm{q}^\top & \bm{p}^\top
    \end{pmatrix}
    \begin{pmatrix}
        \bm{p} \\ -\bm{q}
    \end{pmatrix}
    = \bm{0}
\end{equation}
となる.よって,
式 \eqref{eq:ode_average-vector-field_update-with-hamiltonian}
の両辺と
$\int_{0}^{1} \nabla H((1 - \xi) \bm{y}_n + \xi \bm{y}_{n+1}) d \xi$ の内積をとると
\begin{equation}
    \frac{\bm{y}_{n+1} - \bm{y}_n}{h}
    \int_{0}^{1} \nabla H((1 - \xi) \bm{y}_n + \xi \bm{y}_{n+1}) d \xi
    = 0
\end{equation}
となる.さらに左辺は
\begin{align}
      & \frac{\bm{y}_{n+1} - \bm{y}_n}{h}
    \int_{0}^{1} \nabla H((1 - \xi) \bm{y}_n + \xi \bm{y}_{n+1}) d \xi         \\
    = & \frac{1}{h}
    \int_{0}^{1} \frac{d}{d\xi} H((1 - \xi) \bm{y}_n + \xi \bm{y}_{n+1}) d \xi \\
    = & \frac{H(\bm{y}_{n+1}) - H()\bm{y}_{n})}{h}
\end{align}
と変形できるから,次のようにエネルギーの保存が示される.
\begin{equation}
    H(\bm{y}_{n+1}) - H(\bm{y}_{n}) = 0
\end{equation}

式 \eqref{eq:ode_average-vector-field_update-2-order} は 2 次の精度だが,
3 次と 4 次の公式も文献 \cite{Quispel2008} で示されている.
それらは次の式で与えられる.
\begin{equation}
    \frac{\bm{y}_{n+1} - \bm{y}_n}{h}
    = \left(I + \alpha h^2
    \left(\left. \frac{\partial \bm{f}}{\partial \bm{y}} \right|_{\bm{y} = \hat{\bm{y}}}\right)^2
    \right)
    \int_{0}^{1} \bm{f}((1 - \xi) \bm{y}_n + \xi \bm{y}_{n+1}) d \xi
\end{equation}
パラメータと次数は次のようになる.
\begin{itemize}
    \item $\alpha = 0$ とすると 2 次精度の更新式
          \eqref{eq:ode_average-vector-field_update-2-order}
          が得られる.
    \item $\alpha = -1/12$, $\hat{\bm{y}} = \bm{y}_n$ とすると
          3 次精度の更新式が得られる.
    \item $\alpha = -1/12$, $\hat{\bm{y}} = (\bm{y}_n + \bm{y}_{n+1})/2$ とすると
          4 次精度の更新式が得られる.
\end{itemize}

この手法の更新式は次数に依らず陰的で,
次数が上がるごとに計算が複雑になっていく.
ここで,次数 2 のときの方程式
\begin{equation}
    F(\bm{y}_{n+1}) \equiv
    \frac{\bm{y}_{n+1} - \bm{y}_n}{h} -
    \int_{0}^{1} \bm{f}((1 - \xi) \bm{y}_n + \xi \bm{y}_{n+1}) d \xi
    = 0
\end{equation}
を考える.
これを Newton-Raphson 法(\ref{chap:root-finding_newton-raphson} 章)で解くには,
Jacobian が必要となる.
Jacobian は次のようになる.
\begin{equation}
    \frac{\partial F}{\partial \bm{y}_{n+1}} =
    \frac{1}{h} I -
    \int_{0}^{1} \xi \left. \frac{\partial \bm{f}}{\partial \bm{y}} \right|_{\bm{y} = (1 - \xi) \bm{y}_n + \xi \bm{y}_{n+1}} d \xi
\end{equation}
さらに,Jacobian の変化が十分小さいとすると,
\begin{align}
            & \frac{\partial F}{\partial \bm{y}_{n+1}}                                                \\
    \approx & \frac{1}{h} I -
    \int_{0}^{1} \xi \left. \frac{\partial \bm{f}}{\partial \bm{y}} \right|_{\bm{y} = \bm{y}_n} d \xi \\
    =       & \frac{1}{h} I - \frac{1}{2}
    \left. \frac{\partial \bm{f}}{\partial \bm{y}} \right|_{\bm{y} = \bm{y}_n}
    \label{eq:ode_average-vector-field_approx-jacobian}
\end{align}
のように近似できる.
3 次と 4 次の更新式でも,Newton-Raphson 法の Jacobian においては $h^2$ の項を無視して
式 \eqref{eq:ode_average-vector-field_approx-jacobian} で
近似することができる.
