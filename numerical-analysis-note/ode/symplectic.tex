% !TEX root = ../main.tex
%

\chapter{シンプレクティック積分法}\label{sec:ode_symplectic}

\index{しんぷれくてぃっくせきぶんほう@シンプレクティック積分法}
平均ベクトル場法(\ref{chap:ode_avf} 章)と同様に,
位置 $\bm{q}$,モーメント $\bm{p}$ に対する
ハミルトニアン $H(\bm{q}, \bm{p})$ から得られる正準方程式
\index{せいじゅんほうていしき@正準方程式}
\begin{align}
    \dot{\bm{q}} & = \frac{\partial H}{\partial \bm{p}}  \\
    \dot{\bm{p}} & = -\frac{\partial H}{\partial \bm{q}}
\end{align}
を解くための手法の 1 つに
シンプレクティック積分法 (symplectic integrator) と呼ばれる手法がある
\cite[Section II.16.]{Hairer1993}.
シンプレクティック積分法は,
時間変化に伴う点 $(\bm{q}, \bm{p})$ の変化により
\begin{equation}
    \omega^2 = \sum_{i=1}^{n} dp_i \wedge dq_i
\end{equation}
という量が変化しないという特性を持つ.
そのため,周期運動をする自励系において長時間に渡るシミュレーションを行っても
ステップ幅を固定していればエネルギーの誤差が増大しないという利点がある
\cite{Yoshida2013}
\footnote{ステップ幅を固定しない場合はエネルギーが増大し,%
    解が不安定になることが実験で確認されている%
    \cite[Section II.16.]{Hairer1993}.}.

ここでは,シンプレクティックな Runge-Kutta 法を示す.

\section{陰的 Runge-Kutta 法}

陰的 Runge-Kutta 法の中にはシンプレクティックとなっているものが存在する.
例えば,表 \ref{table:ode_runge-kutta_butcher-array-2stage-butcher-kuntzmann} のような
$s$ 段 Butcher-Kuntzmann 公式はシンプレクティックである
\cite[Section II.16.]{Hairer1993}.

\section{可分ハミルトニアンに対する陽的な分離型 Runge-Kutta 法}

通常の陽的 Runge-Kutta 法はシンプレクティックになり得ないことが証明されているが,
ハミルトニアンが可分
($H(\bm{q}) = T(\bm{p}) + V(\bm{q})$ のように
$\bm{q}$ の項と $\bm{p}$ の項に分割可能)な場合,
$\bm{q}$ と $\bm{p}$ を交互に更新する以下のような分離型 Runge-Kutta 法を考えることにより,
陽的でシンプレクティックな積分公式を得ることができる
\cite[Section II.16.]{Hairer1993}.

\begin{align}
    \bm{P}_i    & = \bm{p}(t) + h \sum_{j} a_{ij} \bm{k}_j,                                                                    &
    \bm{Q}_i    & = \bm{q}(t) + h \sum_{j} \hat{a}_{ij} \bm{l}_j                                                                 \\
    \bm{p}(t+h) & = \bm{p}(t) + h \sum_{j} b_{j} \bm{k}_j,                                                                     &
    \bm{q}(t+h) & = \bm{q}(t) + h \sum_{j} \hat{b}_{j} \bm{l}_j,                                                                 \\
    \bm{k}_i    & = -\frac{\partial H}{\partial \bm{q}} (\bm{P}_i, \bm{Q}_i) = -\frac{\partial V}{\partial \bm{q}} (\bm{Q}_i), &
    \bm{l}_i    & = \frac{\partial H}{\partial \bm{p}} (\bm{P}_i, \bm{Q}_i) = \frac{\partial T}{\partial \bm{p}} (\bm{P}_i)
\end{align}

ここで,
$a_{ij} = 0$ ($i < j$), $\hat{a}_{ij} = 0$ ($i \le j$)
とすると,
シンプレクティックであるための条件は
$a_{ij} = b_j$ ($i \le j$), $\hat{a}_{ij} = \hat{b}_j$ ($i > j$)
であることが示されており \cite[Section II.16.]{Hairer1993},
$\bm{P}_i$ と $\bm{Q}_i$ は次のようになる.

\begin{align}
    \bm{P}_i & = \bm{P}_{i-1} + h b_i \bm{k}_j = \bm{P}_{i-1} - h b_i \frac{\partial V}{\partial \bm{q}} (\bm{Q}_i),                                 &
    \bm{P}_0 & = \bm{p}(t)                                                                                                                             \\
    \bm{Q}_i & = \bm{Q}_{i - 1} + h \hat{b}_{i-1} \bm{l}_{i-1} = \bm{Q}_{i - 1} + h \hat{b}_{i-1} \frac{\partial T}{\partial \bm{p}} (\bm{P}_{i-1}), &
    \bm{Q}_1 & = \bm{q}(t)
\end{align}

このとき,$\bm{P}_i$ と $\bm{Q}_i$ は以下のように交互に計算できる.

\begin{enumerate}
    \item $\bm{P}_0$ と $\bm{Q}_1$ から $\bm{P}_1$ を算出する.
    \item $\bm{P}_1$ と $\bm{Q}_1$ から $\bm{Q}_2$ を算出する.
    \item $\bm{P}_1$ と $\bm{Q}_2$ から $\bm{P}_2$ を算出する.
    \item $\bm{P}_2$ と $\bm{Q}_2$ から $\bm{Q}_3$ を算出する.
    \item 以降同様に続ける.
\end{enumerate}

これらをまとめると,Algorithm \ref{alg:ode_symplectic_explicit-runge-kutta} となる.

\begin{algorithm}[tp]
    \caption{可分ハミルトニアンに対する陽的な分離型 Runge-Kutta 法 \cite[Section II.16.]{Hairer1993}}
    \label{alg:ode_symplectic_explicit-runge-kutta}
    \begin{algorithmic}[1]
        \Procedure{SynplecticExplicitRungeKutta}{$\frac{\partial V}{\partial \bm{q}}, \frac{\partial T}{\partial \bm{p}}, t, \bm{q}(t), \bm{p}(t), h$}
        \State $\bm{Q} \gets \bm{q}(t)$ \Comment{$\bm{Q}_1$}
        \State $\bm{P} \gets \bm{p}(t)$ \Comment{$\bm{P}_0$}
        \For{$i \gets 1, s$} \Comment{$s$ は段数}
        \State $\bm{P} \gets \bm{P} - h b_i \frac{\partial V}{\partial \bm{q}} (\bm{Q})$
        \Comment{$\bm{P}_{i-1}$ と $\bm{Q}_i$ から $\bm{P}_i$ を算出}
        \State $\bm{Q} \gets \bm{Q} + h \hat{b}_i \frac{\partial T}{\partial \bm{p}} (\bm{P})$
        \Comment{$\bm{P}_{i}$ と $\bm{Q}_i$ から $\bm{Q}_{i+1}$ を算出}
        \EndFor
        \State $q(t + h) \gets \bm{Q}$
        \State $p(t + h) \gets \bm{P}$
        \State \Return $(q(t + h), p(t + h))$
        \EndProcedure
    \end{algorithmic}
\end{algorithm}

この手法における係数 $b_i$, $\hat{b}_i$ を
表 \ref{table:ode_symplectic_explicit-runge-kutta-notation} のようにまとめたものが,
表
\ref{table:ode_symplectic_explicit-runge-kutta-1},
\ref{table:ode_symplectic_explicit-runge-kutta-2},
\ref{table:ode_symplectic_explicit-runge-kutta-3},
\ref{table:ode_symplectic_explicit-runge-kutta-4a},
\ref{table:ode_symplectic_explicit-runge-kutta-4b}
である.

\begin{table}[bp]
    \caption{陽的シンプレクティック積分法の係数の表記}
    \label{table:ode_symplectic_explicit-runge-kutta-notation}
    \centering
    \begin{tabular}{c|cccc}
        $b$       & $b_1$       & $b_2$       & $\ldots$ & $b_s$       \\
        $\hat{b}$ & $\hat{b}_1$ & $\hat{b}_2$ & $\ldots$ & $\hat{b}_s$
    \end{tabular}
\end{table}

\begin{table}[p]
    \caption{1 段 1 次の陽的シンプレクティック積分法 \cite[Section II.16.]{Hairer1993}}
    \label{table:ode_symplectic_explicit-runge-kutta-1}
    \centering
    \begin{tabular}{c|c}
        $b$       & $1$ \\
        $\hat{b}$ & $1$
    \end{tabular}
\end{table}

\begin{table}[p]
    \caption{2 段 2 次の陽的シンプレクティック積分法(Leap-frog 法) \cite{Forest1990}}
    \label{table:ode_symplectic_explicit-runge-kutta-2}
    \centering
    \begin{tabular}{c|cc}
        $b$       & $\frac{1}{2}$ & $\frac{1}{2}$ \\
        $\hat{b}$ & $1$           & $0$
    \end{tabular}
\end{table}

\begin{table}[p]
    \caption{3 段 3 次の陽的シンプレクティック積分法 \cite[Section II.16.]{Hairer1993}}
    \label{table:ode_symplectic_explicit-runge-kutta-3}
    \centering
    \begin{tabular}{c|ccc}
        $b$       & $\frac{7}{24}$ & $\frac{3}{4}$  & $-\frac{1}{24}$ \\
        $\hat{b}$ & $\frac{2}{3}$  & $-\frac{2}{3}$ & $-1$
    \end{tabular}
\end{table}

\begin{table}[p]
    \caption{6 段 4 次の陽的シンプレクティック積分法 \cite[Section II.16.]{Hairer1993}}
    \label{table:ode_symplectic_explicit-runge-kutta-4a}
    \centering
    \begin{tabular}{c|cccccc}
        $b$       & $\frac{7}{48}$ & $\frac{3}{8}$  & $-\frac{1}{48}$ & $-\frac{1}{48}$ & $\frac{3}{8}$ & $\frac{7}{48}$ \\
        $\hat{b}$ & $\frac{1}{3}$  & $-\frac{1}{3}$ & $1$             & $-\frac{1}{3}$  & $\frac{1}{3}$ & $0$
    \end{tabular}
\end{table}

\begin{table}[p]
    \caption{4 段 4 次の陽的シンプレクティック積分法($\alpha = 1 - 2^{1/3}$) \cite{Forest1990}}
    \label{table:ode_symplectic_explicit-runge-kutta-4b}
    \centering
    \begin{tabular}{c|cccc}
        $b$       & $\frac{1}{2(1+\alpha)}$ & $\frac{\alpha}{2(1+\alpha)}$ & $\frac{\alpha}{2(1+\alpha)}$ & $\frac{1}{2(1+\alpha)}$ \\
        $\hat{b}$ & $\frac{1}{1+\alpha}$    & $\frac{\alpha-1}{1+\alpha}$  & $\frac{1}{1+\alpha}$         & $0$
    \end{tabular}
\end{table}

ここで,4 次の公式のうち
表\ref{table:ode_symplectic_explicit-runge-kutta-4a}
は
表\ref{table:ode_symplectic_explicit-runge-kutta-3}
の公式を 2 つ結合して得られたものであるのに対し,
表\ref{table:ode_symplectic_explicit-runge-kutta-4b}
は代数的に 4 次の公式を算出したものとなっている.
同様に代数的に得られた 4 段 4 次の公式が
文献 \cite{Candy1991} に示されている.
