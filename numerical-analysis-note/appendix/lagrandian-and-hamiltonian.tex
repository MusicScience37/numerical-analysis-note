% !TEX root = ../main.tex
%

\chapter{Lagrangian と Hamiltonian}\label{chap:lagrangian-and-hamiltonian}

本章では
\index{ラグランジアン}
\index{Lagrangian|see{ラグランジアン}}
ラグランジアン(Lagrangian)
と
\index{ハミルトニアン}
\index{Hamiltonian|see{ハミルトニアン}}
ハミルトニアン(Hamiltonian)
について,
\cite[Chapter 3]{Morse1953}
本書の説明に必要な最低限の理論をまとめておく.

座標 $\bm{q} = (q_1, q_2, \ldots, q_d)^\top \in \setR^d$ と
その速度 $\dot{\bm{q}}$ を用いて
位置エネルギーを以下のように表す.
\begin{equation}
    T(\bm{q}, \dot{\bm{q}}) = \frac{1}{2} \sum_{i=1}^d \sum_{j=1}^d a_{ij}(\bm{q}) \dot{q}_i \dot{q}_j
\end{equation}
これをもとにモーメント \index{モーメント} $\bm{p}$ を以下のように定める.
\begin{equation}
    \bm{p} = \frac{\partial T}{\partial \dot{\bm{q}}}
\end{equation}
さらに,位置エネルギー $V$ を $\bm{q}$ の式で表しておき,
その勾配 $\partial V / \partial q_i$ を求める.
このとき,運動方程式は以下のように書ける.
(ラグランジュの運動方程式)\index{らぐらんじゅのうんどうほうていしき@ラグランジュの運動方程式}
\begin{equation}
    \dot{p}_i = \frac{\partial T}{\partial q_i} - \frac{\partial V}{\partial q_i}
    \label{eq:lagrangian-and-hamiltonian_lagrange-equation}
\end{equation}
ラグランジアン $L(\bm{q}, \dot{\bm{q}}) = T - V$ を定義すると,以下のようにも書ける.
\begin{equation}
    \frac{d}{dt} \frac{\partial L}{\partial \dot{q}_i}
    = \frac{\partial L}{\partial q_i}
\end{equation}
また,全エネルギーの和を $\bm{q}$ と $\bm{p}$ で表現した
関数 $H(\bm{q}, \bm{p}) = T + V$ をハミルトニアンと呼ぶ.
ハミルトニアンを用いると,運動方程式は次の式で表現される.
(正準方程式)\index{せいじゅんほうていしき@正準方程式}
\begin{align}
    \dot{\bm{q}} & = \frac{\partial H}{\partial \bm{p}}  \\
    \dot{\bm{p}} & = -\frac{\partial H}{\partial \bm{q}}
\end{align}
