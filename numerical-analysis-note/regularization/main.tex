% !TEX root = ../main.tex
%

\part{正則化}\label{part:regularization}

\chapter{導入}

\index{せいそくか@正則化}
この章では,正則化のアルゴリズムをまとめる.

未知のパラメータ $\bm{x}$ に合わせて変動するデータ $\bm{y} = A \bm{x}$ をもとに
パラメータ $\bm{x}$ を推定する問題は一般に \index{ぎゃくもんだい@逆問題} 逆問題と呼ばれる.
ここで,$\bm{x}$, $\bm{y}$ はそれぞれ何らかのノルム空間 $X$, $Y$ に存在するベクトルで,
$A$ は何らかの作用素とする.
例えば,録音した音源をもとに音が鳴っている場所を調べたい場合,
音が鳴っている場所が $\bm{x}$ であり,
録音した音源は $\bm{y}$ になる.

逆問題を解く単純な手法として,
ノルム $\|A \bm{x} - \bm{y}\|$ の最小化(最小二乗法)が挙げられるが,
作用素 $A$ の性質によっては,
$\bm{y}$ が少し変動するだけで最小二乗法による解 $\bm{x}$ が大きく変動してしまう場合があったり,
解が複数存在してしまう場合があったりする(\index{ill-posed problem} ill-posed problem と呼ばれる).
そのような場合に,追加の情報をもとに唯一の安定な解を求められるようにする手法を正則化と呼ぶ.
ここでは,特に次のような式を最小化する正則化手法における数値解法をまとめる.

\begin{equation}
    \|A \bm{x} - \bm{y}\|^2 + \lambda R(\bm{x})
    \label{eq:regularization_intro_general-regularization}
\end{equation}

\index{せいそくかぱらめーた@正則化パラメータ}
ここで,$\lambda \in [0, \infty)$ は正則化パラメータと呼ばれるパラメータで,
$R$ は $R : X \to [0, \infty)$ のような関数である.
この式を最小化する解 $\bm{x}_{\lambda}$ は正則化パラメータ $\lambda$ により変化する.
$\lambda = 0$ では正則化の効果がなくなり,
$\lambda$ を大きくすると正則化の効果が強くなる.
$\lambda$ を大きくしすぎると残差 $\|A \bm{x} - \bm{y}\|^2$ が大きくなり,
データ $\bm{y}$ から離れていってしまうため,
正則化パラメータ $\lambda$ を適切に調整することが必要となる.

% !TEX root = ../main.tex
%

\chapter{Tikhonov 正則化}

ここでは基本的な Tikhonov 正則化についてまとめる.
Tikhonov 正則化では,評価関数
\begin{equation}
    E_{\lambda}(\bm{x}) \equiv \|A \bm{x} - \bm{y}\|_2^2 + \lambda \|\bm{x}\|_2^2
    \label{eq:regularization_tikhonov_objective}
\end{equation}
を最小化する.
ここで,
$\bm{x} \in \setC^n$,
$\bm{y} \in \setC^m$,
$A \in \setC^{m \times n}$
である.

\section{係数行列による解}

評価関数 \eqref{eq:regularization_tikhonov_objective} を最小化する $\bm{x}$ は
係数行列 $A$ により次のように表すことができる.

\begin{theorem}\label{theorem:regularization_tikhonov_solution}
    $\lambda > 0$ の場合,
    評価関数 \eqref{eq:regularization_tikhonov_objective} が最小となるのは,
    $\bm{x} = (A^* A + \lambda I)^{-1} A^* \bm{y}$ の場合である.
\end{theorem}
\begin{proof}
    この場合,ノルムを展開することで,次のようになる.
    \begin{align}
        E_{\lambda}(\bm{x})
         & = \|A \bm{x} - \bm{y}\|_2^2 + \lambda \|\bm{x}\|_2^2
        \notag                                                                                              \\
         & = (A \bm{x} - \bm{y})^* (A \bm{x} - \bm{y}) + \lambda \bm{x}^* \bm{x}
        \notag                                                                                              \\
         & = \bm{x}^* (A^* A + \lambda I) \bm{x} - \bm{x}^* A^* \bm{y} - \bm{y}^* A \bm{x} + \|\bm{y}\|_2^2
    \end{align}
    ここで,
    $\lambda > 0$ の場合は
    $\bm{x} \neq \bm{0}$ において
    $\bm{x}^* (A^* A + \lambda I) \bm{x} = \|A \bm{x}\|_2^2 + \lambda \|\bm{x}\|_2^2 > 0$
    となることから,
    エルミート行列 $(A^* A + \lambda I)$ は正定値であり,
    正則となる.
    このことを用いると,さらに次のように展開できる.
    \begin{align}
        E_{\lambda}(\bm{x})
         & = \bm{x}^* (A^* A + \lambda I) \bm{x} - \bm{x}^* A^* \bm{y} - \bm{y}^* A \bm{x} + \|\bm{y}\|_2^2
        \notag                                                                                              \\
         & = \bm{x}^* (A^* A + \lambda I) \bm{x} - \bm{x}^* A^* \bm{y} - \bm{y}^* A \bm{x}
        + \bm{y}^* A (A^* A + \lambda I)^{-1} A^* \bm{y}
        - \bm{y}^* A (A^* A + \lambda I)^{-1} A^* \bm{y}
        + \|\bm{y}\|_2^2
        \notag                                                                                              \\
         & = (\bm{x} - (A^* A + \lambda I)^{-1} A^* \bm{y})^*
        (A^* A + \lambda I)
        (\bm{x} - (A^* A + \lambda I)^{-1} A^* \bm{y})
        - \bm{y}^* A (A^* A + \lambda I)^{-1} A^* \bm{y}
        + \|\bm{y}\|_2^2
    \end{align}
    エルミート行列 $(A^* A + \lambda I)$ が正定値であることから,
    この式が最小となるのは
    $(\bm{x} - (A^* A + \lambda I)^{-1} A^* \bm{y})$
    が零ベクトルとなる場合,
    つまり,
    $\bm{x} = (A^* A + \lambda I)^{-1} A^* \bm{y}$
    の場合である.
\end{proof}

ここで,$\lambda = 0$ でも,行列 $A$ のランクが $n$ の場合は $A^*A$ が正定値になるため同様に解が求まる.

\section{特異値分解による解法}

行列 $A$ を次のように特異値分解する.

\begin{equation}
    A = U
    \begin{pmatrix}
        \Sigma & O \\
        O      & O
    \end{pmatrix}
    V^*
\end{equation}

ここで,$U \in \setC^{m \times m}$ と $V \in \setC^{n \times n}$ はユニタリ行列で,
$\Sigma \in \setR^{r \times r}$ は正の実数による対角行列である.
ただし,ランク $r$ は $m$ または $n$ に等しくても良い.

この分解を用いると,
定理 \ref{theorem:regularization_tikhonov_solution} の解は次のように変形できる.

\begin{align}
    x_{\lambda}
     & \equiv (A^* A + \lambda I)^{-1} A^* \bm{y}
    \notag                                        \\
     & = \left(V
    \begin{pmatrix}
        \Sigma & O \\
        O      & O
    \end{pmatrix}
    U^* U
    \begin{pmatrix}
        \Sigma & O \\
        O      & O
    \end{pmatrix}
    V^* + \lambda I \right)^{-1}
    V
    \begin{pmatrix}
        \Sigma & O \\
        O      & O
    \end{pmatrix}
    U^* \bm{y}
    \notag                                        \\
     & = \left(V
    \begin{pmatrix}
        \Sigma^2 & O \\
        O        & O
    \end{pmatrix}
    V^* + \lambda V V^* \right)^{-1}
    V
    \begin{pmatrix}
        \Sigma & O \\
        O      & O
    \end{pmatrix}
    U^* \bm{y}
    \notag                                        \\
     & = \left(V
    \begin{pmatrix}
        \Sigma^2 + \lambda I & O         \\
        O                    & \lambda I
    \end{pmatrix}
    V^* \right)^{-1}
    V
    \begin{pmatrix}
        \Sigma & O \\
        O      & O
    \end{pmatrix}
    U^* \bm{y}
    \notag                                        \\
     & = V
    \begin{pmatrix}
        (\Sigma^2 + \lambda I)^{-1} & O              \\
        O                           & \lambda^{-1} I
    \end{pmatrix}
    V^*
    V
    \begin{pmatrix}
        \Sigma & O \\
        O      & O
    \end{pmatrix}
    U^* \bm{y}
    \notag                                        \\
     & = V
    \begin{pmatrix}
        (\Sigma^2 + \lambda I)^{-1} \Sigma & O \\
        O                                  & O
    \end{pmatrix}
    U^* \bm{y}
\end{align}

さらに,
$U = (\bm{u}_1, \bm{u}_2, \ldots, \bm{u}_m)$,
$V = (\bm{v}_1, \bm{v}_2, \ldots, \bm{v}_n)$,
$\Sigma = \diag(\sigma_1, \sigma_2, \ldots, \sigma_r)$
とすると,次のように表記できる.

\begin{align}
    x_{\lambda}
     & = \sum_{i = 1}^{r} \frac{\sigma_i}{\sigma_i^2 + \lambda} (\bm{u}_i^* \bm{y}) \bm{v}_i
    \label{eq:regularization_tikhonov_solution-by-svd}
\end{align}

特異値分解を一回行い
$\bm{u}_i^* \bm{y}$
を計算しておけば,
ある正則化パラメータ $\lambda$ に対する解 $\bm{x}_{\lambda}$ は単純な線形和で求めることができる.

また,
$U_1 \equiv (\bm{u}_1, \bm{u}_2, \ldots, \bm{u}_r)$,
$U_2 \equiv (\bm{u}_{r+1}, \bm{u}_{r+2}, \ldots, \bm{u}_m)$
と定義した場合,$U$ がユニタリ行列であることから
$U_1^* U_1 = I$,
$U_1^* U_2 = O$,
$U_2^* U_1 = O$,
$U_2^* U_2 = I$,
$U_1 U_1^* + U_2 U_2^* = I$
が成り立つことを利用すると,
正則化パラメータを評価する際にしばしば利用される評価関数内のノルムは次のように計算できる
\footnote{行列 $U$ のうち $U_1$ の部分だけを求めた方が計算コストを抑えられるため,%
    $U_2$ はなるべく使用しない計算式を求めている.}.

\begin{align}
    \|A \bm{x}_{\lambda} - \bm{y}\|_2^2
     & = \left\| U
    \begin{pmatrix}
        \Sigma & O \\
        O      & O
    \end{pmatrix}
    V^* V
    \begin{pmatrix}
        (\Sigma^2 + \lambda I)^{-1} \Sigma & O \\
        O                                  & O
    \end{pmatrix}
    U^* \bm{y} - \bm{y} \right\|_2^2
    \notag                                                                                        \\
     & = \left\|
    \begin{pmatrix}
        U_1 & U_2
    \end{pmatrix}
    \begin{pmatrix}
        \Sigma (\Sigma^2 + \lambda I)^{-1} \Sigma & O \\
        O                                         & O
    \end{pmatrix}
    \begin{pmatrix}
        U_1^* \\ U_2^*
    \end{pmatrix}
    \bm{y} - \bm{y} \right\|_2^2
    \notag                                                                                        \\
     & = \left\| U_1 \Sigma (\Sigma^2 + \lambda I)^{-1} \Sigma U_1^* \bm{y} - \bm{y} \right\|_2^2
    \notag                                                                                        \\
     & = \left\| U_1 \Sigma (\Sigma^2 + \lambda I)^{-1} \Sigma U_1^* \bm{y}
    - U_1 U_1^* \bm{y} - U_2 U_2^* \bm{y} \right\|_2^2
    \notag                                                                                        \\
     & = \left\| U_1 (\Sigma (\Sigma^2 + \lambda I)^{-1} \Sigma - I) U_1^* \bm{y}
    - U_2 U_2^* \bm{y} \right\|_2^2
    \notag                                                                                        \\
     & = \left\| U_1 (\Sigma (\Sigma^2 + \lambda I)^{-1} \Sigma - I) U_1^* \bm{y} \right\|_2^2
    + \left\| U_2 U_2^* \bm{y} \right\|_2^2
    \notag                                                                                        \\
     & = \left\| (\Sigma (\Sigma^2 + \lambda I)^{-1} \Sigma - I) U_1^* \bm{y} \right\|_2^2
    + \left\| U_2 U_2^* \bm{y} \right\|_2^2
    \notag                                                                                        \\
     & = \sum_{i = 1}^r \left(\frac{\lambda}{\sigma_i^2 + \lambda}\right)^2 (\bm{u}_i^* \bm{y})^2
    + \left\| (I - U_1 U_1^*) \bm{y} \right\|_2^2
\end{align}

\begin{align}
    \|\bm{x}_{\lambda}\|_2^2
     & = \left\| V
    \begin{pmatrix}
        (\Sigma^2 + \lambda I)^{-1} \Sigma & O \\
        O                                  & O
    \end{pmatrix}
    U^* \bm{y} \right\|_2^2
    \notag                                                                                          \\
     & = \left\|
    \begin{pmatrix}
        (\Sigma^2 + \lambda I)^{-1} \Sigma & O \\
        O                                  & O
    \end{pmatrix}
    U^* \bm{y} \right\|_2^2
    \notag                                                                                          \\
     & = \sum_{i = 1}^r \left(\frac{\sigma_i}{\sigma_i^2 + \lambda}\right)^2  (\bm{u}_i^* \bm{y})^2
\end{align}

どちらも正則化パラメータごとに異なる部分は $O(r)$ オーダーで計算できる.

特異値分解による Tikhonov 正則化では,
正則化パラメータを変更した際の再計算が容易なため,
多数の正則化パラメータを試したい場合に便利である.

% !TEX root = ../main.tex
%

\chapter{一般化 Tikhonov 正則化}

前章の Tikhonov 正則化では,
正則化項が $\|\bm{x}\|_2^2$ となっていたが,
正則化項として係数行列を追加した $\|L\bm{x}\|_2^2$ を用いることも考えられる.
例えば,変数 $\bm{x}$ の隣り合う要素の差をとるように係数行列 $L$ を決めることにより,
解が滑らかになるような正則化を行うことができる.
このような一般化した Tikhonov 正則化では,評価関数
\begin{equation}
    E_{\lambda}(\bm{x}) \equiv \|A \bm{x} - \bm{y}\|_2^2 + \lambda \|L \bm{x}\|_2^2
    \label{eq:regularization_gen-tikhonov_objective}
\end{equation}
を最小化する.
ここで,
$\bm{x} \in \setC^n$,
$\bm{y} \in \setC^m$,
$A \in \setC^{m \times n}$,
$L \in \setC^{p \times n}$
である.

ここで,正則化項の変更により注意することがある.
Tikhonov 正則化では,$\lambda > 0$ であれば解が唯一となっていた
(定理 \ref{theorem:regularization_tikhonov_solution}).
しかし,一般化 Tikhonov 正則化においては,
$\lambda > 0$ であるからといって解が唯一になるとは限らない.

\begin{theorem}
    $A$ と $L$ の核空間の共通部分が $\bm{0}$ 以外の要素を持つ場合,
    $\lambda > 0$ であったとしても
    評価関数 \eqref{eq:regularization_gen-tikhonov_objective} が最小となる $\bm{x}$ が
    唯一に定まらない.
\end{theorem}
\begin{proof}
    評価関数 \eqref{eq:regularization_gen-tikhonov_objective} が最小となる
    $\bm{x}$ の 1 つを $\bar{\bm{x}}$ とする.
    また,$A$ と $L$ の核空間の共通部分が $\bm{0}$ 以外に持つ要素の 1 つを $\bm{x}_0$ とする.
    このとき,
    \begin{align}
        E_{\lambda}(\bar{\bm{x}} + \bm{x}_0)
         & = \|A (\bar{\bm{x}} + \bm{x}_0) - \bm{y}\|_2^2 + \lambda \|L (\bar{\bm{x}} + \bm{x}_0)\|_2^2
        \notag                                                                                          \\
         & = \|A \bar{\bm{x}} - \bm{y}\|_2^2 + \lambda \|L \bar{\bm{x}}\|_2^2
        \notag                                                                                          \\
         & = E(\bar{\bm{x}})
    \end{align}
    となる.
    よって,評価関数 $E(\bm{x})$ が最小となる $\bm{x}$ は唯一に定まらない.
\end{proof}

一方,$A$ と $L$ の核空間の共通部分が $\bm{0}$ のみであれば
$\lambda > 0$ のときに解が唯一となる.

\begin{theorem}\label{theorem:regularization_gen-tikhonov_solution}
    $A$ と $L$ の核空間の共通部分が $\bm{0}$ のみでかつ,
    $\lambda > 0$ の場合,
    評価関数 \eqref{eq:regularization_gen-tikhonov_objective} が最小となるのは,
    $\bm{x} = (A^* A + \lambda L^* L)^{-1} A^* \bm{y}$ の場合である.
\end{theorem}
\begin{proof}
    この場合,ノルムを展開することで,次のようになる.
    \begin{align}
        E_{\lambda}(\bm{x})
         & = \|A \bm{x} - \bm{y}\|_2^2 + \lambda \|L \bm{x}\|_2^2
        \notag                                                                                                  \\
         & = (A \bm{x} - \bm{y})^* (A \bm{x} - \bm{y}) + \lambda \bm{x}^* L^* L \bm{x}
        \notag                                                                                                  \\
         & = \bm{x}^* (A^* A + \lambda L^* L) \bm{x} - \bm{x}^* A^* \bm{y} - \bm{y}^* A \bm{x} + \|\bm{y}\|_2^2
    \end{align}
    ここで,
    $A$ と $L$ の核空間の共通部分が $\bm{0}$ のみでかつ
    $\lambda > 0$ の場合は
    $\bm{x} \neq \bm{0}$ において
    $\bm{x}^* (A^* A + \lambda L^* L) \bm{x} = \|A \bm{x}\|_2^2 + \lambda \|L \bm{x}\|_2^2 > 0$
    となることから,
    エルミート行列 $(A^* A + \lambda I)$ は正定値であり,
    正則となる.
    このことを用いると,さらに次のように展開できる.
    \begin{align}
        E_{\lambda}(\bm{x})
         & = \bm{x}^* (A^* A + \lambda L^* L) \bm{x} - \bm{x}^* A^* \bm{y} - \bm{y}^* A \bm{x} + \|\bm{y}\|_2^2
        \notag                                                                                                  \\
         & = \bm{x}^* (A^* A + \lambda L^* L) \bm{x} - \bm{x}^* A^* \bm{y} - \bm{y}^* A \bm{x}
        + \bm{y}^* A (A^* A + \lambda L^* L)^{-1} A^* \bm{y}
        - \bm{y}^* A (A^* A + \lambda L^* L)^{-1} A^* \bm{y}
        + \|\bm{y}\|_2^2
        \notag                                                                                                  \\
         & = (\bm{x} - (A^* A + \lambda L^* L)^{-1} A^* \bm{y})^*
        (A^* A + \lambda L^* L)
        (\bm{x} - (A^* A + \lambda L^* L)^{-1} A^* \bm{y})
        - \bm{y}^* A (A^* A + \lambda L^* L)^{-1} A^* \bm{y}
        + \|\bm{y}\|_2^2
    \end{align}
    エルミート行列 $(A^* A + \lambda L^* L)$ が正定値であることから,
    この式が最小となるのは
    $(\bm{x} - (A^* A + \lambda L^* L)^{-1} A^* \bm{y})$
    が零ベクトルとなる場合,
    つまり,
    $\bm{x} = (A^* A + \lambda L^* L)^{-1} A^* \bm{y}$
    の場合である.
\end{proof}

\section{一般化特異値分解}

$m \ge n \ge p$ の場合は
次のように表される一般化特異値分解を行うことができる \cite{Hansen1998}.

\begin{align}
    A       & =U
    \begin{pmatrix}
        \Sigma & O \\
        O      & I
    \end{pmatrix}
    W^{-1}, &
    L       & =V
    \begin{pmatrix}
        M & O
    \end{pmatrix}
    W^{-1}
\end{align}

ここで,
$U \in \setC^{m \times n}$,
$V \in \setC^{p \times p}$
はユニタリ行列で,
$W \in \setC^{n \times n}$
は正則行列とし,
$\Sigma = \diag(\sigma_1, \sigma_2, \ldots, \sigma_p)$,
$M = \diag(\mu_1, \mu_2, \ldots, \mu_p)$
は対角行列である.
$\sigma_i$ と $\mu_i$ については
$0 \le \sigma_1 \le \sigma_2 \le \ldots \le \sigma_p \le 1$,
$1 \ge \mu_1 \ge \mu_2 \ge \ldots \ge \mu_p > 0$,
$\sigma_i^2 + \mu_i^2 = 1$
を満たすものとし,
$\gamma_i = \sigma_i / \mu_i$
を一般化特異値と呼ぶ.
これを用いると,
次のように評価関数 \eqref{eq:regularization_gen-tikhonov_objective} を最小化する
$\bm{x}$ を表せる\cite{Hansen1998}.

\begin{equation}
    \bm{x}_\lambda =
    \sum_{i=1}^{p} \frac{\gamma_i / \mu_i}{\gamma_i^2+\lambda}
    (\bm{u}_i^*\bm{y}) \bm{w}_i
    +\sum_{i=p+1}^n (\bm{u}_i^*\bm{y}) \bm{w}_i
\end{equation}

$L=I$ のときは,
$p = n$, $\mu_i = 1$ となり,
Tikhonov 正則化の場合の
式 \eqref{eq:regularization_tikhonov_solution-by-svd} に一致する.

\section{単純な L2 ノルムによる Tikhonov 正則化への変換}

一般化 Tikhonov 正則化を通常の Tikhonov 正則化へ
変形することもできる \cite{Hansen1998}.
その変形では,次のように変数を置き換える.
\begin{align}
    \bm{x}      & = L_A^\dagger \bar{\bm{x}} + \bm{x}_0,                     &
    L_A^\dagger & = \left(I - \left(A P_L\right)^\dagger A\right) L^\dagger, &
    \bm{x}_0    & = \left(A P_L\right)^\dagger \bm{y}
    \label{eq:regularization_gen-tikhonov_change-of-variables}
\end{align}
ただし,$P_L = I - L^\dagger L$とする.
一般化 Tikhonov 正則化の問題を数値的に解く際は
この変換で通常の Tikhonov 正則化に直して解くと
安定かつ効率的に解が得られることが知られている \cite{Hansen1998}.
そこで,この変換について以下に示す.

まず,式 \eqref{eq:regularization_gen-tikhonov_change-of-variables} の変換は
次のような意味を持っている.

\begin{lemma}[{\cite[3 節]{Elden1982}}]\label{lemma:regularization_gen-tikhonov_meaning-of-change-of-variables}
    最適化問題
    \begin{align}
        \text{minimize} \hspace{1.5em} & \|A \bm{x} - \bm{y}\|_2 \notag \\
        \text{s.t.} \hspace{1.5em}     & L \bm{x} = \bar{\bm{x}} \notag
    \end{align}
    において,
    行列
    $A \in \setC^{m \times n}$,
    $L \in \setC^{p \times n}$
    ($m \ge n \ge p$)は
    核空間に $\bm{0}$ 以外の共通部分を持たないものとする.
    このとき,この問題の解は
    \begin{equation}
        \bm{x} = L_A^\dagger \bar{\bm{x}} + (A P_L)^\dagger\bm{y}
    \end{equation}
    である.
\end{lemma}
\begin{proof}
    まず,$L\bm{x}=\bar{\bm{x}}$ の一般解は次のように書ける
    (式 \eqref{eq:matrix-computation_moore-penrose_general-least-squares-solution}).
    \begin{equation}
        \bm{x} = L^\dagger \bar{\bm{x}} + P_L \bm{z}
    \end{equation}
    これを目的関数に代入すると
    \begin{equation}
        \|A\bm{x} - \bm{y}\|_2
        =\|A L^\dagger \bar{\bm{x}} + A P_L \bm{z} - \bm{y}\|_2
        =\|(A P_L) \bm{z} - (\bm{y} - A L^\dagger \bar{\bm{x}})\|_2
    \end{equation}
    のように変形でき,
    このノルムを最小化する$\bm{z}$は次のように書ける.
    \begin{equation}
        \bm{z} =
        (A P_L)^\dagger (\bm{y} - A L^\dagger \bar{\bm{x}})
        + (I - (A P_L)^\dagger (A P_L)) \bm{s}
    \end{equation}
    ここで,$\bm{s}$ は任意のベクトルである.
    ここで,$P_L$ は $L$ の核空間への射影演算子であり,
    $A$ と $L$ が核空間に $\bm{0}$ 以外の共通部分を持たないことから,
    $A P_L$ の核空間は $L$ の核空間の直交補空間である.
    任意のベクトル $\bm{a}$ について
    $(A P_L)^\dagger \bm{a}$ は $A P_L$ の核空間の成分を持たないため,
    $L$ の核空間に属していることが分かる.
    一方,
    $(I - (A P_L)^\dagger(A P_L)) \bm{s}$
    は $A P_L$ の核空間に属しており,
    $L$ の核空間に直交する.
    よって,
    \begin{equation}
        P_L \bm{z} =
        (A P_L)^\dagger (\bm{y} - A L^\dagger \bar{\bm{x}})
    \end{equation}
    となる.

    このときの $\bm{x}$ は
    \begin{align}
        \bm{x} & = L^\dagger \bar{\bm{x}} + P_L \bm{z} \notag                                               \\
               & = L^\dagger \bar{\bm{x}} +(A P_L)^\dagger (\bm{y} - A L^\dagger \bar{\bm{x}}) \notag       \\
               & = \left(I - (A P_L)^\dagger A \right)L^\dagger \bar{\bm{x}} + (AP_L)^\dagger \bm{y} \notag \\
               & = L_A^\dagger \bar{\bm{x}} + (A P_L)^\dagger \bm{y}
    \end{align}
    となり,解が得られる.
\end{proof}

この補題を用いることで,次のように評価関数の変換を示すことができる.

\begin{theorem}[{\cite{Hansen1998}}]
    $A$ と $L$ の核空間の共通部分が $\bm{0}$ のみである場合,
    変数変換 \eqref{eq:regularization_gen-tikhonov_change-of-variables} により
    評価関数 \eqref{eq:regularization_gen-tikhonov_objective} を次のように置き換えることができる.
    \begin{equation}
        E'(\bar{\bm{x}}) =
        \left\|\bar{A} \bar{\bm{x}} - \bar{\bm{y}}\right\|_2^2
        + \lambda \|\bar{\bm{x}}\|_2^2
    \end{equation}
    ただし,
    \begin{align}
        \bar{A}      & = A L_A^\dagger                     \\
        \bar{\bm{y}} & = \bm{y} - A \bm{x}_0               \\
        \bm{x}_0     & = \left(A P_L\right)^\dagger \bm{y}
    \end{align}
    である.
\end{theorem}
\begin{proof}
    補題 \ref{lemma:regularization_gen-tikhonov_meaning-of-change-of-variables} より,
    次のように変形できる.

    \begin{align}
        \min_{\bm{x}\in\setC^n} E(\bm{x})
         & = \min_{\bm{x}\in\setC^n} \left( \|A \bm{x} - \bm{y}\|_2^2 + \lambda \|L \bm{x}\|_2^2 \right)
        \notag                                                                                           \\
         & = \min_{\bar{\bm{x}} \in \setC^p}
        \min_{\bm{x} \in \{\bm{x} \in \setC^n \mid L \bm{x} = \bar{\bm{x}} \}}
        \left( \|A \bm{x} - \bm{y}\|_2^2 + \lambda \|L \bm{x}\|_2^2 \right)
        \notag                                                                                           \\
         & = \min_{\bar{\bm{x}} \in \setC^p}
        \left( \left\|A \left(L_A^\dagger \bar{\bm{x}} + \bm{x}_0\right) - \bm{y}\right\|_2^2
        + \lambda \left\|\bar{\bm{x}}\right\|_2^2 \right)
        \notag                                                                                           \\
         & = \min_{\bar{\bm{x}} \in \setC^p}
        \left( \left\|\bar{A} \bar{\bm{x}} - \bar{\bm{y}}\right\|_2^2
        + \lambda \left\|\bar{\bm{x}}\right\|_2^2 \right)
        \notag                                                                                           \\
         & = \min_{\bar{\bm{x}} \in \setC^p} E'(\bar{\bm{x}})
    \end{align}

    以上より,$E(\bm{x})$ の最小化は $E'(\bar{\bm{x}})$ の最小化へ置き換えることができる.
\end{proof}

最後にこの変換をコンピュータ上で行う手法について
文献 \cite{Elden1982, Hansen1994} の結果をまとめる.

\subsection{行列 L が列と同じ数のランクを持つ場合の計算方法}

まず,$L^*$ を QR 分解する.
\begin{equation}
    L^* = \begin{pmatrix}
        V_1 & V_2
    \end{pmatrix}\begin{pmatrix}
        R \\ O
    \end{pmatrix}
\end{equation}

ここで,
$V_1 \in \setC^{n \times p}$,
$V_2 \in \setC^{n \times (n-p)}$
は $V = (V_1, V_2)$ がユニタリ行列になるような
行列とし,
$R \in \setC^{p \times p}$ は
正則な上三角行列である.
このとき,$L^\dagger = V_1 R^{-*}$, $LV_2 = O$ となることと,
$A$ と $L$ の核空間の共通部分が $\bm{0}$ のみであることから,
$AV_2$ は列と同じ数のランクを持つ.
よって,次のように QR 分解できる.
\begin{equation}
    AV_2 = \begin{pmatrix}
        Q_1 & Q_2
    \end{pmatrix}\begin{pmatrix}
        U \\ O
    \end{pmatrix}
\end{equation}

これも同様に
$Q_1 \in \setC^{m \times (n-p)}$,
$Q_2 \in \setC^{m \times (m-n+p)}$
は$Q = (Q_1, Q_2)$ がユニタリ行列になるような行列とし,
$U \in \setC^{(n - p) \times (n - p)}$
は正則な上三角行列である.
このとき,
\begin{align}
    \bm{x}_0 & = \left(A (I - L^\dagger L)\right)^\dagger \bm{y} \notag \\
             & = \left(A V_2 V_2^*\right)^\dagger \bm{y}
\end{align}
となる.

ここで,$(AV_2V_2^*)^\dagger$ について定義に沿って考える.
最適化問題
\begin{align}
    \text{minimize} \hspace{2em} & \|\bm{x}\|_2 \notag                                                                    \\
    \text{s.t.} \hspace{2em}     & \|AV_2V_2^* \bm{x} - \bm{a}\|_2 = \min_{\bm{x}} \|AV_2V_2^* \bm{x} - \bm{a}\|_2 \notag
\end{align}
において,
$\bm{x} = V_1 \bm{s}_1 + V_2 \bm{s}_2$
とおくと,
$\|A V_2 V_2^* \bm{x} - \bm{a}\|_2 = \|A V_2 \bm{s}_2 - \bm{a}\|_2$
となる.
これを最小化すると$\bm{s}_2=(AV_2)^\dagger\bm{a}$となる.
また,
$\|\bm{x}\|_2^2 = \|\bm{s}_1\|_2^2 + \|\bm{s}_2\|_2^2$
となるから,
$\bm{s}_2 = (A V_2)^\dagger \bm{a}$
は固定された状態で
$\|\bm{x}\|_2$
を最小化すると
$\bm{s}_1=\bm{0}$
となる.
よって,この最適化問題の解は
$\bm{x} = V_2 (A V_2)^\dagger \bm{a}$
となり,
$(A V_2 V_2^*)^\dagger = V_2 (A V_2)^\dagger$
とわかる.

よって,
\begin{align}
    \bm{x}_0 & = \left(A (I - L^\dagger L)\right)^\dagger \bm{y} \notag \\
             & = V_2 \left(A V_2\right)^\dagger \bm{y} \notag           \\
             & = V_2 U^{-1} Q_1^* \bm{y}
\end{align}
となる.また,
\begin{align}
    \bar{\bm{y}} & = \bm{y} - A \bm{x}_0 \notag                \\
                 & = \bm{y} - A V_2 U^{-1} Q_1^* \bm{y} \notag \\
                 & = (I - Q_1 Q_1^*) \bm{y} \notag             \\
                 & = Q_2 Q_2^* \bm{y}
\end{align}
と書ける.さらに,
\begin{align}
    L_A^\dagger & = \left(I - \left(A (I - L^\dagger L)\right)^\dagger A\right)L^\dagger \notag \\
                & = (I - V_2 U^{-1} Q_1^* A) V_1 R^{-*}
\end{align}
となるから,
\begin{align}
    \bar{A} & = A L_A^\dagger \notag                         \\
            & = A (I - V_2 U^{-1} Q_1^* A) V_1 R^{-*} \notag \\
            & = (A - A V_2 U^{-1} Q_1^* A) V_1 R^{-*} \notag \\
            & = (I - A V_2 U^{-1} Q_1^*) A V_1 R^{-*} \notag \\
            & = (I - Q_1 U U^{-1} Q_1^*) A V_1 R^{-*} \notag \\
            & = (I - Q_1 Q_1^*) A V_1 R^{-*} \notag          \\
            & = Q_2 Q_2^* A V_1 R^{-*}
\end{align}
と変形できる.
そして,$\bar{\bm{x}}$ から $\bm{x}$への変換は
\begin{align}
    \bm{x} & = L_A^\dagger \bar{\bm{x}} + \bm{x}_0 \notag                                        \\
           & = (I - V_2 U^{-1} Q_1^* A) V_1 R^{-*} \bar{\bm{x}} + V_2 U^{-1} Q_1^* \bm{y} \notag \\
           & = (I - V_2 U^{-1} Q_1^* A) L^\dagger \bar{\bm{x}} + V_2 U^{-1} Q_1^* \bm{y} \notag  \\
           & = L^\dagger \bar{\bm{x}} + V_2 U^{-1} Q_1^* (\bm{y} - A L^\dagger \bar{\bm{x}})
\end{align}
でできる.
さらに,$Q_2^* Q_2 = I$を用いれば,
\begin{align}
    \left\|\bar{A} \bar{\bm{x}} - \bar{\bm{y}}\right\|_2
                   & = \left\|\tilde{A} \bar{\bm{x}} - \tilde{\bm{y}}\right\|_2, &
    \tilde{A}      & = Q_2^* A V_1 R^{-*},                                       &
    \tilde{\bm{y}} & = Q_2^* \bm{y}
\end{align}
のように書き換えることもできる.

\subsection{行列 L が一般の行列の場合の計算方法}

まず,$L$ を特異値分解する.
\begin{equation}
    L = W
    \begin{pmatrix}
        \Omega & O \\
        O      & O
    \end{pmatrix}
    V^*
\end{equation}
ここで,$L$ はランク $r$ で
$W \in \setC^{p \times p}$,
$V \in \setC^{n \times n}$
はユニタリ行列,
$\Omega \in \setR^{r \times r}$
は正数の対角成分による対角行列とする.
このとき,$W = (W_1, W_2)$ なる
行列 $W_1 \in \setC^{p \times r}$ と $W_2 \in \setC^{p \times (p-r)}$,
$V = (V_1, V_2)$ なる
行列 $V_1 \in \setC^{n \times r}$ と $V_2 \in \setC^{n \times (n-r)}$
を定義する.

このとき,$L$ が列と同じ数のランクを持つ場合と同様にして
\begin{align}
    \bm{x}_0       & = V_2 U^{-1} Q_1^* \bm{y}                        \\
    \bar{\bm{y}}   & = Q_2 Q_2^* \bm{y}                               \\
    L_A^\dagger    & = (I - V_2 U^{-1} Q_1^* A) V_1 \Omega^{-1} W_1^* \\
    \bar{A}        & = Q_2 Q_2^* A V_1 \Omega^{-1} W_1^*              \\
    \tilde{A}      & = Q_2^* A V_1 \Omega^{-1} W_1^*                  \\
    \tilde{\bm{y}} & = Q_2^* \bm{y}
\end{align}
が得られる.

% !TEX root = ../main.tex
%

\chapter{L-curve}

ここでは正則化パラメータを決める手法の1つである
L-curve法について説明する.

\begin{figure}[tp]\centering
    \includegraphics{./regularization/L-curve.pdf}
    \caption{L-curve の概形}
    \label{fig:regularization_l-curve_l-curve}
\end{figure}

正則化の式 \eqref{eq:regularization_intro_general-regularization} を最小化する
$\bm{x}$ を $\bm{x}_\lambda$ として,
横軸を残差 $\|A\bm{x}_\lambda-\bm{y}\|$,
縦軸を正則化項 $R(\bm{x}_\lambda)$ とし,
正則化パラメータ $\lambda$ を変えたときの
プロットを L-curve と呼ぶ.
L-curve は一般に図 \ref{fig:regularization_l-curve_l-curve} のような L の字を
描いている場合が多い.
単純にノルムをプロットするよりも
両対数グラフにプロットする方が
後述する曲線の特徴をはっきりさせられる他,
スケールの違いによる影響を
抑えられるなどのメリットもあるという
\cite{Hansen1998}.

図 \ref{fig:regularization_l-curve_l-curve} のように L-curve が得られた場合,
L の字の曲がり角にあたる赤い点の部分から左上の方では
残差がほとんど減らずに正則化項が増加しており,
右下の方では正則化項がほとんど減らずに残差が増えているため,
両方のバランスが取れている L の字の曲がり角を取るのが良いと考えられる.
そこで,L-curve の曲がり角を数値的に求める手法について考える.

ここでは,文献 \cite{Hansen1998} に従い次の関数の組を考える.
\begin{equation}
    (\xi(\lambda), \eta(\lambda))
    =(\log\|A \bm{x}_\lambda - \bm{y}\|, \log{R(\bm{x})})
\end{equation}
L-curve の曲がり角は曲線 $(\xi(\lambda), \eta(\lambda))$ の
曲率が最も大きい部分と考えられるため,曲率
\begin{equation}
    \kappa(\lambda) =
    \frac{\xi'\eta'' - \xi''\eta'}
    {\left( (\xi')^2 + (\eta')^2 \right)^{3/2}}
\end{equation}
を求め,それを最大化する.
曲率が何らかの手法で求まれば,
1 変数の最適化については
\ref{chap:opt_one-dim-section-search} 章で説明しているため,
ここでは曲率の計算法について考える.

% !TEX root = ../main.tex
%

\chapter{Generalized Cross Validation}

正則化パラメータを求めるための手法の 1 つに
Generalized Cross Validation (GCV) がある.
GCV では,次の式の最小化を行う
\cite{Wahba1990}.

\begin{equation}
    V(\lambda)
    = \frac{\frac{1}{m} \|A \bm{x}_\lambda - \bm{y}\|^2}
    {\left|\frac{1}{m} \tr\{I - P_A(\lambda)\}\right|^2}
    \label{eq:regularization_gcv_gdv-evaluation-function}
\end{equation}

ここで,$\bm{x}_\lambda$ は
正則化の式 \eqref{eq:regularization_intro_general-regularization} を最小化する
$\bm{x}$であり,
$P_A(\lambda)$ は influence matrix と呼ばれるもので,
データ $\bm{y}$ から解 $\bm{x}_\lambda$ を求める作用素を $A_\lambda^\#$ としたときに,
\begin{equation}
    P_A(\lambda) \equiv
    \frac{\partial A A_\lambda^\# \bm{y}}{\partial \bm{y}}
    \label{eq:regularization_gcv_influence-matrix}
\end{equation}
のように定義される.

\section{導出}

まず,GCV の基になっている Ordinary Cross Validation (OCV) を
文献 \cite{Wahba1990} に沿って示す.

OCV では次の関数を考える.
\begin{equation}
    V_0(\lambda) \equiv
    \frac{1}{m} \sum_{i=1}^m
    \left|y_i - A_i \bm{x}_\lambda^{(i)}\right|^2
    \label{eq:regularization_gcv_ocv-evaluation-function}
\end{equation}
ただし,$\bm{x}_\lambda^{(i)}$ は $y_i$ 以外のデータから求めた解で,
$A_i$ は解からデータ $y_i$ を推定する作用素である.

ここで次の補題が成り立つ.
\begin{lemma}[文献 {\cite{Wahba1990}} の補題 4.2.1 より, leaving-out-one lemma]
    評価関数
    \begin{equation}
        \frac{1}{m} \left(\left|z - A_k \bm{x}\right|^2
        + \sum_{i \neq k} \left|y_i - A_i \bm{x}\right|^2\right)
        + \lambda R(\bm{x})
    \end{equation}
    を最小化するような $\bm{x}$ を $\bm{h}_\lambda[k,z]$ とし,
    評価関数
    \begin{equation}
        \frac{1}{m}
        \sum_{i \neq k} \left|y_i - A_i \bm{x}\right|^2
        + \lambda R(\bm{x})
    \end{equation}
    を最小化するような $\bm{x}$ を $\bm{x}_\lambda^{(k)}$ としたとき,
    $h_\lambda[k, A_k \bm{x}_\lambda^{(k)}] = \bm{x}_\lambda^{(k)}$
    が成り立つ.
\end{lemma}
\begin{proof}
    $\tilde{y}_k = A_k \bm{x}_\lambda^{(k)}$とし,
    $\bm{x} \neq \bm{x}_\lambda^{(k)}$とする.
    このとき,
    \begin{align}
         & \hphantom{=}
        \frac{1}{m} \left(\left|\tilde{y}_k - A_k \bm{x}_\lambda^{(k)}\right|^2
        + \sum_{i \neq k} \left|y_i - A_i \bm{x}_\lambda^{(k)}\right|^2\right)
        + \lambda R(\bm{x}_\lambda^{(k)})
        \notag                                                                         \\
         & = \frac{1}{m} \sum_{i \neq k} \left|y_i - A_i \bm{x}_\lambda^{(k)}\right|^2
        + \lambda R(\bm{x}_\lambda^{(k)})
        \notag                                                                         \\
         & < \frac{1}{m} \sum_{i \neq k} \left|y_i - A_i \bm{x}\right|^2
        + \lambda R(\bm{x})
        \notag                                                                         \\
         & \le \frac{1}{m} \left(\left|\tilde{y}_k - A_k \bm{x}\right|^2
        +\sum_{i \neq k} \left|y_i - A_i \bm{x}\right|^2\right)
        + \lambda R(\bm{x})
    \end{align}
    のように変形できる.
    よって,
    $h_\lambda[k, A_k \bm{x}_\lambda^{(k)}] = \bm{x}_\lambda^{(k)}$
    である.
\end{proof}

これを用い,次の量を考える.
\begin{equation}
    \tilde{p}_{kk}(\lambda)
    \equiv \frac{A_k \bm{x}_\lambda - A_k \bm{x}_\lambda^{(k)}}
    {y_k - A_k \bm{x}_\lambda^{(k)}}
\end{equation}
まず,定義から,
\begin{equation}
    y_k - A_k \bm{x}_\lambda^{(k)} =
    \frac{y_k - A_k \bm{x}_\lambda}{1 - \tilde{p}_{kk}(\lambda)}
\end{equation}
が成り立つ.
また,補題から
\begin{equation}
    \tilde{p}_{kk}(\lambda)
    = \frac{A_k h[k,y_k] - A_k h[k,\tilde{y}_k]}{y_k - \tilde{y}_k}
\end{equation}
が成り立つ.
さらに,これは 1 次近似により
\begin{equation}
    \tilde{p}_{kk}(\lambda)
    \approx \frac{\partial A_k h[k,y_k]}{\partial y_k}
    =\frac{\partial A_k \bm{x}_\lambda}{\partial y_k}
    =\frac{\partial A_k A_\lambda^\# \bm{y}}{\partial y_k}
    =p_{kk}(\lambda)
\end{equation}
と書ける.
ただし,$p_{kk}(\lambda)$ は
influence matrix $P_A(\lambda)$ の $(k,k)$ 成分である.

このことを利用して OCV の評価関数
\eqref{eq:regularization_gcv_ocv-evaluation-function}
を変形すると,
\begin{equation}
    V_0(\lambda)=
    \frac{1}{m} \sum_{i=1}^m
    \left|y_i - A_i \bm{x}_\lambda^{(i)}\right|^2
    \approx
    \frac{1}{m} \sum_{i=1}^m
    \frac{\left|y_i - A_i \bm{x}_\lambda\right|^2}
    {\left|1 - p_{kk}(\lambda)\right|^2}
\end{equation}
のように書ける.

$p_{kk}(\lambda)$
の代わりに,influence matrix の対角成分の平均値
$\tr P_A(\lambda)/n$
を用い,
$\bm{y}$
の回転に影響されない評価関数を作ると,
GCVの評価関数 \eqref{eq:regularization_gcv_gdv-evaluation-function} が得られる.

