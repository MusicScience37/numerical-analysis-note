% !TEX root = ../main.tex
%

\chapter{スパース正則化}\label{chap:regularization_sparse}
\index{すぱーすせいそくか@スパース正則化}

本章では,解またはその微分のスパース性を要求する正則化手法についてまとめる.

Tikhonov 正則化では,未知パラメータ $\bm{x}$ に関する
線型方程式 $A \bm{x} = \bm{y}$ の係数行列 $A$ の特異値の分布が悪く
単純な最小二乗解 $\bm{x} = A^\dagger \bm{y}$ が安定して計算できない場合に,
解の 2-ノルム $\|\bm{x}\|_2$ が大きくなりすぎないようにするため,
ただの最小二乗法の代わりに評価関数
\begin{equation}
    E_{\lambda}(\bm{x}) \equiv \|A \bm{x} - \bm{y}\|_2^2 + \lambda \|\bm{x}\|_2^2
\end{equation}
を最小化していた.
一方,スパース正則化では 2-ノルムの代わりに 1-ノルムを用い,例えば評価関数
\begin{equation}
    E_{\lambda}(\bm{x}) \equiv \|A \bm{x} - \bm{y}\|_2^2 + \lambda \|\bm{x}\|_1^2
    \label{eq:regularization_sparse_l1_objective}
\end{equation}
を最小化する.
1-ノルム $\|\bm{x}\|_1$ で正則化を行うと,
$\bm{x}$ の要素のうち多くの要素は 0 になるように,
つまりスパースになるように $\bm{x}$ を求めることができる.
また,$\bm{x}$ が空間上のパラメータの分布を示す場合,
その微分を計算するような行列 $D$ を用いたノルム $\|D \bm{x}\|_1$ を用いて正則化することにより,
空間上でパラメータが変化する箇所を少なくするという正則化
(Total variation 正則化
\index{Total variation せいそくか@Total variation 正則化}
\index{TVせいそくか@TV 正則化|see{Total variation 正則化}})
を行うこともできる.
これらのようなスパース正則化には,
\begin{itemize}
    \item 未知パラメータ $\bm{x}$ の要素数よりもデータ $\bm{y}$ の要素数の方が少ないという
          情報が少ない状況(「劣決定」\index{れつけってい@劣決定}と呼ばれる)でも
          比較的安定して $\bm{x}$ を求めることができる.
\end{itemize}
という利点がある.
しかし一方で,
\begin{itemize}
    \item Tikhonov 正則化と違って解を陽的に求めることができない.
    \item 評価関数は微分できない点を持つ.
\end{itemize}
といった欠点を持つ.
そこで,スパース正則化に特化した反復法が考えられている.
そのようなスパース正則化のアルゴリズムの例について以下に示す.
