% !TEX root = ../../main.tex
%

\section{Total generalized variation 正則化}
\index{Total generalized variation せいそくか@Total generalized variation 正則化}
\index{TGVせいそくか@TGV 正則化|see{Total generalized variation 正則化}}

TV 正則化は領域内の 1 階微分をスパースにする正則化だが,
それを拡張して 2 階以上の微分までスパースにする
TGV (total generalized variation) 正則化が考えられている
\cite{Bredies2010}.

ここでは,2 階微分までスパースにする
2 階の TGV(「$\mathrm{TGV}_{\alpha}^2$」と書かれる)について説明する.
$\bm{x}$ から各点の 1 階微分を算出する行列を $D$,
1 階微分から 2 階微分を算出する行列を $E$ とすると,
$\mathrm{TGV}_{\alpha}^2$ の正則化項は
\begin{equation}
    \mathrm{TGV}_{\alpha}^2(\bm{x}) \equiv
    \min_{\bm{x}_1 + \bm{x}_2 = \bm{x}} \left(
    \|D \bm{x}_1\|_1 + \alpha \|E D \bm{x}_2\|_1
    \right)
\end{equation}
のように書ける
(文献 \cite{Bredies2010} の積分による表現を離散化したもの).
ここで,$\alpha$ は正の定数である.
この正則化項では,1 階微分と 2 階微分をスパースにしている.

$\mathrm{TGV}_{\alpha}^2$ の正則化項は,
$\bm{z} = D \bm{x}_2$
とすると
\begin{equation}
    \mathrm{TGV}_{\alpha}^2(\bm{x}) =
    \min_{\bm{z}} \left(
    \|D \bm{x} - \bm{z}\|_1 + \alpha \|E \bm{z}\|_1
    \right)
\end{equation}
のようにも書ける
(文献 \cite{Li2016} における表現).

$\mathrm{TGV}_{\alpha}^2$ により正則化する評価関数は例えば
\begin{equation}
    E_{\lambda}(\bm{x}, \bm{z}) \equiv \|A \bm{x} - \bm{y}\|_2^2
    + \lambda \left( \|D \bm{x} - \bm{z}\|_1 + \alpha \|E \bm{z}\|_1 \right)
\end{equation}
のように書ける.
$\bm{x}$ と $\bm{z}$ の両方を変化させて最小値を探索する必要がある.
そのアルゴリズムについて以下にまとめる.

\subsection{交互方向乗数法}
\index{こうごほうこうじょうすうほう@交互方向乗数法}

文献 \cite{Li2016} では,
$\mathrm{TGV}_{\alpha}^2$ を含む評価関数の最適化について
交互方向乗数法(\ref{sec:optimization_admm} 節)を用いている.
その解法をもとに,評価関数の最適化問題を以下のように書き換える.
\begin{equation}
    \begin{aligned}
        \text{minimize} \hspace{1em} & \|A \bm{x} - \bm{y}\|_2^2
        + \lambda \left( \|\bm{s}\|_1 + \alpha \|\bm{t}\|_1 \right)
        \\
        \text{s.t.} \hspace{1em}     & D \bm{x} - \bm{z} = \bm{s} \\
                                     & E \bm{z} = \bm{t}
    \end{aligned}
\end{equation}
ここで,最適化変数は $\bm{x}$, $\bm{z}$, $\bm{s}$, $\bm{t}$ である.

ラグランジュ乗数 $\bm{p}$, $\bm{u}$ を用いて
拡張ラグランジュ関数は以下のように書ける.
\begin{equation}
    L_{\rho}(\bm{x}, \bm{z}, \bm{s}, \bm{t}, \bm{p}, \bm{u})
    \equiv
    \|A \bm{x} - \bm{y}\|_2^2
    + \lambda \left( \|\bm{s}\|_1 + \alpha \|\bm{t}\|_1 \right)
    + \bm{p}^\top (D \bm{x} - \bm{z} - \bm{s})
    + \bm{u}^\top (E \bm{z} - \bm{t})
    + \frac{\rho}{2} \|D \bm{x} - \bm{z} - \bm{s}\|_2^2
    + \frac{\rho}{2} \|E \bm{z} - \bm{t}\|_2^2
\end{equation}

交互方向乗数法の更新式は以下のようになる.
\begin{align}
    \bm{x}_{k+1} & = \argmin_{\bm{x}}
    L_{\rho}(\bm{x}, \bm{z}_k, \bm{s}_k, \bm{t}_k, \bm{p}_k, \bm{u}_k)
    \\
    \bm{z}_{k+1} & = \argmin_{\bm{z}}
    L_{\rho}(\bm{x}_{k+1}, \bm{z}, \bm{s}_k, \bm{t}_k, \bm{p}_k, \bm{u}_k)
    \\
    \bm{s}_{k+1} & = \argmin_{\bm{s}}
    L_{\rho}(\bm{x}_{k+1}, \bm{z}_{k+1}, \bm{s}, \bm{t}_k, \bm{p}_k, \bm{u}_k)
    \\
    \bm{t}_{k+1} & = \argmin_{\bm{t}}
    L_{\rho}(\bm{x}_{k+1}, \bm{z}_{k+1}, \bm{s}_{k+1}, \bm{t}, \bm{p}_k, \bm{u}_k)
    \\
    \bm{p}_{k+1} & = \bm{p}_k + \rho (D \bm{x}_{k+1} - \bm{z}_{k+1} - \bm{s}_{k+1})
    \\
    \bm{u}_{k+1} & = \bm{u}_k + \rho (E \bm{z}_{k+1} - \bm{t}_{k+1})
\end{align}

この更新式を計算可能なように単純化する.
まず,$\bm{x}$ の更新式は,$D$ が列フルランクとすると
\begin{align}
    \bm{x}_{k+1}
     & = \argmin_{\bm{x}}
    L_{\rho}(\bm{x}, \bm{z}_k, \bm{s}_k, \bm{t}_k, \bm{p}_k, \bm{u}_k)
    \notag                                        \\
     & = \argmin_{\bm{x}} \left(
    \|A \bm{x} - \bm{y}\|_2^2
    + \bm{p}_k^\top (D \bm{x} - \bm{z}_k - \bm{s}_k)
    + \frac{\rho}{2} \|D \bm{x} - \bm{z}_k - \bm{s}_k\|_2^2
    \right)
    \notag                                        \\
     & = (2 A^\top A + \rho D^\top D)^{-1} \left(
    2 A^\top \bm{y} - D^\top \bm{p}_k + \rho D^\top \bm{z}_k + \rho D^\top \bm{s}_k
    \right)
\end{align}
となる.
$\bm{z}$ の更新式は,
\begin{align}
    \bm{z}_{k+1}
     & = \argmin_{\bm{z}}
    L_{\rho}(\bm{x}_{k+1}, \bm{z}, \bm{s}_k, \bm{t}_k, \bm{p}_k, \bm{u}_k)
    \notag                                    \\
     & = \argmin_{\bm{z}} \left(
    \bm{p}_k^\top (D \bm{x}_{k+1} - \bm{z} - \bm{s}_k)
    + \bm{u}_k^\top (E \bm{z} - \bm{t}_k)
    + \frac{\rho}{2} \|D \bm{x}_{k+1} - \bm{z} - \bm{s}_k\|_2^2
    + \frac{\rho}{2} \|E \bm{z} - \bm{t}_k\|_2^2
    \right)
    \notag                                    \\
     & = (\rho I + \rho E^\top E)^{-1} \left(
    \bm{p}_k - E^\top \bm{u}_k + \rho D \bm{x}_{k+1} - \rho \bm{s}_k + \rho E^\top \bm{t}_k
    \right)
\end{align}
となる.
$\bm{s}$ の更新式は,
\begin{align}
    \bm{s}_{k+1}
     & = \argmin_{\bm{s}}
    L_{\rho}(\bm{x}_{k+1}, \bm{z}_{k+1}, \bm{s}, \bm{t}_k, \bm{p}_k, \bm{u}_k)
    \notag                                                                                               \\
     & = \argmin_{\bm{s}} \left(
    \lambda \|\bm{s}\|_1
    + \bm{p}_k^\top (D \bm{x}_{k+1} - \bm{z}_{k+1} - \bm{s})
    + \frac{\rho}{2} \|D \bm{x}_{k+1} - \bm{z}_{k+1} - \bm{s}\|_2^2
    \right)
    \notag                                                                                               \\
     & = \argmin_{\bm{s}} \left(
    \lambda \|\bm{s}\|_1
    + \frac{\rho}{2} \left\|D \bm{x}_{k+1} - \bm{z}_{k+1} - \bm{s} + \frac{\bm{p}_k}{\rho} \right\|_2^2
    \right)
    \notag                                                                                               \\
     & = \mathcal{T}_{\lambda/\rho} \left( D \bm{x}_{k+1} - \bm{z}_{k+1} + \frac{\bm{p}_k}{\rho} \right)
\end{align}
となる.
$\bm{t}$ の更新式は,
\begin{align}
    \bm{t}_{k+1}
     & = \argmin_{\bm{t}}
    L_{\rho}(\bm{x}_{k+1}, \bm{z}_{k+1}, \bm{s}_{k+1}, \bm{t}, \bm{p}_k, \bm{u}_k)
    \notag                                                                                       \\
     & = \argmin_{\bm{t}} \left(
    \lambda \alpha \|\bm{t}\|_1
    + \bm{u}_k^\top (E \bm{z}_{k+1} - \bm{t})
    + \frac{\rho}{2} \|E \bm{z}_{k+1} - \bm{t}\|_2^2
    \right)
    \notag                                                                                       \\
     & = \argmin_{\bm{t}} \left(
    \lambda \alpha \|\bm{t}\|_1
    + \frac{\rho}{2} \left\|E \bm{z}_{k+1} - \bm{t} + \frac{\bm{u}_k}{\rho} \right\|_2^2
    \right)
    \notag                                                                                       \\
     & = \mathcal{T}_{\lambda/\rho \alpha} \left( E \bm{z}_{k+1} + \frac{\bm{u}_k}{\rho} \right)
\end{align}
となる.
以上をまとめると,更新式は以下のようになる.
\begin{align}
    \bm{x}_{k+1}
     & = (2 A^\top A + \rho D^\top D)^{-1} \left(
    2 A^\top \bm{y} - D^\top \bm{p}_k + \rho D^\top \bm{z}_k + \rho D^\top \bm{s}_k
    \right)
    \\
    \bm{z}_{k+1}
     & = (\rho I + \rho E^\top E)^{-1} \left(
    \bm{p}_k - E^\top \bm{u}_k + \rho D \bm{x}_{k+1} - \rho \bm{s}_k + \rho E^\top \bm{t}_k
    \right)
    \\
    \bm{s}_{k+1}
     & = \mathcal{T}_{\lambda/\rho} \left( D \bm{x}_{k+1} - \bm{z}_{k+1} + \frac{\bm{p}_k}{\rho} \right)
    \\
    \bm{t}_{k+1}
     & = \mathcal{T}_{\lambda/\rho \alpha} \left( E \bm{z}_{k+1} + \frac{\bm{u}_k}{\rho} \right)
    \\
    \bm{p}_{k+1}
     & = \bm{p}_k + \rho (D \bm{x}_{k+1} - \bm{z}_{k+1} - \bm{s}_{k+1})
    \\
    \bm{u}_{k+1}
     & = \bm{u}_k + \rho (E \bm{z}_{k+1} - \bm{t}_{k+1})
\end{align}
