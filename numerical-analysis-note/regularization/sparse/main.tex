% !TEX root = ../../main.tex
%

\chapter{スパース正則化}\label{chap:regularization_sparse}
\index{すぱーすせいそくか@スパース正則化}

本章では,解またはその微分のスパース性を要求する正則化手法についてまとめる.

Tikhonov 正則化では,未知パラメータ $\bm{x}$ に関する
線型方程式 $A \bm{x} = \bm{y}$ の係数行列 $A$ の特異値の分布が悪く
単純な最小二乗解 $\bm{x} = A^\dagger \bm{y}$ が安定して計算できない場合に,
解の 2-ノルム $\|\bm{x}\|_2$ が大きくなりすぎないようにするため,
ただの最小二乗法の代わりに評価関数
\begin{equation}
    E_{\lambda}(\bm{x}) \equiv \|A \bm{x} - \bm{y}\|_2^2 + \lambda \|\bm{x}\|_2^2
\end{equation}
を最小化していた.
一方,スパース正則化では 2-ノルムの代わりに 1-ノルムを用い,例えば評価関数
\begin{equation}
    E_{\lambda}(\bm{x}) \equiv \|A \bm{x} - \bm{y}\|_2^2 + \lambda \|\bm{x}\|_1
\end{equation}
を最小化する.
1-ノルム $\|\bm{x}\|_1$ で正則化を行うと,
$\bm{x}$ の要素のうち多くの要素は 0 になるように,
つまりスパースになるように $\bm{x}$ を求めることができる.
また,$\bm{x}$ が何らかの領域におけるパラメータの分布を示す場合,
その微分を計算するような行列 $D$ を用いたノルム $\|D \bm{x}\|_1$ を用いて正則化することにより,
空間上でパラメータが変化する箇所を少なくするという正則化
(Total variation 正則化
\index{Total variation せいそくか@Total variation 正則化}
を行うこともできる.
これらのようなスパース正則化には,
\begin{itemize}
    \item 未知パラメータ $\bm{x}$ の要素数よりもデータ $\bm{y}$ の要素数の方が少ないという
          情報が少ない状況(「劣決定」\index{れつけってい@劣決定}と呼ばれる)でも
          比較的安定して $\bm{x}$ を求めることができる.
\end{itemize}
という利点がある.
しかし一方で,
\begin{itemize}
    \item Tikhonov 正則化と違って解を陽的に求めることができない.
    \item 評価関数は微分できない点を持つ.
\end{itemize}
といった欠点を持つ.
そこで,スパース正則化に特化した反復法が考えられている.
そのようなスパース正則化のアルゴリズムの例を本章では紹介していく.

% !TEX root = ../../main.tex
%

\section{基本的な理論}

本節では,スパース正則化においてよく使用される収縮演算子や変数の分離の仕方などについて説明する.

\subsection{収縮演算子}
\index{しゅしゅくえんざんし@収縮演算子}
\index{shrinkage operator|see{収縮演算子}}

収縮演算子 (shrinkage operator)
は,次のように変数の絶対値を小さくする演算子である.
\begin{equation}
    \mathcal{T}_a(x) \equiv
    \begin{cases}
        x - a & \text{for $x > a$}  \\
        x + a & \text{for $x < -a$} \\
        0     & \text{otherwise}
    \end{cases}
\end{equation}
また,ベクトル $\bm{x}$ について各要素に上記の操作を行った結果は
$\mathcal{T}_a(\bm{x})$ のように示す.

この演算子は,
\begin{equation}
    \mathcal{T}_a(x) =
    \argmin_{\bm{y}} \left(
    \|\bm{y}\|_1 + \frac{1}{2a} \|\bm{y} - \bm{x}\|_2^2
    \right)
\end{equation}
のように 1-ノルムを含む最小化と関係している.
微分ができない点を持つ 1-ノルムが含まれるような評価関数を最小化にあたってこの性質が便利なため,
本章で紹介するアルゴリズムは,
この等式の右辺のような最小化で 1-ノルムの項を扱えるように作られている.

\subsection{変数の分離}

本章で扱う評価関数は,
$\|A \bm{x} - \bm{b}\|_2^2 + \|D \bm{x}\|_1$ のように
微分できる最小二乗の項と微分できない正則化項の組み合わせからできている.
そのような評価関数の最小化を
\begin{align}
    \text{minimize} \hspace{1em} & \|A \bm{x} - \bm{b}\|_2^2 + \|\bm{d}\|_1 \\
    \text{s.t.} \hspace{1em}     & D \bm{x} = \bm{d}
\end{align}
のように最小二乗の項と正則化項で別の変数により扱う.
このように変数を分けて別々に最小化を考えるようにすることで
計算を楽にするというアイデアが,
本章で扱うアルゴリズムの多くで使用されている.

% !TEX root = ../../main.tex
%

\section[L1 正則化]{$L_1$ 正則化}
\index{L1せいそくか@$L_1$ 正則化}

本節では,最も単純な 1-ノルムによる正則化($L_1$ 正則化)による評価関数
\begin{equation}
    E_{\lambda}(\bm{x}) \equiv \|A \bm{x} - \bm{y}\|_2^2 + \lambda \|\bm{x}\|_1
    \label{eq:regularization_sparse_l1_objective}
\end{equation}
を考える.
このような正則化は lasso とも呼ばれる \cite{Boyd2010}.
この評価関数の最小化を行うアルゴリズムについて以下にまとめる.

\subsection{交互方向乗数法}
\index{こうごほうこうじょうすうほう@交互方向乗数法}

式 \eqref{eq:regularization_sparse_l1_objective} の評価関数の最小化を
\begin{equation}
    \begin{aligned}
        \text{minimize} \hspace{1em} & \|A \bm{x} - \bm{y}\|_2^2 + \lambda \|\bm{z}\|_1 \\
        \text{s.t.} \hspace{1em}     & \bm{x} - \bm{z} = \bm{0}
    \end{aligned}
\end{equation}
と書き換えることで,
交互方向乗数法(\ref{sec:optimization_admm} 節)により
評価関数を最小化するような解 $\bm{x}$ を求めることができる \cite{Boyd2010}.

このとき,拡張ラグランジュ関数は
\begin{equation}
    L_{\rho}(\bm{x}, \bm{z}, \bm{p}) \equiv
    \|A \bm{x} - \bm{y}\|_2^2 + \lambda \|\bm{z}\|_1
    + \bm{p}^\top (\bm{x} - \bm{z})
    + \frac{\rho}{2} \|\bm{x} - \bm{z}\|_2^2
\end{equation}
となる.
ここで,ベクトル $\bm{p}$ はラグランジュ乗数,
$\rho$ は正の定数である.

交互方向乗数法における $\bm{x}$ の更新式は,
\begin{align}
    \bm{x}_{k+1}
     & = \argmin_{\bm{x}} L_{\rho}(\bm{x}, \bm{z}_k, \bm{p}_k)
    \notag                                                                          \\
     & = \argmin_{\bm{x}} \left( \|A \bm{x} - \bm{y}\|_2^2 + \lambda \|\bm{z}_k\|_1
    + \bm{p}_k^\top (\bm{x} - \bm{z}_k)
    + \frac{\rho}{2} \|\bm{x} - \bm{z}_k\|_2^2 \right)
    \notag                                                                          \\
     & = \argmin_{\bm{x}} \left( \|A \bm{x} - \bm{y}\|_2^2
    + \bm{p}_k^\top (\bm{x} - \bm{z}_k)
    + \frac{\rho}{2} \|\bm{x} - \bm{z}_k\|_2^2 \right)
\end{align}
となる.
$\bm{x}$ については 2 次関数になっているため,
この最小化は陽的に解を求めることができ,以下のようになる.
\begin{align}
    \bm{x}_{k+1}
     & = (2 A^\top A + \rho I)^{-1} (2 A^\top \bm{y} - \bm{p}_k + \rho \bm{z}_k)
\end{align}

$\bm{z}$ の更新式は,
\begin{align}
    \bm{z}_{k+1}
     & = \argmin_{\bm{z}} L_{\rho}(\bm{x}_{k+1}, \bm{z}, \bm{p}_k)
    \notag                                                                              \\
     & = \argmin_{\bm{z}} \left( \|A \bm{x}_{k+1} - \bm{y}\|_2^2 + \lambda \|\bm{z}\|_1
    + \bm{p}_k^\top (\bm{x}_{k+1} - \bm{z})
    + \frac{\rho}{2} \|\bm{x}_{k+1} - \bm{z}\|_2^2 \right)
    \notag                                                                              \\
     & = \argmin_{\bm{z}} \left( \lambda \|\bm{z}\|_1
    + \bm{p}_k^\top (\bm{x}_{k+1} - \bm{z})
    + \frac{\rho}{2} \|\bm{x}_{k+1} - \bm{z}\|_2^2 \right)
    \notag                                                                              \\
     & = \argmin_{\bm{z}} \left( \lambda \|\bm{z}\|_1
    + \frac{\rho}{2} \left\|\bm{x}_{k+1} - \bm{z} + \frac{\bm{p}_k}{\rho} \right\|_2^2 \right)
    \notag                                                                              \\
     & = \mathcal{T}_{\lambda/\rho} \left( \bm{x}_{k+1} + \frac{\bm{p}_k}{\rho} \right)
\end{align}
となる.
(最後の変形は \ref{sec:regularization_shrinkage-operator} 節を参照.)

$\bm{p}$ の更新式まで含めると以下のようになる
\footnote{文献 \cite{Boyd2010} と定式化のときの係数が異なるため,%
    最終的な更新式の係数が異なっている.}.
\begin{align}
    \bm{x}_{k+1} & = (2 A^\top A + \rho I)^{-1} (2 A^\top \bm{y} - \bm{p}_k + \rho \bm{z}_k)
    \\
    \bm{z}_{k+1} & = \mathcal{T}_{\lambda/\rho} \left( \bm{x}_{k+1} + \frac{\bm{p}_k}{\rho} \right)
    \\
    \bm{p}_{k+1} & = \bm{p}_k + \rho (\bm{x}_{k+1} - \bm{z}_{k+1})
\end{align}

\subsection{FISTA}

式 \eqref{eq:regularization_sparse_l1_objective} の
評価関数の最小化のために考えられたアルゴリズムの 1 つに
FISTA (Fast Iterative Shrinkage-Thresholding Algorithm) \cite{Beck2009}
\index{Fast Iterative Shrinkage-Thresholding Algorithm}
\index{FISTA|see{Fast Iterative Shrinkage-Thresholding Algorithm}}
がある.

FISTA においては,以下のような一般化した評価関数を考える \cite{Beck2009}.
\begin{equation}
    F(\bm{x}) \equiv f(\bm{x}) + g(\bm{x})
\end{equation}
ここで,
関数 $f : \setR^n \to \setR$ は凸関数であり,
連続微分可能で,
勾配が
\begin{equation}
    \|\nabla f(\bm{x}) - \nabla f(\bm{y})\|_2 \le L \|\bm{x} - \bm{y}\|_2
\end{equation}
のように定数 $L > 0$ で Lipschitz 連続になっているものとする.
また,関数 $g : \setR^n \to \setR$ は凸関数であり,
微分可能とは限らない連続関数である.
そして,関数 $F$ は $\setR^n$ 上に最小値を持つものとする.

\begin{algorithm}[t]
    \caption{FISTA \cite{Beck2009}}
    \label{alg:regularization_fista-with-constant-stepsize}
    \begin{algorithmic}[1]
        \Procedure{FISTA}{$\bm{x}_0$, $L$}
        \State $\bm{y}_1 \gets \bm{x}_0$
        \State $t_1 \gets 1$
        \State $k \gets 1$
        \Loop
        \State $\bm{x}_k \gets \bm{p}_L(\bm{y}_k)$
        \State $t_{k+1} \gets \frac{\displaystyle 1 + \sqrt{1 + 4 t_k^2}}{\displaystyle 2}$
        \State $\bm{y}_{k+1} \gets \bm{x}_k + \frac{\displaystyle t_k - 1}{\displaystyle t_{k+1}} (\bm{x}_k - \bm{x}_{k-1})$
        \EndLoop
        \EndProcedure
    \end{algorithmic}
\end{algorithm}

この評価関数について,点 $\bm{y} \in \setR^n$ 周りで近似する関数
\begin{equation}
    Q_L(\bm{x}, \bm{y}) \equiv f(\bm{y})
    + \nabla f(\bm{y})^\top (\bm{x} - \bm{y})
    + \frac{L}{2} \|\bm{x} - \bm{y}\|_2^2 + g(\bm{x})
\end{equation}
を考え,それが最小になる点を求める関数
\begin{equation}
    \bm{p}_L(\bm{y}) \equiv \argmin_{\bm{x}} Q_L(\bm{x}, \bm{y})
\end{equation}
を定義する \cite{Beck2009}.
この関数は
\begin{equation}
    \bm{p}_L(\bm{y}) = \argmin_{\bm{x}} \left(
    g(\bm{x}) + \frac{L}{2} \left\|\bm{x}
    - \left( \bm{y} - \frac{1}{L} \nabla f(\bm{y}) \right) \right\|_2^2
    \right)
\end{equation}
とも書ける \cite{Beck2009}.
この関数を用いて
Algorithm \ref{alg:regularization_fista-with-constant-stepsize}
のように解 $\bm{x}_k$ を更新していく.

% !TEX root = ../../main.tex
%

\section{Total variation 正則化}
\index{Total variation せいそくか@Total variation 正則化}
\index{TVせいそくか@TV 正則化|see{Total variation 正則化}}

何らかの領域における分布
\footnote{2 次元や 3 次元の領域上の分布だけでなく,%
    1 次元の線分上の分布や,3 次元空間上に存在する曲面上の分布でも良い.}
を示すベクトル $\bm{x}$ について,
微分値を算出する行列 $D$ を用いた正則化項 $\|D \bm{x}\|_1$ による正則化は
total variation 正則化と呼ばれる.
このとき,評価関数は
\begin{equation}
    E_{\lambda}(\bm{x}) \equiv \|A \bm{x} - \bm{y}\|_2^2 + \lambda \|D \bm{x}\|_1
\end{equation}
のようになる.
本節ではこのような評価関数の最小化について説明する.

\subsection{交互方向乗数法}
\index{こうごほうこうじょうすうほう@交互方向乗数法}

評価関数の最小化を
\begin{align}
    \text{minimize} \hspace{1em} & \|A \bm{x} - \bm{y}\|_2^2 + \lambda \|\bm{d}\|_1 \\
    \text{s.t.} \hspace{1em}     & D \bm{x} - \bm{d} = \bm{0}
\end{align}
と書き換えることで,
交互方向乗数法(\ref{sec:optimization_admm} 節)により
評価関数を最小化するような解 $\bm{x}$ を求めることができる \cite{Boyd2010}
\footnote{文献 \cite{Boyd2010} においては「Generalized Lasso」というタイトルで記載され,%
    行列 $D$ は微分に関係ない行列も含めている.}.

このとき,拡張ラグランジュ関数は
\begin{equation}
    L_{\rho}(\bm{x}, \bm{d}, \bm{p}) \equiv
    \|A \bm{x} - \bm{y}\|_2^2 + \lambda \|\bm{d}\|_1
    + \bm{p}^\top (D \bm{x} - \bm{d})
    + \frac{\rho}{2} \|D \bm{x} - \bm{d}\|_2^2
\end{equation}
となる.

$\bm{x}$ の更新式は
\begin{align}
    \bm{x}_{k+1}
     & = \argmin_{\bm{x}} L_{\rho}(\bm{x}, \bm{d}_k, \bm{p}_k)
    \notag                                                                          \\
     & = \argmin_{\bm{x}} \left( \|A \bm{x} - \bm{y}\|_2^2 + \lambda \|\bm{d}_k\|_1
    + \bm{p}_k^\top (D \bm{x} - \bm{d}_k)
    + \frac{\rho}{2} \|D \bm{x} - \bm{d}_k\|_2^2 \right)
    \notag                                                                          \\
     & = \argmin_{\bm{x}} \left( \|A \bm{x} - \bm{y}\|_2^2
    + \bm{p}_k^\top (D \bm{x} - \bm{d}_k)
    + \frac{\rho}{2} \|D \bm{x} - \bm{d}_k\|_2^2 \right)
\end{align}
となる.
行列 $D$ が列フルランクであればこの最小化を
\begin{align}
    \bm{x}_{k+1}
     & = (2 A^\top A + \rho D^\top D)^{-1} (2 A^\top \bm{y} - D^\top \bm{p}_k + \rho D^\top \bm{d}_k)
\end{align}
で実行できる.

$\bm{d}$ の更新式は
\begin{align}
    \bm{d}_{k+1}
     & = \argmin_{\bm{d}} L_{\rho}(\bm{x}_{k+1}, \bm{d}, \bm{p}_k)
    \notag                                                                                \\
     & = \argmin_{\bm{d}} \left( \|A \bm{x}_{k+1} - \bm{y}\|_2^2 + \lambda \|\bm{d}\|_1
    + \bm{p}_k^\top (D \bm{x}_{k+1} - \bm{d})
    + \frac{\rho}{2} \|D \bm{x}_{k+1} - \bm{d}\|_2^2 \right)
    \notag                                                                                \\
     & = \argmin_{\bm{d}} \left( \lambda \|\bm{d}\|_1
    + \bm{p}_k^\top (D \bm{x}_{k+1} - \bm{d})
    + \frac{\rho}{2} \|D \bm{x}_{k+1} - \bm{d}\|_2^2 \right)
    \notag                                                                                \\
     & = \argmin_{\bm{d}} \left( \lambda \|\bm{d}\|_1
    + \frac{\rho}{2} \left\| D \bm{x}_{k+1} - \bm{d} + \frac{\bm{p}_k}{\rho} \right\|_2^2 \right)
    \notag                                                                                \\
     & = \mathcal{T}_{\lambda/\rho} \left( D \bm{x}_{k+1} + \frac{\bm{p}_k}{\rho} \right)
\end{align}
となる.
(最後の変形は \ref{sec:regularization_shrinkage-operator} 節を参照.)

$\bm{p}$ の更新式まで含めると以下のようになる
\footnote{文献 \cite{Boyd2010} と定式化のときの係数が異なるため,%
    最終的な更新式の係数が異なっている.}.
\begin{align}
    \bm{x}_{k+1}
     & = (2 A^\top A + \rho D^\top D)^{-1} (2 A^\top \bm{y} - D^\top \bm{p}_k + \rho D^\top \bm{d}_k)
    \\
    \bm{d}_{k+1}
     & = \mathcal{T}_{\lambda/\rho} \left( D \bm{x}_{k+1} + \frac{\bm{p}_k}{\rho} \right)
    \\
    \bm{p}_{k+1}
     & = \bm{p}_k + \rho (D \bm{x}_{k+1} -\bm{d}_{k+1})
\end{align}

\subsection{Split Bregman 法}

TV 正則化の評価関数を最小化する手法の 1 つに
Split Bregman 法 \cite{Goldstein2009}
\index{Split Bregman ほう@Split Bregman 法}
がある.

Split Bregman 法について説明するにあたって,
まず次のような等号制約条件付きの最適化問題を考える
\cite{Goldstein2009}
\footnote{文字の重複を防ぐため,文献 \cite{Goldstein2009} における文字とは変えている.}.
\begin{align}
    \text{minimize} \hspace{1em} & E(\bm{u})         \\
    \text{s.t.} \hspace{1em}     & G \bm{u} = \bm{b}
\end{align}
ここで,$\bm{u}$ は最適化の変数で,$G$ は線型作用素,$\bm{b}$ は定数のベクトルであり,
関数 $E(\bm{u})$ は凸関数とする.
関数 $E(\bm{u})$ は微分できなくても良い.
文献 \cite{Bregman1967} で提案された Bregman 反復を利用することで,
この最適化問題が以下の反復により解けることが示されている
\cite{Goldstein2009}.
\begin{align}
    \bm{u}_{k+1} & = \argmin_{\bm{u}} \left( E(\bm{u}) - \bm{p}_k^\top (\bm{u} - \bm{u}_k)
    + \frac{\mu}{2} \|G \bm{u} - \bm{b}\|_2^2 \right)
    \\
    \bm{p}_{k+1} & = \bm{p}_k - \mu G^\top (G \bm{u}_{k+1} - \bm{b})
\end{align}
ここで,$\mu$ は正の定数である.
この反復は,以下のように行うこともできる
\cite{Goldstein2009}.
\begin{align}
    \bm{u}_{k+1} & = \argmin_{\bm{u}} \left( E(\bm{u}) + \frac{\mu}{2} \|G \bm{u} - \bm{b}_k\|_2^2 \right)
    \\
    \bm{b}_{k+1} & = \bm{b}_k + \bm{b} - G \bm{u}_k
\end{align}

Split Bregman 法では,TV 正則化における評価関数の最小化を以下のような最適化問題に置き換えることで,
上述の反復法を適用する.
\begin{align}
    \text{minimize} \hspace{1em} & \|A \bm{x} - \bm{y}\|_2^2 + \lambda \|\bm{d}\|_1 \\
    \text{s.t.} \hspace{1em}     & \bm{d} = D \bm{x}
\end{align}
このとき,
\begin{align}
    \bm{u} & = \begin{pmatrix}
                   \bm{x} \\ \bm{d}
               \end{pmatrix}
    ,      &
    G      & = \begin{pmatrix}
                   D & -I
               \end{pmatrix}
    ,      &
    \bm{b} & = \bm{0}
\end{align}
とすることで,以下の反復法が得られる.
\begin{align}
    (\bm{x}_{k+1}, \bm{d}_{k+1}) & =
    \argmin_{\bm{x}, \bm{d}} \left( \|A \bm{x} - \bm{y}\|_2^2 + \lambda \|\bm{d}\|_1
    + \frac{\mu}{2} \|D \bm{x} - \bm{d} - \bm{b}_k\|_2^2
    \right)
    \\
    \bm{b}_{k+1}                 & = \bm{b}_k - D \bm{x}_k + \bm{d}
\end{align}
さらに,$\bm{x}$, $\bm{d}$ に関する反復を以下の 2 つに分割する.
\begin{align}
    \bm{x}_{k+1} & =
    \argmin_{\bm{x}} \left( \|A \bm{x} - \bm{y}\|_2^2 + \lambda \|\bm{d}_k\|_1
    + \frac{\mu}{2} \|D \bm{x} - \bm{d}_k - \bm{b}_k\|_2^2
    \right)
    \\
    \bm{d}_{k+1} & =
    \argmin_{\bm{d}} \left( \|A \bm{x}_{k+1} - \bm{y}\|_2^2 + \lambda \|\bm{d}\|_1
    + \frac{\mu}{2} \|D \bm{x}_{k+1} - \bm{d} - \bm{b}_k\|_2^2
    \right)
\end{align}

ここで,$\bm{x}$ に関する反復で最小化する関数を
\begin{equation}
    f(\bm{x}) \equiv \|A \bm{x} - \bm{y}\|_2^2 + \lambda \|\bm{d}_k\|_1
    + \frac{\mu}{2} \|D \bm{x} - \bm{d}_k - \bm{b}_k\|_2^2
\end{equation}
とおくと,その勾配は
\begin{align}
    \nabla f(\bm{x})
     & =
    2 A^\top (A \bm{x} - \bm{y}) + \mu D^\top (D \bm{x} - \bm{d}_k - \bm{b}_k)
    \notag \\
     & =
    (2 A^\top A + \mu D^\top D) \bm{x}
    - (2 A^\top \bm{y} + \mu D^\top \bm{d}_k + \mu D^\top \bm{b}_k)
\end{align}
となり,Hessian は
\begin{align}
    \nabla^2 f(\bm{x})
     & =
    2 A^\top A + \mu D^\top D
\end{align}
となる.
$D$ が列フルランクであれば
\footnote{ここでは $A$ を列フルランクにしても良いが,%
    スパース最適化を適用する場合 $A$ が横長の行列になっていて列フルランクを満たしようがない場合もあるため,%
    $D$ を列フルランクにする想定で説明している.},
Hessian は正定値行列になり,
関数 $f$ は強凸関数となる.
よって,関数 $f$ が最小となる $\bm{x}$ は $\nabla f(\bm{x}) = \bm{0}$ となるような $\bm{x}$ であり,
$\bm{x}$ の更新式は以下のようになる.
\begin{equation}
    \bm{x}_{k+1} =
    (2 A^\top A + \mu D^\top D)^{-1}
    (2 A^\top \bm{y} + \mu D^\top \bm{d}_k + \mu D^\top \bm{b}_k)
\end{equation}

また,$\bm{d}$ の更新は収縮演算子を用いることで
\begin{equation}
    \bm{d}_{k+1} = \mathcal{T}_{\lambda / \mu} (D\bm{x}_{k+1} - \bm{b}_k)
\end{equation}
と書ける.

以上をまとめると,以下の更新式が得られる
\footnote{実は,交互方向乗数法と変数が定数倍異なっているだけで,同じ更新式になっている.}.
\begin{align}
    \bm{x}_{k+1}
     & = (2 A^\top A + \mu D^\top D)^{-1}
    (2 A^\top \bm{y} + \mu D^\top \bm{d}_k + \mu D^\top \bm{b}_k)
    \\
    \bm{d}_{k+1}
     & = \mathcal{T}_{\lambda / \mu} (D\bm{x}_{k+1} - \bm{b}_k)
    \\
    \bm{b}_{k+1}
     & = \bm{b}_k - D \bm{x}_k + \bm{d}
\end{align}

