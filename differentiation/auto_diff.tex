% !TEX root = ../main.tex
%

\chapter{自動微分}

自動微分 (automatic differentiation) では,
関数 $\bm{f} : \setR^m \to \setR^n$ の
関数値 $\bm{f}(\bm{x})$ を求めるために必要な演算をもとに
Jacobian $\partial \bm{f} / \partial \bm{x}$ を自動で算出する
\cite{Kubota1998}.

自動微分では,微分/偏微分における合成関数の微分公式
\begin{equation}
    \frac{\partial \bm{f}(\bm{g}(\bm{x}))}{\partial \bm{x}}
    = \frac{\partial \bm{f}(\bm{g}(\bm{x}))}{\partial \bm{g}(\bm{x})}
    \frac{\partial \bm{g}(\bm{x})}{\partial \bm{x}}
    \label{eq:differentiation_auto-diff_chain-rule}
\end{equation}
を利用する.
右辺値を右側から計算していくか,左側から計算していくかにより,
それぞれ前進型自動微分 (forward-mode automatic differentiation),
後退型自動微分 (backward-mode automatic differentiation) と呼び分けられる.

\section{前進型自動微分}

前進型自動微分では,
式 \eqref{eq:differentiation_auto-diff_chain-rule} における
$\bm{g}(\bm{x})$ の Jacobian が分かっている状態から
$\bm{f}(\bm{g}(\bm{x}))$ の Jacobian を算出する.

例えば,$\sin(\cos(x))$ を計算する場合,次のようにする.

\begin{enumerate}
    \item 変数値 $x$ と微分係数 $1$ から計算を始める.
    \item 中間的な関数値 $y = \cos{x}$ を算出するのに併せて微分係数 $y' = -\sin{x}$ を算出する.
    \item 関数値 $z = \sin{y}$ を算出するのに併せて微分係数 $z' = y' \cos{y}$ を算出する.
\end{enumerate}

この仕組みでは,
各変数に対して微分係数用の領域を用意して計算を進めていく.

\section{後退型自動微分}

前進型自動微分では,
式 \eqref{eq:differentiation_auto-diff_chain-rule} における
$\bm{f}(\bm{y})$ の Jacobian が分かっている状態から
$\bm{f}(\bm{g}(\bm{x}))$ の Jacobian を算出する.

例えば,$\sin(\cos(x))$ を計算する場合,次のようにする.

\begin{enumerate}
    \item 中間的な関数値 $y = \cos{x}$ を算出し,$\cos$ の計算をしたことを記録する.
    \item 関数値 $z = \sin{y}$ を算出し,$\sin$ の計算をしたことを記録する.
    \item 微分係数 $1$ から計算を始め,$1 \cdot dz/dy = \cos{y}$ を計算する.
    \item $1 \cdot dz/dy \cdot dy/dx = \cos{y} \cdot (-\sin{x})$ を計算する.
\end{enumerate}
