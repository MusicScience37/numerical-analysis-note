% !TEX root = ../main.tex
%

\section{線形方程式の反復解法}

ここでは,変数 $\bm{x} \in \setC^n$ に関する線形方程式
$A \bm{x} = \bm{b}$
($A \in \setC^{m \times n}$, $\bm{b} \in \setC^m$)
を反復的に解くアルゴリズムについて説明する.
このようなアルゴリズムは以下のような場合に役に立つ.
\begin{itemize}
    \item 行列のサイズ $m$, $n$ が大きいが,行列 $A$ が疎行列になっている場合,
          行列分解では大きな密行列ができてしまってコンピューターのメモリが足りなくなるが,
          疎行列 $A$ を疎行列のまま扱える反復解法であれば使用できる.
          このような状況は,例えば偏微分方程式の数値計算において発生する.
    \item 行列 $A$ 自身を計算するのは困難だが,
          行列 $A$ を与えられたベクトル $\bm{x}$ にかけた $A \bm{x}$ は比較的容易に計算できる場合,
          $A \bm{x}$ さえ計算できれば使用できるタイプの反復解法を使用して
          線形方程式 $A \bm{x} = \bm{b}$ を解くことができる.
          このような状況は,
          例えば \ref{sec:ode_runge-kutta_without-jacobian} 節において扱う.
\end{itemize}

\subsection{単純な反復による解法}

まず,線形方程式について単純な反復による解法の例を示す.

\begin{algorithm}[tp]
    \caption{Jacobi 法の反復 \cite{Golub2013}}
    \label{alg:matrix-computation_jacobi-iteration}
    \begin{algorithmic}
        \Procedure{JacobiIterate}{$A$, $\bm{b}$, $\bm{x}$}
        \For{$i = 1, 2, \ldots, n$}
        \State $x_i' \gets \frac{\displaystyle b_i - \sum_{j = 1}^{i - 1} A_{ij} x_j %
                - \sum_{j = i + 1}^{n} A_{ij} x_j}{\displaystyle A_{ii}}$
        \EndFor
        \State $\bm{x} \gets \bm{x}'$
        \EndProcedure
    \end{algorithmic}
\end{algorithm}

\begin{algorithm}[tp]
    \caption{Gauss-Seidel 法の反復 \cite{Golub2013}}
    \label{alg:matrix-computation_gauss-seidel-iteration}
    \begin{algorithmic}
        \Procedure{GaussSeidelIterate}{$A$, $\bm{b}$, $\bm{x}$}
        \For{$i = 1, 2, \ldots, n$}
        \State $x_i \gets \frac{\displaystyle b_i - \sum_{j = 1}^{i - 1} A_{ij} x_j %
                - \sum_{j = i + 1}^{n} A_{ij} x_j}{\displaystyle A_{ii}}$
        \EndFor
        \EndProcedure
    \end{algorithmic}
\end{algorithm}

Jacobi 法
\index{Jacobiほう@Jacobi 法}
では,
Algorithm \ref{alg:matrix-computation_jacobi-iteration}
の反復を繰り返すことで解を求める.
一方,Gauss-Seidel 法
\index{GaussSeidelほう@Gauss-Seidel 法}
では,
Algorithm \ref{alg:matrix-computation_gauss-seidel-iteration}
の反復を繰り返すことで解を求める.
Jacobi 法では前の解のみから次の解の各要素を求めるのに対し,
Gauss-Seidel 法では反復中の最新の解も使用するという違いがある.

\TODO{SOR, SSOR 法も書く.}

\subsection{Krylov 部分空間法}
\index{Krylovぶぶんくうかんほう@Krylov 部分空間法}

ここでは,Krylov 部分空間法と呼ばれる種類のアルゴリズムの例を示す.

\TODO{対称行列の CG 法と PCG 法に触れておきたい.}

行列 $A$ が対称とは限らない正方行列である場合に使用できる反復解法の例として,
BiCGstab (Algorithm \ref{alg:matrix-computation__bicgstab})
\index{BiCGstab}
\footnote{実装を意識して一部記法の変更を加えている.}
が挙げられる.
ベクトル $\bm{x}$ について $A \bm{x}$ が計算できれば使用できるアルゴリズムとなっている.

\begin{algorithm}[tp]
    \caption{BiCGstab \cite{Golub2013}}
    \label{alg:matrix-computation__bicgstab}
    \begin{algorithmic}
        \Procedure{BiCGstab}{$A \in \setR^{n \times n}, \bm{b} \in \setR^n, \bm{x}_0 \in \setR^n$}
        \State $\bm{r} \gets \bm{b} - A \bm{x}_0$
        \State $\tilde{\bm{r}}_0$ を零ベクトルでない値に設定
        \Comment 残差 $\bm{r}$ にしておけば良い.
        \State $\bm{x} \gets \bm{x}_0$
        \State $\bm{p} \gets \bm{r}$
        \State $\rho \gets \tilde{\bm{r}}_0^\top \bm{r}$
        \Loop
        \State $\bm{p}' \gets A \bm{p}$
        \State $\mu \gets \frac{\tilde{\bm{r}}_0^\top \bm{r}}{\tilde{\bm{r}}_0^\top \bm{p}'}$
        \State $\bm{s} \gets \bm{r} - \mu \bm{p}'$
        \State $\bm{s}' \gets A \bm{s}$
        \State $\omega \gets \frac{\bm{s}^\top \bm{s}'}{\|\bm{s}'\|_2^2}$
        \State $\bm{x} \gets \bm{x} + \mu \bm{p} + \omega \bm{s}$
        \State $\bm{r} \gets \bm{s} - \omega \bm{s}'$
        \If{$\|\bm{r}\|_2 < tolerance$}
        \State \Return $\bm{x}$
        \EndIf
        \State $\rho_{old} \gets \rho$
        \State $\rho \gets \tilde{\bm{r}}_0^\top \bm{r}$
        \State $\tau \gets \frac{\rho \mu}{\rho_{old} \omega}$
        \State $\bm{p} \gets \bm{r} + \tau(\bm{p} - \omega \bm{p}')$
        \EndLoop
        \EndProcedure
    \end{algorithmic}
\end{algorithm}

\subsection{Algebraic Multigrid 法}

\index{AlgebricMultigridほう@Algebric Multigrid 法}
\index{AMGほう@AMG 法|see{Algebraic Multigrid 法}}

ここで,
大規模で対称な疎行列 $A \in \setR^{n \times n}$ を係数とする
線形方程式 $A \bm{x} = \bm{b}$ の解法の 1 つである
Algebraic Multigrid (AMG) 法
\cite{Ruge1987}
について説明する.

偏微分方程式の解法においては,
対称となる領域(1 次元の弦,2 次元の水面や膜,3 次元の室内など)をグリッドに分け,
各点ごとに近傍の点との位置関係などをもとにして
係数行列 $A$ が作られる.
このような場合,方程式 $A \bm{x} = \bm{b}$ は近傍の点との関係のみを示すことになるが,
細かいグリッドにおける方程式 $A \bm{x} = \bm{b}$ と
荒いグリッドにおける方程式 $A' \bm{x}' = \bm{b}'$ を用意しておくことで,
領域全体の大まかな傾向は荒いグリッドで計算し,
細かい部分は細かいグリッドで計算するといった使い分けができるようになり,
細かいグリッドにおける詳細な解の計算を高速化することができる.
そのようにして,複数の細かさの異なるグリッドを使い分けて
最終的には細かいグリッドにおける詳細な解まで求められるようにしていくというのが
基本的な Multigrid 法の考え方となっている.
そのような細かさの異なるグリッドを実際のグリッドの形状から作るのではなく,
細かいグリッドにおける係数行列の値をもとに
荒いグリッドにおける係数行列を求めるようにして実現するのが
AMG 法である.

\begin{algorithm}[tp]
    \caption{Algebraic Multigrid (AMG) 法の準備(概要) \cite{Ruge1987,Wolters2002}}
    \label{alg:matrix-computation_amg_setup}
    \begin{algorithmic}
        \Procedure{AMGSetup}{$A$}
        \State 係数行列 $A$ のグリッドの点のインデックスの集合を $\Omega_1$ とする.
        \State $A_1 \gets A$
        \State $m \gets 1$
        \Loop
        \State $A_m$ の値をもとに $\Omega_m$ から一段荒いグリッドに使用するインデックスを選択し,
        その集合を $\Omega_{m+1}$ とする.
        \State $\Omega_{m+1}$ 上のベクトルを $\Omega_{m}$ 上のベクトルに変換する
        補間の行列 $P_{m+1}^{m}$ を生成する.
        \State $\Omega_{m}$ 上のベクトルを $\Omega_{m+1}$ 上のベクトルに変換する
        行列 $P_{m}^{m+1} = {P_{m+1}^{m}}^\top$ を用意する.
        \State $\Omega_{m+1}$ 上の係数行列 $A_{m+1} = P_{m}^{m+1} A_{m} P_{m+1}^{m}$ を算出する.
        \If{係数行列 $A_{m+1}$ が行列分解を適用できる程度に小さくなった場合}
        \State \Return
        \EndIf
        \State $m \gets m + 1$
        \EndLoop
        \EndProcedure
    \end{algorithmic}
\end{algorithm}

AMG 法では,
Algorithm \ref{alg:matrix-computation_amg_setup}
のようにして係数行列 $A$ から反復的に荒いグリッドを生成していく.
反復を停止する条件として,
文献 \cite{Wolters2002} ではグリッドの点数が 1000 を下回ることが用いられている.
1 段階荒いグリッドを求めるアルゴリズムは,
Algorithm \ref{alg:matrix-computation_amg_select-coarse-points1},
Algorithm \ref{alg:matrix-computation_amg_select-coarse-points2}
からなる.

\begin{algorithm}[tp]
    \caption{Algebraic Multigrid (AMG) 法における荒いグリッドの点の選択(ステップ 1) \cite{Ruge1987}}
    \label{alg:matrix-computation_amg_select-coarse-points1}
    \begin{algorithmic}
        \Procedure{AMGSelectCoarsePointsStep1}{$A_m$, $\Omega_m$}
        \State $C \gets \emptyset$
        \Comment{荒いグリッドの点の集合}
        \State $F \gets \emptyset$
        \Comment{細かいグリッドのみの点の集合}
        \State $U \gets \Omega_m$
        \Comment{未確認の点の集合}
        %
        \ForAll{$i \in \Omega_m$}
        \State $S_i \gets \left\{j \in \Omega_m \relmiddle| %
            \left| [A_m]_{ij} \right| \ge \theta %
            \max_{k \in \Omega_m, k \neq i} \left| [A_m]_{ik} \right| \right\}$
        \Comment{$\theta$ は $\theta \in (0, 1]$ な定数}
        \EndFor
        \ForAll{$i \in \Omega_m$}
        \State $S_i^\top \gets \left\{ j \in \Omega_m \relmiddle| i \in S_j \right\}$
        \State $\lambda_i \gets \left| S_i^\top \right|$
        \EndFor
        %
        \While{$U \neq \emptyset$}
        \State $U$ から $\lambda_i$ が最も大きい $i$ を選択する.
        \State $C \gets C \cup \{i\}$, $U \gets U \setminus \{i\}$
        \ForAll{$j \in S_i^\top \cap U$}
        \State $F \gets F \cup \{j\}$, $U \gets U \setminus \{j\}$
        \ForAll{$k \in S_j \cap U$}
        \State $\lambda_k \gets \lambda_k + 1$
        \EndFor
        \EndFor
        \ForAll{$j \in S_i \cap U$}
        \State $\lambda_j \gets \lambda_j - 1$
        \EndFor
        \EndWhile
        \EndProcedure
    \end{algorithmic}
\end{algorithm}

Algorithm \ref{alg:matrix-computation_amg_select-coarse-points1}
では,グリッドの点に対する評価値 $\lambda_i$ が高い点を荒いグリッドの点に選択していく.
その $\lambda_i$ の基準となっている集合が
\begin{equation}
    S_i \gets \left\{j \in \Omega_m \relmiddle| %
    \left| [A_m]_{ij} \right| \ge \theta %
    \max_{k \in \Omega_m, k \neq i} \left| [A_m]_{ik} \right| \right\}
\end{equation}
だが,$i, j \in \Omega_m$ が
$\left| [A_m]_{ij} \right| \ge \theta %
    \max_{k \in \Omega_m, k \neq i} \left| [A_m]_{ik} \right|$
を満たすとき $i$ は $j$ に強く接続している(strongly connected, strongly depends)という
\cite{Ruge1987}.
ここで,$\theta$ は $\theta \in (0, 1]$ を満たす定数で,
実用的には 0.25 にしておけば良い
\cite{Ruge1987}.
アルゴリズム中に出てくる $S_i^\top$ は $i$ に強く接続している点の集合となっており,
強く接続している点が多いような点は優先的に荒いグリッドに含むようになっている.
アルゴリズムの後半で $\lambda_i$ の値が更新されているが,
更新されたあとの値は
$\lambda_i = \left|S_i^\top \cap U\right| + 2 \left|S_i^\top \cap F\right|$
となっており,
荒いグリッドに選択しないことにした点の集合 $F$ に含まれる点との
関係が強い点は次の荒いグリッドへ追加する点に選択されやすいようになっている.

\begin{algorithm}[tp]
    \caption{Algebraic Multigrid (AMG) 法における荒いグリッドの点の選択(ステップ 2)%
        (\cite{Ruge1987} をもとに一部変更)}
    \label{alg:matrix-computation_amg_select-coarse-points2}
    \begin{algorithmic}
        \Procedure{AMGSelectCoarsePointsStep2}{$A_m$, $C$, $F$}
        \State $T = \emptyset$
        \While{$F \setminus T \neq \emptyset$}
        \State $i \in F \setminus T$ を任意に選択する.
        \State $T \gets T \cup \{i\}$
        \State $C_i \gets S_i \cap C$
        \Comment{$i$ に強く接続している荒いグリッド上の点}
        \State $D_i^s \gets S_i \setminus C_i$
        \Comment{$i$ に強く接続している荒いグリッドにない点}
        \State $\tilde{C}_i \gets \emptyset$
        \Comment{荒いグリッドへ追加する点の仮の置き場所}
        \ForAll{$j \in D_i^s$}
        \If{$S_j \cap C_i = \emptyset$}
        \If{$\tilde{C}_i = \emptyset$}
        \State $\tilde{C}_i \gets \{j\}$
        \State $C_i \gets C_i \cup \{j\}$, $D_i^s \gets D_i^s \setminus \{j\}$
        \State for 文の処理をやり直す.
        \Else
        \State $C \gets C \cup \{i\}$, $F \gets F \setminus \{i\}$
        \State for 文を抜けて次の $i$ の選択に移る.
        \EndIf
        \EndIf
        \EndFor
        \State $C \gets C \cup \tilde{C}_i$, $F \gets F \setminus \tilde{C}_i$
        \EndWhile
        \EndProcedure
    \end{algorithmic}
\end{algorithm}

Algorithm \ref{alg:matrix-computation_amg_select-coarse-points2}
では,
各 $i \in F$ について,各 $j \in S_i$ は $C$ にあるか,
$C_i = C \cap S_i$ の少なくとも 1 点に強く接続しているという条件を満たすように
荒いグリッドに選択する点を調整している.
この条件を満たしていると効率よく計算できたという
\cite{Ruge1987}.

荒いグリッドを求めたあとは,
補間の行列 $P_{m+1}^m$ を求める.
$P_{m+1}^m$ の値として,文献 \cite{Wolters2002} では
\begin{equation}
    [P_{m+1}^m]_{ij} =
    \begin{cases}
        1                                      & i = j \in C                    \\
        \frac{1}{\left|S_i^\top \cap C\right|} & i \in F, j \in S_i^\top \cap C \\
        0                                      & \mathrm{otherwise}
    \end{cases}
\end{equation}
が挙げられている.

\begin{algorithm}[tp]
    \caption{Algebraic Multigrid (AMG) 法による反復 \cite{Wolters2002}}
    \label{alg:matrix-computation_amg_iterate}
    \begin{algorithmic}
        \Procedure{AMGIterate}{$m$, $\bm{x}_m$, $\bm{b}_m$}
        \Comment{$m$ 番目に細かいグリッドで $A_m \bm{x}_m = \bm{b}_m$ を解くために反復する.}
        \If{最も細かいグリッドの場合}
        \State 行列分解を用いて $\bm{x}_m \gets A_m^{-1} \bm{b}_m$ とする.
        \State \Return $\bm{x}_m$
        \EndIf
        %
        \State $\bm{x}_m \gets \mathrm{SMOOTH}(A_m, \bm{x}_m, \bm{b}_m)$
        \Comment{反復解法で $\bm{x}_m$ を更新する.}
        %
        \State $\bm{r}_m \gets \bm{b}_m - A_m \bm{x}_m$
        \State $\bm{r}_{m+1} \gets P_{m}^{m+1} \bm{r}_m$
        \State $\bm{e}_{m+1} \gets \mathrm{AMGIterate}($m + 1$, \bm{0}, \bm{r}_{m+1})$
        \State $\bm{e}_m \gets P_{m+1}^{m} \bm{e}_{m+1}$
        \Comment{$A_m (\bm{x}_m + \bm{e}_m) = \bm{b}_m$ を荒いグリッドで解いたことになる.}
        \State $\bm{x}_m \gets \bm{x}_m + \bm{e}_m$
        %
        \State $\bm{x}_m \gets \mathrm{SMOOTH}(A_m, \bm{x}_m, \bm{b}_m)$
        \State \Return $\bm{x}_m$
        \EndProcedure
    \end{algorithmic}
\end{algorithm}

荒いグリッドを求めたあとは
Algorithm \ref{alg:matrix-computation_amg_iterate}
のような処理を反復的に行って解を更新する.
アルゴリズム中の SMOOTH に使用する反復解法としては
Gauss-Seidel 法(Algorithm \ref{alg:matrix-computation_gauss-seidel-iteration})
\index{GaussSeidelほう@Gauss-Seidel 法}
が挙げられる
\cite{Ruge1987,Wolters2002}.
1 回の SMOOTH で行う Gauss-Seidel 法の反復の回数は 1 回で良い
\cite{Wolters2002}.

AMG 法を既存の CG 法の前処理に使用した AMG-CG 法
\index{AMGCGほう@AMG-CG 法}
\index{CGほう@CG 法}
も考えられており,
単純な前処理よりも効率よく方程式を解くことができるという
\cite{Wolters2002}.
