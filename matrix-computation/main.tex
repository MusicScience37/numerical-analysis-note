% !TEX root = ../main.tex
%

\chapter{行列演算}

本章では,基本的な行列演算についてまとめる.

\section{記号}

この章で使用する記号を以下に示す.

\begin{explainlist}
    $\setR$ & 実数の集合 \\
    $\setC$ & 複素数の集合 \\
    $\|\bm{x}\|_2$ & ベクトル $\bm{x}$ の2-ノルム \\
    $A^*$ & 行列 $A$ のエルミート転置(随伴行列),または作用素 $A$ の随伴作用素 \\
    $A^{-*}$ & 行列 $A$ のエルミート転置の逆行列 \\
    $A^\dagger$ & 行列 $A$ の Moore-Penrose の一般化逆行列 \\
    $\diag(a_1, a_2, \ldots, a_n)$ & 対角要素に $a_1, a_2, \ldots, a_n$ を持つ対角行列 \\
\end{explainlist}

\section{Moore-Penrose の一般化逆行列}

最小二乗法を理論的に扱う際に,
Moore-Penrose の一般化逆行列がしばしば使用される.
本資料でも使用するため,以下にその定義と性質を示す.

行列 $A \in \setC^{m \times n}$ と
ベクトル $\bm{y} \in \setC^m$ に対して,
次の最適化問題を考える.
\begin{align}
    \text{minimize} \hspace{2em} & \|\bm{x}\|_2
    \notag                                                                                          \\
    \text{s.t.} \hspace{2em}     & \|A\bm{x}-\bm{y}\|_2 =  \min_{\bm{x}} \{ \|A\bm{x}-\bm{y}\|_2 \}
    \notag
\end{align}

この最適化問題の解は $\bm{x}=G\bm{y}$ のように
ベクトル $\bm{y}$ に依らない行列
$G \in \setC^{n \times m}$ を用いて表せる.
この行列 $G$ を Moore-Penrose の一般化逆行列と呼び,
$A^\dagger$ で表す \cite[定義3]{Rao1971}.
$\|A \bm{x} - \bm{y}\|_2$ を最小化するだけでは
$A$ の核空間が $\bm{0}$ 以外の要素を持つ場合に解が 1 つに定まらないが,
$\|\bm{x}\|_2$ が最小となるものを選ぶことにより,
解が 1 つに定まる.
なお,定義から,$A^\dagger \bm{y}$ は $A$ の核空間の成分を持たないことが示せる.

行列 $A$ がランク $n$ の場合,
$\|A \bm{x} - \bm{y}\|_2$ を最小化する $\bm{x}$ は唯一であり,
Moore-Penrose の一般化逆行列を用いて
$\bm{x} = A^\dagger \bm{y}$ と表される.
さらに,一般の行列 $A$ においては,
任意のベクトル $z \in \setC^n$ を用いて
\begin{equation}
    \bm{x} = A^\dagger \bm{y} + (I - A^\dagger A) \bm{z}
    \label{eq:matrix-computation_moore-penrose_general-least-squares-solution}
\end{equation}
のように最小二乗解を表すことができる \cite[定理2.3.1]{Rao1971}.
なお,$I - A^\dagger A$ は $A$ の核空間への射影行列となっている.

Moore-Penrose の一般化逆行列は,
ランクが $m$ または $n$ の場合に通常の逆行列を用いて次のように表現できる.

\begin{itemize}
    \item $A \in \setC^{m \times n}$ がランク $m$ の場合
          \begin{equation}
              A^\dagger = A^* (A A^*)^{-1}
          \end{equation}
    \item $A \in \setC^{m \times n}$ がランク $n$ の場合
          \begin{equation}
              A^\dagger = (A^* A)^{-1} A^*
          \end{equation}
\end{itemize}

\section{QR 分解}

行列 $A \in \setC^{m \times n}$ が
$m \ge n$ かつランク $n$ の場合に次のような分解を行うことができ,
QR 分解と呼ばれる.

\begin{equation}
    A = Q
    \begin{pmatrix}
        R \\ O
    \end{pmatrix}
\end{equation}

ここで,$Q \in \setC^{m \times m}$ はユニタリ行列であり,
$R \in \setC^{n \times n}$ は正則な上三角行列である.

行列 $Q$ を $Q = (Q_1, Q_2)$ のように
行列 $Q_1 \in \setC^{m \times n}$ と
行列 $Q_2 \in \setC^{m \times (n-m)}$ に分割すると
$A = Q_1 R$ となり,
$A^\dagger = R^{-1} Q_1^*$のように
Moore-Penrose の一般化逆行列を求められる.

\section{特異値分解}

続いて,特異値分解についても触れておく.
特異値分解には複数の定義があるが,
ここでは行列 $A \in \setC^{m \times n}$ の特異値分解は
\begin{equation}
    A = U
    \begin{pmatrix}
        \Sigma & O \\
        O      & O
    \end{pmatrix}
    V^*
\end{equation}
とする.
ここで,$\Sigma \in \setR^{r \times r}$ は
正数の対角要素による対角行列で,
$U \in \setC^{m \times m}$ と
$V \in \setC^{n \times n}$ はユニタリ行列とする.

QR 分解と同様に
$U = (U_1, U_2)$,
$U_1 \in \setC^{m \times r}$,
$U_2 \in \setC^{m \times (m-r)}$,
$V = (V_2, V_2)$,
$V_1 \in \setC^{n \times r}$,
$V_2 \in \setC^{n \times (n-r)}$ のように
行列を分割すると,
$A = U_1 \Sigma V_1^*$となり,
$A^\dagger = V_1 \Sigma^{-1} U_1^*$ が示せる.
