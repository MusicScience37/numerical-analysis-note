% !TEX root = ../main.tex
%

\section{QR 分解}

\index{QRぶんかい@QR 分解}
行列 $A \in \setC^{m \times n}$ が
$m \ge n$ かつランク $n$ の場合に次のような分解を行うことができ,
QR 分解と呼ばれる.

\begin{equation}
    A = Q
    \begin{pmatrix}
        R \\ O
    \end{pmatrix}
\end{equation}

ここで,$Q \in \setC^{m \times m}$ はユニタリ行列であり,
$R \in \setC^{n \times n}$ は正則な上三角行列である.

行列 $Q$ を $Q = (Q_1, Q_2)$ のように
行列 $Q_1 \in \setC^{m \times n}$ と
行列 $Q_2 \in \setC^{m \times (n-m)}$ に分割すると
$A = Q_1 R$ となり,
$A^\dagger = R^{-1} Q_1^*$のように
Moore-Penrose の一般化逆行列を求められる.

\section{特異値分解}

\index{とくいちぶんかい@特異値分解}
\index{Singular Value Decomposition|see{特異値分解}}
続いて,特異値分解についても触れておく.
特異値分解には複数の定義があるが,
ここでは行列 $A \in \setC^{m \times n}$ の特異値分解は
\begin{equation}
    A = U
    \begin{pmatrix}
        \Sigma & O \\
        O      & O
    \end{pmatrix}
    V^*
\end{equation}
とする.
ここで,$\Sigma \in \setR^{r \times r}$ は
正数の対角要素による対角行列で,
$U \in \setC^{m \times m}$ と
$V \in \setC^{n \times n}$ はユニタリ行列とする.

QR 分解と同様に
$U = (U_1, U_2)$,
$U_1 \in \setC^{m \times r}$,
$U_2 \in \setC^{m \times (m-r)}$,
$V = (V_2, V_2)$,
$V_1 \in \setC^{n \times r}$,
$V_2 \in \setC^{n \times (n-r)}$ のように
行列を分割すると,
$A = U_1 \Sigma V_1^*$となり,
$A^\dagger = V_1 \Sigma^{-1} U_1^*$ が示せる.
