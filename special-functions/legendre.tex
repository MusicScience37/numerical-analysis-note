% !TEX root = ../main.tex
%

\section{Legendre 関数}\label{sec:special-function_legendre-function}

\index{Legendre かんすう@Legendre 関数}
Legendre 関数は,$n=0,1,\ldots$ について次の式で表される
\cite[Section 5.2]{Morse1953}.
\begin{equation}
    P_n(x) = \frac{1}{2^n n!} \frac{d^n}{dx^n} (x - 1)^n
\end{equation}

Legendre 関数は $n$ 次多項式で,区間 $[-1, 1]$ において次のような直交性を持つ
\cite[Section 6.3]{Morse1953}.
\begin{equation}
    \int_{-1}^{1} P_n(x) P_m(x) = \frac{2}{2n + 1} \delta_{nm}
\end{equation}
さらに,$n$ 次の Legendre 関数は $n-1$ 次以下の任意の多項式 $f(x)$ と次のように直交する.
\begin{equation}
    \int_{-1}^{1} P_n(x) f(x) = 0
\end{equation}

計算には次のような公式を用いる\cite[Section6.3]{Morse1953}.
\begin{gather}
    P_{n+1}(x) = \frac{2n+1}{n+1} x P_n(x) - \frac{n}{n+1} P_{n-1}(x) \\
    (1-x^2) P_n'(x) = n(P_{n-1}(x) - x P_n(x))
\end{gather}
