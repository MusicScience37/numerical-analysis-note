% !TEX root = ../main.tex
%

\chapter{1 次元制約なし最適化}

1 次元の変数 $x \in \setR$ における関数 $f : \setR \to \setR$ の最小化においては,
比較的簡単なアルゴリズムが存在する.

\section{黄金比探索}\label{sec:optimization_golden-section-search}

黄金比探索 (golden section search, \cite[10.2]{Press2007}) では,
区間を黄金比で分割していきながら最適解を探索する.

まず,区間 $[a,b]$ に対して中間点 $c$ を
\begin{equation}
    \frac{c - a}{b - a} = \omega
\end{equation}
の比でとる($0 < \omega < 1/2$).そして,$c$ と対称な位置に点 $d$ をとる.つまり,
\begin{equation}
    \frac{b - d}{b - a} = \omega
\end{equation}
とする.このとき,
\begin{equation}
    \frac{d - a}{b - a} = 1 - \frac{b - d}{b - a} = 1 - \omega
\end{equation}
である.ここで,
\begin{itemize}
    \item $f(c) < f(d)$ となった場合,最適解を探索する区間を $[a, b]$ から $[a, d]$ に更新する.
    \item $f(c) > f(d)$ となった場合,最適解を探索する区間を $[a, b]$ から $[c, b]$ に更新する.
\end{itemize}
のようにし,更新後の区間における中間点の相対位置が
区間 $[a, b]$ における点 $c$ と変わらないようにするため以下の方程式を立てる.
\begin{align}
    \frac{d - c}{d - a} &= \omega, &
    \frac{d - c}{b - c} &= \omega
\end{align}
$b - a$ に対する比を用いることで,どちらも
\begin{align}
    \frac{1 - 2\omega}{1 - \omega} &= \omega
\end{align}
と変形でき,これを $0 < \omega < 1/2$ の制限のもとで解くと
\begin{equation}
    \omega = \frac{3 - \sqrt{5}}{2} \approx 0.3819660112501052
\end{equation}
となる.
なお,このアルゴリズムの区間の分割において,以下のように黄金比が登場する.
\begin{equation}
    \frac{b - a}{b - c} = \frac{1}{1 - \omega} = \frac{1 + \sqrt{5}}{2} \approx 1.618033988749895
\end{equation}
