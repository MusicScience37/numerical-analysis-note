% !TEX root = ../main.tex
%

\subsection{勾配を用いた最適化}

ここでは,勾配を用いた最適化アルゴリズムをまとめる.

\begin{algorithm}[tp]
    \caption{勾配による最適化}
    \label{optimization_unconstrained_descent-methods_general-descent-method}
    \begin{algorithmic}
        \Procedure{DescentMethod}{$f, \bm{x}_0$}
            \For{$i = 1,2,\ldots$}
                \State 更新方向 $\bm{d}_i \in \setR^n$ を算出する
                \State 直線探索により更新方向に掛ける係数 $t_i$ を決定する
                \Comment{通常 $f(\bm{x}_{i-1} + t_i \bm{d}_i) < f(\bm{x}_{i-1})$ とする}
                \State $\bm{x}_i \gets \bm{x}_{i-1} + t_i \bm{d}_i$
                \If{終了条件を満たしている}
                    \State \Return $\bm{x}_i$
                \EndIf
            \EndFor
        \EndProcedure
    \end{algorithmic}
\end{algorithm}

勾配を用いた最適化アルゴリズムでは,
一般に
Algorithm \ref{optimization_unconstrained_descent-methods_general-descent-method}
のような手順で反復的に最適化を進めていく.
更新方向の算出方法により,
最急降下法,Newton 法,共役勾配法のような様々なアルゴリズムが存在する.

\subsubsection{直線探索}

まずは,直線探索の方法をまとめる.
直線探索では
$\nabla f(\bm{x}_{i-1})^\top \bm{d}_i < 0$ となっている
(つまり,$\bm{d}_i$ は目的関数が減少する方向になっている)
ことを前提とする.

直線探索の方法として,以下のような方法が挙げられる.

\begin{itemize}
    \item 厳密直線探索
    \item Backtracking line search
\end{itemize}

厳密直線探索は,更新後の目的関数の値
$f(\bm{x}_{i-1} + t_i \bm{d}_i)$
が最小となる $t_i$ を探索する.

\begin{algorithm}[tp]
    \caption{Backtracking Line Search \cite[Section 9.2]{Boyd2004}}
    \label{optimization_unconstrained_descent-methods_BacktrackingLineSearch}
    \begin{algorithmic}
        \Procedure{BacktrackingLineSearch}{$f, \bm{x}_{i-1}, \bm{d}_i$}
            \State $t_i \gets 1$
            \While{$f(\bm{x}_{i-1} + t_i \bm{d}_i) > f(\bm{x}_{i-1}) + \alpha t_i \nabla f(\bm{x}_{i-1})^\top \bm{d}_i$}
                \State $t_i \gets \beta t_i$
            \EndWhile
        \EndProcedure
    \end{algorithmic}
\end{algorithm}

\begin{figure}[tp]
    \centering
    \includegraphics[width=0.7\linewidth]{optimization/Armijo-rule-image.pdf}
    \caption{Armijo の条件(式\eqref{optimization_unconstrained_descent-methods_Armijo-rule})のイメージ}
    \label{optimization_unconstrained_descent-methods_Armijo-rule-image}
\end{figure}

Backtracking Line Search \cite[Section 9.2]{Boyd2004} は
Armijo の条件 \cite[Section 7.5]{Luenberger2003}
\begin{equation}
    f(\bm{x}_{i-1} + t_i \bm{d}_i) \le f(\bm{x}_{i-1}) + \alpha t_i \nabla f(\bm{x}_{i-1})^\top \bm{d}_i
    \label{optimization_unconstrained_descent-methods_Armijo-rule}
\end{equation}
を利用する.
ここで,$\alpha$ は $\alpha \in (0,1)$ を満たす定数であり,
Armijo の条件は,
図\ref{optimization_unconstrained_descent-methods_Armijo-rule-image}のように
十分小さい $t_i$ を選択するための条件となっている.
Backtracking Line Search では,
$\alpha \in (0, 1/2)$ とし,
Algorithm \ref{optimization_unconstrained_descent-methods_BacktrackingLineSearch}
のように
$t_i$ を初期値 1 から $\beta \in (0, 1)$ 倍していき,
式 \eqref{optimization_unconstrained_descent-methods_Armijo-rule} を満たすものを探索する.
一般に,パラメータ $\alpha$, $\beta$ は
$\alpha \in [0.01, 0.3]$, $\beta \in [0.1, 0.8]$ の範囲で設定される
\cite[Section 9.2]{Boyd2004}.

\subsubsection{最急降下法}

最急降下法では,更新方向を $\bm{d}_i = -\nabla f(\bm{x}_{i-1})$ とする.
確実に目的関数の減少する方向を示しており,
ここで示す他のアルゴリズムよりも更新方向の算出が簡単である.
目的関数が強凸関数である場合において,
最適解への収束が証明されている
\cite[Section 9.3.1]{Boyd2004}.

\subsubsection{Newton 法}

Newton 法では,
目的関数が狭義凸関数である(つまり,Hessian $\nabla^2 f(\bm{x}_{i-1})$ が正定値である)場合を対象とし,
更新方向を
$\bm{d}_i = -\nabla^2 f(\bm{x}_{i-1})^{-1} \nabla f(\bm{x}_{i-1})$
とする.
$\nabla^2 f(\bm{x}_{i-1})$ が正定値である場合,
$\nabla^2 f(\bm{x}_{i-1})^{-1}$ も正定値になる
\footnote{%
対称行列 $A$ が正定値である場合,$A$ の固有値は正の実数である.%
$A$ は固有値分解により $A=VDV^\top$ ($D$ は固有値による対角行列,$V$ は直交行列)と書くことができるため,%
$A^{-1} = VD^{-1}V^\top$ となる.%
よって,$A^{-1}$ の固有値も全て正の実数であり,%
$A^{-1}$ は正定値である.%
}
ため,
最適解でない $\bm{x}_{i-1}$ においては
$\nabla f(\bm{x}_{i-1})^\top \bm{d}_i = -\nabla f(\bm{x}_{i-1})^\top \nabla^2 f(\bm{x}_{i-1})^{-1} \nabla f(\bm{x}_{i-1}) < 0$
となり,目的関数が減少する方向になっていることを確認できる.
Newton 法の収束性については \cite[Section 9.5.3, 9.6.4]{Boyd2004} にて議論されている.

\subsubsection{共役勾配法}

共役勾配法では,
\begin{align}
    \bm{d}_1 &= -\nabla f(\bm{x}_{i-1}) \\
    \bm{d}_i &= -\nabla f(\bm{x}_{i-1}) + \gamma_i \bm{d}_{i-1} \\
    \gamma_i &= 
        \frac{(\nabla f(\bm{x}_{i-1}) - \nabla f(\bm{x}_{i-2}))^\top \nabla f(\bm{x}_{i-1})}
        {\|\nabla f(\bm{x}_{i-2})\|_2^2}
\end{align}
のように更新方向を算出する
\cite[Section 8.6]{Luenberger2003}.
$\gamma_i$ については複数の形式があるが,
ここで示している Polak-Ribiere 法は一般により良い結果が得られるという
\cite[Section 8.6]{Luenberger2003}, \cite[Section 10.8]{Press2007}.
Newton 法では計算量の多い逆行列の計算が必要だが,
共役勾配法では計算量が変数の次元のオーダーに収まるため,
各反復の計算時間を抑えられる.
