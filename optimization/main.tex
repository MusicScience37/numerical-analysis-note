% !TEX root = ../main.tex
%

\chapter{最適化}

ここでは,最適化のアルゴリズムをまとめる.

最適化問題は一般に次のように書ける.

\begin{align}\label{optimization_general_problem}
    \text{minimize} \hspace{1em} & f(\bm{x})                 \\
    \text{s.t.} \hspace{1em}     & \bm{g}(\bm{x}) \le \bm{0} \\
                                 & \bm{h}(\bm{x}) = \bm{0}
\end{align}

ここで,
$\bm{x} \in \setR^n$,
$f : \setR^n \to \setR$,
$\bm{g} : \setR^n \to \setR^m$,
$\bm{h} : \setR^n \to \setR^r$
とする.
また,ベクトル同士の「$\le$」による比較は
全ての要素において「$\le$」の関係が成り立つことを意味する.

問題\eqref{optimization_general_problem}において,
関数$f$は目的関数,
$\bm{g}(\bm{x}) \le \bm{0}$, $\bm{h}(\bm{x}) = \bm{0}$は制約条件と呼ばれ,
制約条件を満たす中で目的関数を最小化する問題を示している.
最小化でなく最大化で定式化する場合もあるが,
ここでは最小化に統一して説明する.

\section{記号}

本章で使用する記号を以下に示す.

\begin{explainlist}
    $\setR$ & 実数の集合 \\
    $\nabla f$ & 関数 $f : \setR^n \to \setR$ の勾配 \\
    $\nabla^2 f$ & 関数 $f : \setR^n \to \setR$ の Hessian \\
    $O$ & 零行列 \\
    $A \succ o$ & 正方行列 $A$ は正定値 \\
    $A \succeq o$ & 正方行列 $A$ は半正定値 \\
\end{explainlist}

\section{基本的な定義と性質}

\begin{definition}[{\cite[Section 6.4]{Luenberger2003}},{\cite[Section 3.1.1]{Boyd2004}}]
    関数 $f : \setR^n \to \setR$ が
    $\forall \bm{x}, \bm{y} \in \setR^n$, $\forall \alpha \in [0, 1]$ に対して
    \begin{equation}
        f\left(\alpha \bm{x} + (1-\alpha) \bm{y}\right)
        \le \alpha f(\bm{x}) + (1-\alpha) f(\bm{y})
    \end{equation}
    を満たす場合,関数 $f$ を凸関数(convex function)と呼ぶ.
\end{definition}

\begin{theorem}[{\cite[Section 6.4]{Luenberger2003}},{\cite[Section 3.1.3]{Boyd2004}}]
    $C^1$ 級関数 $f : \setR^n \to \setR$ が凸関数であるための必要十分条件は
    $\forall \bm{x}, \bm{y} \in \setR^n$ に対して
    \begin{equation}
        f(\bm{y}) \ge f(\bm{x}) + \nabla f(\bm{x})^\top (\bm{y} - \bm{x})
    \end{equation}
    が成り立つことである.
\end{theorem}

\begin{theorem}[{\cite[Section 6.4]{Luenberger2003}},{\cite[Section 3.1.3]{Boyd2004}}]
    $C^2$ 級関数 $f : \setR^n \to \setR$ が凸関数であるための必要十分条件は
    $\forall \bm{x} \in \setR^n$ に対して
    \begin{equation}
        \nabla^2 f(\bm{x}) \succeq O
    \end{equation}
    が成り立つことである.
\end{theorem}

\begin{definition}[{\cite[Section 6.4]{Luenberger2003}},{\cite[Section 3.1.1]{Boyd2004}}]
    関数 $f : \setR^n \to \setR$ が
    $\forall \bm{x}, \bm{y} \in \setR^n$, $\forall \alpha \in [0, 1]$ に対して
    \begin{equation}
        f\left(\alpha \bm{x} + (1-\alpha) \bm{y}\right)
        < \alpha f(\bm{x}) + (1-\alpha) f(\bm{y})
    \end{equation}
    を満たす場合,関数 $f$ は狭義凸関数(strictly convex function)であるという.
\end{definition}

% !TEX root = ../main.tex
%

\section{制約なし最適化}

ここでは,次のような制約のない最適化問題の解法をまとめる.

\begin{align}
    \text{minimize} \hspace{1em}& f(\bm{x}) \\
    \text{s.t.} \hspace{1em}& \bm{x} \in \setR^n
\end{align}

% !TEX root = ../main.tex
%

\subsection{勾配を用いた最適化}

ここでは,勾配を用いた最適化アルゴリズムをまとめる.

\begin{algorithm}[tp]
    \caption{勾配による最適化}
    \label{optimization_unconstrained_descent-methods_general-descent-method}
    \begin{algorithmic}
        \Procedure{DescentMethod}{$f, \bm{x}_0$}
            \For{$i = 1,2,\ldots$}
                \State 更新方向 $\bm{d}_i \in \setR^n$ を算出する
                \State 直線探索により更新方向に掛ける係数 $t_i$ を決定する
                \Comment{通常 $f(\bm{x}_{i-1} + t_i \bm{d}_i) < f(\bm{x}_{i-1})$ とする}
                \State $\bm{x}_i \gets \bm{x}_{i-1} + t_i \bm{d}_i$
                \If{終了条件を満たしている}
                    \State \Return $\bm{x}_i$
                \EndIf
            \EndFor
        \EndProcedure
    \end{algorithmic}
\end{algorithm}

勾配を用いた最適化アルゴリズムでは,
一般に
Algorithm \ref{optimization_unconstrained_descent-methods_general-descent-method}
のような手順で反復的に最適化を進めていく.
更新方向の算出方法により,
最急降下法,Newton 法,共役勾配法のような様々なアルゴリズムが存在する.

\subsubsection{直線探索}

まずは,直線探索の方法をまとめる.
直線探索では
$\nabla f(\bm{x}_{i-1})^\top \bm{d}_i < 0$ となっている
(つまり,$\bm{d}_i$ は目的関数が減少する方向になっている)
ことを前提とする.

直線探索の方法として,以下のような方法が挙げられる.

\begin{itemize}
    \item 厳密直線探索
    \item Backtracking line search
\end{itemize}

厳密直線探索は,更新後の目的関数の値
$f(\bm{x}_{i-1} + t_i \bm{d}_i)$
が最小となる $t_i$ を探索する.

\begin{algorithm}[tp]
    \caption{Backtracking Line Search \cite[Section 9.2]{Boyd2004}}
    \label{optimization_unconstrained_descent-methods_BacktrackingLineSearch}
    \begin{algorithmic}
        \Procedure{BacktrackingLineSearch}{$f, \bm{x}_{i-1}, \bm{d}_i$}
            \State $t_i \gets 1$
            \While{$f(\bm{x}_{i-1} + t_i \bm{d}_i) > f(\bm{x}_{i-1}) + \alpha t_i \nabla f(\bm{x}_{i-1})^\top \bm{d}_i$}
                \State $t_i \gets \beta t_i$
            \EndWhile
        \EndProcedure
    \end{algorithmic}
\end{algorithm}

\begin{figure}[tp]
    \centering
    \includegraphics[width=0.7\linewidth]{optimization/Armijo-rule-image.pdf}
    \caption{Armijo の条件(式\eqref{optimization_unconstrained_descent-methods_Armijo-rule})のイメージ}
    \label{optimization_unconstrained_descent-methods_Armijo-rule-image}
\end{figure}

Backtracking Line Search \cite[Section 9.2]{Boyd2004} は
Armijo の条件 \cite[Section 7.5]{Luenberger2003}
\begin{equation}
    f(\bm{x}_{i-1} + t_i \bm{d}_i) \le f(\bm{x}_{i-1}) + \alpha t_i \nabla f(\bm{x}_{i-1})^\top \bm{d}_i
    \label{optimization_unconstrained_descent-methods_Armijo-rule}
\end{equation}
を利用する.
ここで,$\alpha$ は $\alpha \in (0,1)$ を満たす定数であり,
Armijo の条件は,
図\ref{optimization_unconstrained_descent-methods_Armijo-rule-image}のように
十分小さい $t_i$ を選択するための条件となっている.
Backtracking Line Search では,
$\alpha \in (0, 1/2)$ とし,
Algorithm \ref{optimization_unconstrained_descent-methods_BacktrackingLineSearch}
のように
$t_i$ を初期値 1 から $\beta \in (0, 1)$ 倍していき,
式 \eqref{optimization_unconstrained_descent-methods_Armijo-rule} を満たすものを探索する.
一般に,パラメータ $\alpha$, $\beta$ は
$\alpha \in [0.01, 0.3]$, $\beta \in [0.1, 0.8]$ の範囲で設定される
\cite[Section 9.2]{Boyd2004}.

\subsubsection{最急降下法}

最急降下法では,更新方向を $\bm{d}_i = -\nabla f(\bm{x}_{i-1})$ とする.
確実に目的関数の減少する方向を示しており,
ここで示す他のアルゴリズムよりも更新方向の算出が簡単である.
目的関数が強凸関数である場合において,
最適解への収束が証明されている
\cite[Section 9.3.1]{Boyd2004}.

\subsubsection{Newton 法}

Newton 法では,
目的関数が狭義凸関数である(つまり,Hessian $\nabla^2 f(\bm{x}_{i-1})$ が正定値である)場合を対象とし,
更新方向を
$\bm{d}_i = -\nabla^2 f(\bm{x}_{i-1})^{-1} \nabla f(\bm{x}_{i-1})$
とする.
$\nabla^2 f(\bm{x}_{i-1})$ が正定値である場合,
$\nabla^2 f(\bm{x}_{i-1})^{-1}$ も正定値になる
\footnote{%
対称行列 $A$ が正定値である場合,$A$ の固有値は正の実数である.%
$A$ は固有値分解により $A=VDV^\top$ ($D$ は固有値による対角行列,$V$ は直交行列)と書くことができるため,%
$A^{-1} = VD^{-1}V^\top$ となる.%
よって,$A^{-1}$ の固有値も全て正の実数であり,%
$A^{-1}$ は正定値である.%
}
ため,
最適解でない $\bm{x}_{i-1}$ においては
$\nabla f(\bm{x}_{i-1})^\top \bm{d}_i = -\nabla f(\bm{x}_{i-1})^\top \nabla^2 f(\bm{x}_{i-1})^{-1} \nabla f(\bm{x}_{i-1}) < 0$
となり,目的関数が減少する方向になっていることを確認できる.
Newton 法の収束性については \cite[Section 9.5.3, 9.6.4]{Boyd2004} にて議論されている.

\subsubsection{共役勾配法}

共役勾配法では,
\begin{align}
    \bm{d}_1 &= -\nabla f(\bm{x}_{i-1}) \\
    \bm{d}_i &= -\nabla f(\bm{x}_{i-1}) + \gamma_i \bm{d}_{i-1} \\
    \gamma_i &= 
        \frac{(\nabla f(\bm{x}_{i-1}) - \nabla f(\bm{x}_{i-2}))^\top \nabla f(\bm{x}_{i-1})}
        {\|\nabla f(\bm{x}_{i-2})\|_2^2}
\end{align}
のように更新方向を算出する
\cite[Section 8.6]{Luenberger2003}.
$\gamma_i$ については複数の形式があるが,
ここで示している Polak-Ribiere 法は一般により良い結果が得られるという
\cite[Section 8.6]{Luenberger2003}, \cite[Section 10.8]{Press2007}.
Newton 法では計算量の多い逆行列の計算が必要だが,
共役勾配法では計算量が変数の次元のオーダーに収まるため,
各反復の計算時間を抑えられる.


