% !TEX root = ../main.tex
%

\chapter{多次元の数値積分}

多次元領域上の積分は,
変数変換で 1 次元ずつの多重積分にし,
各次元に対して 1 次元の数値積分手法を適用すれば算出できる.
しかし,よくある領域形状(三角形,直方体,四面体など)については,
積分する領域の形状に合わせた様々な数値積分公式が研究されている
\cite{Cools1993}.

\section{三角形上の数値積分}

2 次元平面上のメッシュや,
3 次元空間内の曲面を表すメッシュなど,
三角形が必要になる場面はしばしば存在する.
そのような三角形での使用を目的とした数値積分手法の 1 つを紹介する.

三点 $\bm{a}, \bm{b}, \bm{c}$ からなる三角形
\footnote{何次元空間の三角形かは問わない.}
において,
三角形上の点 $\bm{p}$ を重心 $\bm{g}$ を用いて,
\begin{equation}
    \bm{p} = \xi_1 (\bm{a} - \bm{g}) + \xi_2 (\bm{b} - \bm{g}) + \xi_3 (\bm{c} - \bm{g}) + \bm{g}
\end{equation}
のように
$\xi_1 \ge 0$,
$\xi_2 \ge 0$,
$\xi_3 \ge 0$,
$\xi_1 + \xi_2 + \xi_3 \le 1$
なる $\xi_1$, $\xi_2$, $\xi_3$ の組で示すことができる.
この原点を中心とした座標系は
\index{barycentric coordinates}
barycentric coordinates
と呼ばれる.
文献 \cite{Laurie1982} では,
この barycentric coordinates を用いた
対称的な積分公式 \index{CUBTRI} CUBTRI を算出している.
また,CUBTRI は Gauss-Kronrod 積分公式(\ref{sec:integration_one-dim_gauss-kronrod} 節)と同様に
埋め込み型の公式であり,共通の分点を持つ複数の近似値を算出できる.

文献 \cite{Laurie1982} における積分公式は次のように計算する.
\begin{align}
    I_5[f]     & = \Delta \sum_{i=0}^2 w_5^{(i)} I^{(i)}[f] \\
    I_8[f]     & = \Delta \sum_{i=0}^5 w_8^{(i)} I^{(i)}[f] \\
    I^{(i)}[f] & = \frac{1}{6} \sum_{\substack{k = 1, 2, 3  \\ l = 1, 2, 3 \\ k \neq l}}
    f(\xi_k^{(i)}, \xi_l^{(i)})
\end{align}
ここで,
$I_5[f]$, $I_8[f]$ はそれぞれ関数 $f$ の 5, 8 次精度の積分となっている.
$\Delta$ は三角形の面積であり,
パラメータ $w_5^{(i)}$, $w_8^{(i)}$, $\xi_k^{(i)}$ は
表 \ref{table:integration_multi-dim_cubtri-parameters} のようになっている.

\begin{table}[bp]
    \caption{CUBTRI \cite{Laurie1982} のパラメータ($\phi = \sqrt{15}$, $\sigma = \sqrt{7}$)}
    \label{table:integration_multi-dim_cubtri-parameters}
    \centering
    \begin{tabular}{c|ccc}
        $i$                                                   &
        $\xi_1^{(i)}$                                         &
        $\xi_2^{(i)}$                                         &
        $\xi_3^{(i)}$                                           \\
        \hline
        0                                                     &
        $1 / 3$                                               &
        $1 / 3$                                               &
        $1 / 3$                                                 \\
        1                                                     &
        $3 / 7 + 2 \phi / 21$                                 &
        $2 / 7 - \phi / 21$                                   &
        $2 / 7 - \phi / 21$                                     \\
        2                                                     &
        $3 / 7 - 2 \phi / 21$                                 &
        $2 / 7 + \phi / 21$                                   &
        $2 / 7 + \phi / 21$                                     \\
        3                                                     &
        $4 / 9 + \phi / 9 + \sigma / 9 - \sigma \phi / 45$    &
        $5 / 18 - \phi / 18 - \sigma / 18 + \sigma \phi / 90$ &
        $5 / 18 - \phi / 18 - \sigma / 18 + \sigma \phi / 90$   \\
        4                                                     &
        $4 / 9 - \phi / 9 + \sigma / 9 + \sigma \phi / 45$    &
        $5 / 18 + \phi / 18 - \sigma / 18 - \sigma \phi / 90$ &
        $5 / 18 + \phi / 18 - \sigma / 18 - \sigma \phi / 90$   \\
        5                                                     &
        $5 / 18 - \phi / 18 - \sigma / 18 + \sigma \phi / 90$ &
        $5 / 18 + \phi / 18 - \sigma / 18 - \sigma \phi / 90$ &
        $4 / 9 + \sigma / 9$
    \end{tabular}
    \begin{tabular}{c|c}
        $i$ &
        $w_5^{(i)}$            \\
        \hline
        0   &
        $9 / 40$               \\
        1   &
        $31 / 80 - \phi / 400$ \\
        2   &
        $31 / 80 + \phi / 400$ \\
    \end{tabular}
    \begin{tabular}{c|c}
        $i$ &
        $w_8^{(i)}$                                                    \\
        \hline
        0   &
        $7137 / 62720 - 45 \sigma / 1568$                              \\
        1   &
        $-9301697 / 4695040 - 13517313 \phi / 23475200
        + 764885 \sigma / 939008 + 198763 \phi \sigma / 939008$        \\
        2   &
        $-9301697 / 4695040 + 13517313 \phi / 23475200
        + 764885 \sigma / 939008 - 198763 \phi \sigma / 939008$        \\
        3   &
        $102791225 / 59157504 + 23876225 \phi / 59157504
        - 34500875 \sigma / 59157504 - 9914825 \phi \sigma / 59157504$ \\
        4   &
        $102791225 / 59157504 - 23876225 \phi / 59157504
        - 34500875 \sigma / 59157504 + 9914825 \phi \sigma / 59157504$ \\
        5   &
        $11075 / 8064 - 125 \sigma / 288$                              \\
    \end{tabular}
\end{table}
