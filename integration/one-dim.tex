% !TEX root = ../main.tex
%

\chapter{1 次元の数値積分}

本章では,
1 次元における積分
\begin{equation}
    \int_{a}^{b} f(x) dx
\end{equation}
に対する数値積分の手法についてまとめる.

\section{Gauss 積分公式}

1 次元における積分に対する数値積分の手法の 1 つに Gauss 積分公式がある.
($a$, $b$ は $-\infty$, $\infty$ でも良い.)

Gauss 積分公式では,次のように $x_1, x_2, \ldots, x_n$ 上での関数値の重み付き平均を用いる
\cite{Mori1993}.
\begin{equation}
    \int_{a}^{b} f(x) w(x) dx = \sum_{k = 1}^n w_k f(x_k)
\end{equation}
分点 $x_1, x_2, \ldots, x_n$ と重み $w_1, w_2, \ldots, w_n$ は,
区間 $(a, b)$ と重み関数 $w(x)$ に応じて決まる.

\subsection{Gauss-Legendre 公式}

$a$, $b$ が有限な場合に利用できる Gauss-Legendre 公式では,
Legendre 関数 $P_n(x)$ の $n$ 個の零点を分点 $x_k$ に用い,
重み関数は次のように算出する\cite{Mori1993}.
\begin{equation}
    w_k = \frac{2(1 - x_k^2)}{(n P_{n-1}(x_k))^2}
\end{equation}
分点 $x_k$ の算出には Newton-Raphson 法(\ref{chap:root-finding_newton-raphson} 章)を用いることができる.
初期値には
\begin{equation}
    x_k \approx \cos{\frac{\pi (k - 0.25)}{n + 0.5}}
\end{equation}
を使用すると良い\cite{Mori1993}.
また,$s_k = -x_{n-k}, w_k = w_{n-k}$ を用いて計算量を半減させることができる.

\section{二重指数関数型公式}

1 次元の数値積分において,
Gauss 公式よりも効率の良い積分手法として,
二重指数関数型公式(Double Exponential Formula, DE Formula)が存在する
\cite[6.1 節 (b)]{Mori1993}, \cite[Section 4.5]{Press2007}.
二重指数関数型公式では,
区間 $(a, b)$ 上での積分において次の式による変数変換を行い,
区間 $(-\infty, \infty)$ 上での積分にする.
\begin{equation}
    x = \frac{1}{2}(a + b) + \frac{1}{2}(b - a) \tanh \left(\frac{\pi}{2} \sinh{t}\right)
\end{equation}
$\tanh$, $\sinh$ はどちらも指数関数で定義されるため,
名前通り二重の指数関数による積分となっている.

二重指数関数型公式は,
無限積分
\begin{equation}
    \int_{-\infty}^{\infty} f(x) dx
\end{equation}
が有限和
\begin{equation}
    h \sum_{k = -\infty}^{\infty} f(kh)
\end{equation}
でよく近似できるということに基づいている\cite[6.1 節 (b)]{Mori1993}.

\subsection{有限区間上の積分}

1 次元の有限区間上の積分
\begin{equation}
    I = \int_{a}^{b} f(x) dx
\end{equation}
を考える.変数変換
\begin{equation}
    x = \phi(t) \equiv \frac{1}{2}(a + b) + \frac{1}{2}(b - a) \tanh \left(\frac{\pi}{2} \sinh{t}\right)
\end{equation}
を適用する場合,
\begin{equation}
    \frac{dx}{dt} = \phi'(t)
    = \frac{\pi}{4} (b - a) \frac{\cosh{t}}{\cosh^2 \left(\frac{\pi}{2} \sinh{t}\right)}
\end{equation}
を用いて
\begin{equation}
    I = \int_{a}^{b} f(x) dx
    = \int_{-\infty}^{\infty} f(\phi(t)) \phi'(t)
\end{equation}
のように近似でき,
\begin{equation}
    T_h = h \sum_{k = -N}^{N} f(\phi(kh)) \phi'(kh)
\end{equation}
で近似できる.
ここで,パラメータ $c$ は $1$ か $\pi/2$ とする場合が多い\cite[Section 4.5]{Press2007}.
また,パラメータ $h$ の最適値は
\begin{equation}
    h \approx \frac{\log(2 \pi N)}{N}
\end{equation}
であり,この場合の数値積分の誤差のオーダーは
\begin{equation}
    |T_h - I| \approx \exp(-kN / \log{N})
\end{equation}
となる\cite[Section 4.5]{Press2007}.
点数 $N$ を 2 倍にすると有効桁数が約 2 倍になる.

なお,実装時はオーバーフローしないように
\begin{align}
    \phi'(t)
     & = \frac{\pi}{4} (b - a) \frac{\cosh{t}}{\cosh^2 \left(\frac{\pi}{2} \sinh{t}\right)}                                              \\
     & = \frac{\pi}{4} (b - a) \frac{4 \cosh{t}}{(\exp\left(\frac{\pi}{2} \sinh{t}\right) + \exp\left(-\frac{\pi}{2} \sinh{t}\right))^2} \\
     & = \pi (b - a) \frac{\cosh{t} \exp(-\pi \sinh{t})}{(1 + \exp(-\pi \sinh{t}))^2}
\end{align}
とすると良い\cite[Section 4.5.2]{Press2007}.

\subsection{半無限区間上の積分}

1 次元の半無限区間上の積分
\begin{equation}
    I = \int_{0}^{\infty} f(x) dx
\end{equation}
を考える.
これに対する二重指数関数型公式の変数変換は
\begin{equation}
    x = \phi(t) \equiv \exp(\pi \sinh{t})
\end{equation}
である\cite[Section 4.5.3]{Press2007}.
微分すると
\begin{equation}
    \frac{dx}{dt} = \phi'(t)
    = \pi \exp(\pi \sinh{t}) \cosh{t}
\end{equation}
となる.

半無限区間上の積分,および次節の無限区間上の積分においては,
変数変換後の変数 $t$ において区間 $[-4, 4]$ までの範囲で有限和による近似を行えば良い
\cite[4.5.3]{Press2007}.

\subsection{無限区間上の積分}

1 次元の無限区間上の積分
\begin{equation}
    I = \int_{-\infty}^{\infty} f(x) dx
\end{equation}
を考える.
これに対する二重指数関数型公式の変数変換は
\begin{equation}
    x = \phi(t) \equiv \sinh \left(\frac{\pi}{2} \sinh{t}\right)
\end{equation}
である\cite[Section 4.5.3]{Press2007}.
微分すると
\begin{equation}
    \frac{dx}{dt} = \phi'(t)
    = \frac{\pi}{2} \cosh \left(\frac{\pi}{2} \sinh{t}\right) \cosh{t}
\end{equation}
となる.
