% !TEX root = ../main.tex
%

\chapter{1 次元の数値積分}

本章では,
1 次元における積分
\begin{equation}
    \int_{a}^{b} f(x) dx
\end{equation}
に対する数値積分の手法についてまとめる.

\section{Gauss 積分公式}

1 次元における積分に対する数値積分の手法の 1 つに Gauss 積分公式がある.
($a$, $b$ は $-\infty$, $\infty$ でも良い.)

Gauss 積分公式では,次のように $x_1, x_2, \ldots, x_n$ 上での関数値の重み付き平均を用いる
\cite{Mori1993}.
\begin{equation}
    \int_{a}^{b} f(x) w(x) dx = \sum_{k = 1}^n w_k f(x_k)
\end{equation}
分点 $x_1, x_2, \ldots, x_n$ と重み $w_1, w_2, \ldots, w_n$ は,
区間 $(a, b)$ と重み関数 $w(x)$ に応じて決まる.

\subsection{Gauss-Legendre 公式}

$a$, $b$ が有限な場合に利用できる Gauss-Legendre 公式では,
Legendre 関数 $P_n(x)$ の $n$ 個の零点を分点 $x_k$ に用い,
重み関数は次のように算出する\cite{Mori1993}.
\begin{equation}
    w_k = \frac{2(1 - x_k^2)}{(n P_{n-1}(x_k))^2}
\end{equation}
分点 $x_k$ の算出には Newton-Raphson 法(\ref{chap:root-finding_newton-raphson} 章)を用いることができる.
初期値には
\begin{equation}
    x_k \approx \cos{\frac{\pi (k - 0.25)}{n + 0.5}}
\end{equation}
を使用すると良い\cite{Mori1993}.
また,$s_k = -x_{n-k}, w_k = w_{n-k}$ を用いて計算量を半減させることができる.
