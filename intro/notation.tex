% !TEX root = main.tex
%

\chapter{記号}

本書において共通で使用する記号を以下に示す.
ただし,一部の章は分野独特の表記を用いる場合があるため,
章専用の記号の一覧を持つ場合がある.

\begin{explainlist}
    $\setR$ & 実数の集合 \\
    $\setC$ & 複素数の集合 \\
    $\Im(z)$ & 複素数 $z$ の虚部 \\
    $A^*$ & 行列 $A$ のエルミート転置(随伴行列),または作用素 $A$ の随伴作用素 \\
    $A^{-*}$ & 行列 $A$ のエルミート転置の逆行列 \\
    $A^\dagger$ & 行列 $A$ の Moore-Penrose の一般化逆行列 \\
    $O$ & 零行列 \\
    $I$ & 単位行列 \\
    $\diag(a_1, a_2, \ldots, a_n)$ & 対角要素に $a_1, a_2, \ldots, a_n$ を持つ対角行列 \\
    $\bm{a}_i$ & 行列 $A$ の $i$ 列目 \\
    $[A\bm{b}]_i$ & ベクトル $A\bm{b}$ の第 $i$ 要素 \\
    $\|\bm{x}\|$ & 一般のノルム \\
    $\|\bm{x}\|_2$ & ベクトル $\bm{x}$ の2-ノルム \\
    $\nabla f$ & 関数 $f : \setR^n \to \setR$ の勾配 \\
    $\nabla^2 f$ & 関数 $f : \setR^n \to \setR$ の Hessian \\
    $\frac{\partial \bm{f}}{\partial \bm{x}}$ & 関数 $\bm{f} : \setR^n \to \setR^n$ の Jacobian \\
    $A \succ O$ & 正方行列 $A$ は正定値である \\
    $A \succeq O$ & 正方行列 $A$ は半正定値である \\
    $A \succeq B$ & $A - B \succeq O$ となる \\
    $f'(x)$ & 1 変数関数 $f : \setR \to \setR$ の微分係数 \\
    $\dot{\bm{y}}$ & 関数 $\bm{y}(t)$ の一階微分 \\
    $\ddot{\bm{y}}$ & 関数 $\bm{y}(t)$ の二階微分 \\
    $\dddot{\bm{y}}$ & 関数 $\bm{y}(t)$ の三階微分 \\
    $\ddddot{\bm{y}}$ & 関数 $\bm{y}(t)$ の四階微分 \\
    $P_n(x)$ & Legendre 関数(\ref{sec:special-function_legendre-function} 節) \\
    $n!$ & $n$ の階乗 \\
    $\delta_{ij}$ & Kronecker のデルタ \\
\end{explainlist}
